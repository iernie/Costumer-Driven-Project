% Activate the following line by filling in the right side. If for example the name of the root file is Main.tex, write
% "...root = Main.tex" if the chapter file is in the same directory, and "...root = ../Main.tex" if the chapter is in a subdirectory.
 
%!TEX root =  

\chapter{Design}

This document describes the design phase of the program, where the program architecture is established. Several critical decisions are made in this phase and the design and architecture decisions impacts the way the implementation phase proceeds as it defines how the final software system is decomposed into modules, and how these modules behave and interact with each other. Details regarding programming languages, algorithms, data structures, coding standards and other software engineering features must also be established prior to proceeding from this phase.

\section{Development Tools and Technologies} 
This section details some of the choices that were made regarding development tools and technologies for this project.

\subsection{Documents and Source Code Repositories}
For software projects of a certain scale, version control is an important technology that allows project members to work simultaneously against the same files without causing inconsistencies. Version control systems also allow for comparison with older versions, tracking changes and restoring previous copies in the case of errors.

For source code repositories and version control, \textbf{Git} was selected. Git is a distributed, open source version control system that is available for all platforms. Git is also used for hosting the \LaTeX documents that comprise this report. For other documents such as meeting minutes, agendas, status reports, time reporting and certain planning documents, Google Docs is used.


\subsection{Programming Languages}
\textbf{Java} is chosen as the primary programming language for implementing the majority of the code. Java is an object oriented programming language providing a level of abstraction appropriate for the task at hand in addition to a rich set of libraries, including the SWING library for GUI programming, and several libraries for networking. It also simplifies writing a browser plugin as major browsers such as Google Chrome and Firefox employ Javascript as 

\subsection{Databases}
For testing purposes, it has been decided on using flat file storage of privacy policy data using Java's \texttt{Serializable} interface. However, the output functionality is to be written in a generic fashion to simplify use of database systems such as MySQL, CouchDB and so forth. 

\subsection{Third Party Libraries}
 
For developing the CBR as well as P3P parsing components of the Privacy Advisor, a decision had to be made regarding the usage of third party libraries, either for components or for the entire CBR system. We basically considered two options with respect to the CBR system, the first was to use a full third party CBR system (jColibri) and the second was to use a third party system for the retrieval component of the CBR system (i.e. a k Nearest Neighbors (kNN) implementation).


\subsubsection{Third Party CBR System}
The customer (SINTEF ICT) suggested looking into an open source CBR library developed at the Universidad Complutense de Madrid.
 
jColibri is a CBR system that has been under development for well over 10 years and is a very comprehensive system allowing for database interfaces and several other features,  and is according to the customer, a popular choice in academia for CBR projects. It is also written in Java, which of course makes interfacing it simple from our own Java project.
 
However, its comprehensiveness also means that it takes more reading to understand and properly apply to the project at hand, and due to its size and poor documentation, jColibri was ultimately deemed unfit for the Privacy Advisor project. Due to the limited time resources available to this project, the risks associated with spending a large amount of time on a third party library that eventually would not be running was to high.
 
\subsubsection{Third Party k Nearest Neighbors Implementations}
Since kNN is a standard classification algorithm, there are several open source implementations available. Limiting the search space to Java implementations, a library called �The Java Machine Learning Library� (JavaML) was the primary candidate, as it provided a clean and simple interface and allowed for extracting confidence measures.
 
The problem with this library relates to the nature of distance metrics used in classifying privacy policies which is compositional in a way that is non-trivial to handle in JavaML. Furthermore, JavaML seems to operate only on arrays of floating point numbers, which means the distance metric must be defined in two stages; first mapping from �policy domain� to real numbers, then in terms of a metric on real vectors.

\subsubsection{XML Parser}
???

\section{Standards}
To achieve clean and reusable code, the project has adopted Oracle's Coding Conventions for the Java Programming Language\footnote{http://www.oracle.com/technetwork/java/codeconvtoc-136057.html}. This is mentioned in the requirements specification due to the high likelihood of the customer having to change the source code for later adaptations.


\section{Architecture}
To implement the Privacy Agent, a class structure is built around the CBR agent model discussed in T�ndel and Nyre with certain additions and refinements to handle data structures, algorithms, IO and so forth. 