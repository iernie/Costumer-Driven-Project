\documentclass[a4paper]{article}

\begin{document}

\section{Testing documentation}
\subsection{UNIT-01 - Command line interface functionality}
	\begin{center}
		\begin{tabular}{ |  p{3cm} | p{7cm} | }
			\hline
			Item & Description \\ [5pt] \hline \hline
			Name & Command line interface (CLI) functionality \\  [5pt] \hline
			Test identifier & UNIT-01 \\  [5pt] \hline
			Person responsible & Henrik Knutsen \\  [5pt] \hline
			Date of execution & October 27th \\  [5pt] \hline

			Execution steps & 	\begin{enumerate}
							\item Run the program for every type of argument
							\item Run the program without arguments
						\end{enumerate} \\ [5pt] \hline

			Steps executed & 	\begin{enumerate}
							\item
							\begin{enumerate}
								\item Run with -logloc logTest.txt
								\item Run with -loglevel ALL
								\item Run with -inDBLoc database.db
								\item Run with -outDBLoc database.db
								\item Run with -inWeightsLoc testWeights.cfg
								\item Run with -outWeightsLoc testWeights.cfg
								\item Run with -newDB false
								\item Run with -p3pLocation ticketmaster1.xml
								\item Run with -p3pDirLocation P3P
								\item Run with -blanketAccept true
								\item Run with -newPolicyLoc test.xml
								\item Run with -userInit true
								\item Run with -userResponse reject
								%\item Run with -cbrV bitmapDistanceWisOne,Reduction_KNN,Conclusion_Simple,LearnAlgSimpler
								%\item Run with -userIO UserIO_Simple
								%\item Run with -policyDB PDatabase
								%\item Run with -genConfig test.cfg
								\item Run with -NetworkRType NRCouchdb
								\item Run with -NetworkROptions privacydb,false,http,vm-6113.idi.ntnu.no,5984,PA,1234
								\item Run with -confidenceLevel 1.5
								\item Run with -useNet true
							\end{enumerate}
							\item Run the program without arguments
						\end{enumerate} \\ [5pt] \hline
			
			Expected results & 	\begin{enumerate}
							\item
							\begin{enumerate}
								\item Logfile is created at the specified filepath
								\item loglevel is set to the specified value
								\item inDBLoc is set to the specified value
								\item outDBLoc is set to the specified value
								\item inWeightsLoc is set to the specified filepath
								\item outWeightsLoc is set to the specified filepath
								\item newDB is set to false
								\item p3pLocation is set to the specified filepath
								item p3pDirLocation is set to the specified filepath
								\item recommendation is automatically accepted
								\item newPolicyLoc is set to the specified filepath
								\item userInit is set to the specified value
								%\item cbrV is set to the specified classes
								%\item userIO is set to the specified class
								%\item policyDB is set to the specified class
								%\item genConfig is set to the specified filepath. Values of the specified config file are loaded
								\item NetworkRType is set to the specified class
								\item NetworkROptions is set to the specified values
								\item confidenceLevel is set to the specified value
								\item useNet is set to the specified value
							\end{enumerate}
							\item All the values are loaded from the config file
						\end{enumerate} \\ [5pt] \hline

			Step results & 	\begin{enumerate}
							\item
							\begin{enumerate}
								\item Logfile is created at the specified filepath
								\item loglevel is set to ALL
								\item inDBLoc is set to database.db
								\item outDBLoc is set to database.db
								\item inWeightsLoc is set to testWeights.cfg
								\item outWeightsLoc is set to testWeights.cfg
								\item newDB is set to false
								\item p3pLocation is set to ticketmaster1.xml
								\item p3pDirLocation is set to P3P
								\item The recommendation is automatically accepted
								\item newPolicyLoc is set to test.xml
								\item userInit is set to true
								%\item cbrV is set to bitmapDistanceWisOne,Reduction_KNN,Conclusion_Simple,LearnAlgSimpler
								%\item userIO is set to UserIO_Simple
								%\item Not implemented, on TODO list
								%\item genConfig is set to test.cfg. Values in test.cfg are loaded
								\item NetworkRType is set to NRCouchdb
								\item NetworkROptions is set to privacydb,false,http,vm-6113.idi.ntnu.no,5984,PA,1234
								\item confidenceLevel is set to 1.5, recommendation is gives from (default) community database
								\item useNet is set to true, networking is enabled
							\end{enumerate}
							\item All the values are loaded from the default config file
						\end{enumerate} \\ [5pt] \hline

			Test conclusion & 	\begin{enumerate}
							\item
							\begin{enumerate}
								\item PASS
								\item PASS
								\item PASS
								\item PASS
								\item PASS
								\item PASS								
								\item PASS
								\item PASS
								\item PASS
								\item PASS
								\item PASS								
								\item PASS
								\item PASS
								\item PASS
								\item PASS
								\item PASS
								\item NO PASS
								\item PASS
								\item PASS
								\item NO PASS
								\item PASS
							\end{enumerate}
							\item PASS
						\end{enumerate}
						Test not passed. 1Q and 1T failed \\ [5pt] \hline

			Comments & --Check if userResponse is supposed to be accept / reject 
					--Check confidenceLevel again when networking is working, currently using default
					\\ [5pt] \hline

		\end{tabular}
	\end{center}


\subsection{UNIT-02 - P3P parser}
	\begin{center}
		\begin{tabular}{ |  p{3cm} | p{7cm} | }
			\hline
			Item & Description \\ [5pt] \hline \hline
			Name & P3P parser \\  [5pt] \hline
			Test identifier & UNIT-02 \\  [5pt] \hline
			Person responsible & Henrik Knutsen \\  [5pt] \hline
			Date of execution & October 27th \\  [5pt] \hline

			Execution steps & 	\begin{enumerate}
							\item Run a P3P xml in the P3P parser and print the parsed fields and their values to console
							\item Manually compare the printed fields and values with the contents of the P3P xml
						\end{enumerate} \\ [5pt] \hline

			Steps executed & 	\begin{enumerate}
							\item Test for barnesandnoble.com
							\begin{enumerate}
								\item barnesandnoble.xml is parsed and printed
								\item Contents of the xml is compared with to what was printed in step 1(a)
							\end{enumerate}
							\item Test for daduru.com
							\begin{enumerate}
								\item daduru.com is parsed and printed
								\item Contents of the xml is compared with to what was printed in step 2(a)
							\end{enumerate}
							\item Test for ssa.gov
							\begin{enumerate}
								\item ssa.gov is parsed and printed
								\item Contents of the xml is compared with to what was printed in step 3(a)
							\end{enumerate}
							\item Test for toysrus.com
							\begin{enumerate}
								\item toysrus.com is parsed and printed
								\item Contents of the xml is compared with to what was printed in step 4(a)
							\end{enumerate}
						\end{enumerate} \\ [5pt] \hline
			
			Expected results &	\begin{enumerate}
							\item The P3P xml is parsed successfully. It's content is printed to console
							\item The printed output have the same fields, each having the same value as those in the xml
						\end{enumerate}
							 \\  [5pt] \hline

			Step results & 	\begin{enumerate}
							\item Results for barnesandnoble.com
							\begin{enumerate}
								\item The P3P xml is parsed successfully. It's content is printed to console
								\item The printed output have the same fields, each having the same values as those in the xml
							\end{enumerate}
							\item Results for daduru.com
							\begin{enumerate}
								\item The P3P xml is parsed successfully. It's content is printed to console
								\item The printed output have the same fields, each having the same values as those in the xml
							\end{enumerate}
							\item Results for ssa.gov
							\begin{enumerate}
								\item ssa.gov is parsed and printed
								\item Contents of the xml is compared with to what was printed in step 3(a)
							\end{enumerate}
							\item Results for toysrus.com
							\begin{enumerate}
								\item The P3P xml is parsed successfully. It's content is printed to console
								\item The printed output have the same fields, each having the same values as those in the xml
							\end{enumerate}
						\end{enumerate}
							 \\  [5pt] \hline

			Test conclusion & 	\begin{enumerate}
							\item PASS
							\item PASS
							\item PASS
							\item PASS
						\end{enumerate}
						Test passed \\  [5pt] \hline
		\end{tabular}
	\end{center}

\subsection{UNIT-03 - Local database}
		\begin{center}
		\begin{tabular}{ |  p{3cm} | p{7cm} | }
			\hline
			Item & Description \\ [5pt] \hline \hline
			Name & Local database \\  [5pt] \hline
			Test identifier & UNIT-03 \\  [5pt] \hline
			Person responsible & Henrik Knutsen \\  [5pt] \hline
			Date of execution & October 27th \\  [5pt] \hline

			Execution steps & 	\begin{enumerate}
							\item Write policy A to the local database
							\item Write policy B to the local database
							\item Read and print policy A from the local database
							\item Read and print policy A from the local database
							\item Compare the written policy A and the read policy A
							\item Compare the written policy B and the read policy B
						\end{enumerate} \\ [5pt] \hline

			Steps executed & 	\begin{enumerate}
							\item Program is started writing policy A to an empty database, and writing policy A to the database
							\item Accept recommendation and chose to save the new action and policy
							\item Print the new database after policy B was added (step 2)
							\item Also done in step 3
							\item The contents of policy A that was to be written in step 1 is compared to what was printed of policy A in step 3
							\item The contents of policy B that was to be written in step 2 is compared to what was printed of policy B in step 3
						\end{enumerate} \\ [5pt] \hline
			
			Expected results &	\begin{enumerate}
							\item Policy A is successfully written to the database file
							\item Policy B is successfully written to the database file
							\item Policy A is successfully read from the database file and printed
							\item Policy B is successfully read from the database file and printed
							\item The written policy A and the read policy A are identical. They both have the same fields, with the same values
							\item The written policy B and the read policy B are identical. They both have the same fields, with the same values
						\end{enumerate}
							 \\  [5pt] \hline

			Step results & 	\begin{enumerate}
							\item Policy A was successfully written to the database file
							\item Policy B was successfully written to the database file
							\item Database was successfully printed
							\item Same as step 3
							\item The loaded and printed policy A is the same as the one that was written
							\item The loaded and printed policy B is the same as the one that was written
						\end{enumerate}
							 \\  [5pt] \hline

			Test conclusion & 	\begin{enumerate}
							\item PASS
							\item PASS
							\item PASS
							\item PASS
							\item PASS
							\item PASS
						\end{enumerate}
						Test passed \\  [5pt] \hline
			Comments &	- \\ [5pt] \hline
		\end{tabular}
	\end{center}

\subsection{UNIT-04 - Graphical user interface (GUI) functionality}
		\begin{center}
		\begin{tabular}{ |  p{3cm} | p{7cm} | }
			\hline
			Item & Description \\ [5pt] \hline \hline
			Name & Graphical user interface (GUI) functionality \\  [5pt] \hline
			Test identifier & UNIT-04 \\  [5pt] \hline
			Person responsible & Henrik Knutsen \\  [5pt] \hline
			Date of execution & October 29th \\  [5pt] \hline

			Execution steps & 	\begin{enumerate}
							\item Run the program using the GUI
							\item Test every option in the menu bar
							\item Test every button in the load config file prompt
							\item Test every scroll bar
							\item Resize the window
						\end{enumerate} \\ [5pt] \hline

			Steps executed & 	\begin{enumerate}
							\item Program started with graphical user interface
							\item  
							\begin{enumerate}
								\item Clicked load config file
								\item Clicked load db file
								\item Clicked run
								\item Clicked exit
							\end{enumerate}
							\item ---
							\item 
							\begin{enumerate}
								\item Used scrollbar for the printed contents of the chosen database element (upper section)
								\item Used scrollbar for the contents of the database file (middle / lower section)
							\end{enumerate}
							\item Attempted to resize the window
						\end{enumerate} \\ [5pt] \hline
			
			Expected results &	\begin{enumerate}
							\item The program starts and loads the graphical user interface
							\item  
							\begin{enumerate}
								\item Prompt for setting config values is opened
								\item The specified database is loaded
								\item The program gives a popup with a recommendation for the selected policy based on the history in the specified database
								\item Program closes
							\end{enumerate}
							\item ---
							\item 
							\begin{enumerate}
								\item The scrollbar scrolls through the list, up and down, from top to bottom, successfully
								\item The scrollbar scrolls through the list, up and down, from top to bottom, successfully
							\end{enumerate}
							\item Window can be resized without having elements of the GUI overlapping. The elements and panes should scales with the main window
						\end{enumerate}
							 \\  [5pt] \hline

			Step results & 	\begin{enumerate}
							\item The program starts and loads the graphical user interface
							\item 
							\begin{enumerate}
								\item Graphical interface for setting config values is opened
								\item The specified database file is loaded
								\item The program gives a popup with a recommendation and options for accepting or rejecting
								\item Program closes
							\end{enumerate}
							\item ----
							\item  
							\begin{enumerate}
								\item The scrollbar scrolls through the list, up and down, fro top to bottom, successfully
								\item Error when scrolling to the top of the list (nullPointerException). Bugging out the elements of the list the xml
							\end{enumerate}
							\item Can't rescale the window
						\end{enumerate}
							 \\  [5pt] \hline

			Test conclusion & 	\begin{enumerate}
							\item PASS
							\item 
							\begin{enumerate}
								\item PASS
								\item PASS
								\item PASS
								\item PASS
							\end{enumerate}
							\item ----
							\item  
							\begin{enumerate}
								\item PASS
								\item NO PASS
							\end{enumerate}
							\item NO PASS
						\end{enumerate}
						Test not passed. 3, 4B and 5 failed \\  [5pt] \hline
		\end{tabular}
	\end{center}

\subsection{UNIT-05 - Algorithm classification}
			\begin{center}
		\begin{tabular}{ |  p{3cm} | p{7cm} | }
			\hline
			Item & Description \\ [5pt] \hline \hline
			Name & Algorithm classification \\  [5pt] \hline
			Test identifier & UNIT-05 \\  [5pt] \hline
			Person responsible & Henrik Knutsen \& Dimitry Kongevold \\  [5pt] \hline
			Date of execution & October 30th \\  [5pt] \hline

			Execution steps & 	\begin{enumerate}
							\item Load a set of policies into the database file
							\item Manually calculate and write down the distances between the single policy and each of the policies in the history
							\item Run the distance algorithm on a single policy and the history and compare the distances that are returned by the algorithm with the manually calculated distances from step 2
							\item Manually find the k policies with the lowest distances
							\item Run the reduction algorithm with necessary input to find the k nearest policies and compare the k policies returned by the reduction algorithm with those found in step 4
							\item Run the conclusion algorithm and verify the results returned by the algorithm ???
						\end{enumerate} \\ [5pt] \hline

			Steps executed & 	\begin{enumerate}
							\item Created a test domain by loading six policies into the history
							\item Manually calculated the distances between the policy to be classified and each of the six policies, and inserted the reulsts into a table
							\item Created a JUnit test testReduction_KNN. Asserting that the six values returned by the algorithm is the same as those calculated manually in step 2
							\item Found the k nearest policies from what was calculated in step 2
							\item Created a JUnit test testReduction_KNN. Asserting that the reduction algorithm returns the same k nearest policies are those found manually in step 4
							\item Created a JUnit test testReduction_KNN ???
						\end{enumerate} \\ [5pt] \hline
			
			Expected results &	\begin{enumerate}
							\item The policies are added to the database file
							\item The six distances are obtained
							\item The JUnit test passes
							\item The k policies are obtained
							\item The JUnit test passes
							\item The JUnit test passes ???
						\end{enumerate}
							 \\  [5pt] \hline

			Step results & 	\begin{enumerate}
							\item The policies are added to the database file
							\item The six distances are obtained
							\item The JUnit is successful
							\item The k policies are obtained
							\item The JUnit test is \underline{un}successful. The most distant policy is returned
							\item The JUnit test is not run
						\end{enumerate}
							 \\  [5pt] \hline

			Test conclusion & 	\begin{enumerate}
							\item PASS
							\item PASS
							\item PASS
							\item PASS
							\item NO PASS
							\item NO PASS
						\end{enumerate}
						Test failed. 1E and 1F failed \\  [5pt] \hline
		\end{tabular}
	\end{center}

\subsection{UNIT-06 - Algorithm learning}
	\begin{center}
		\begin{tabular}{ |  p{3cm} | p{7cm} | }
			\hline
			Item & Description \\ [5pt] \hline \hline
			Name & Algorithm learning \\  [5pt] \hline
			Test identifier & UNIT-06 \\  [5pt] \hline
			Person responsible & Henrik Knutsen \& Neshahavan Karunakaran \\  [5pt] \hline
			Date of execution & November 4th \\  [5pt] \hline

			Execution steps & 	\begin{enumerate}
							\item a
						\end{enumerate} \\ [5pt] \hline

			Steps executed & 	\begin{enumerate}
							\item b
						\end{enumerate} \\ [5pt] \hline
			
			Expected results &	\begin{enumerate}
							\item c
						\end{enumerate}
							 \\  [5pt] \hline

			Step results & 	\begin{enumerate}
							\item d
						\end{enumerate}
							 \\  [5pt] \hline

			Test conclusion & 	\begin{enumerate}
							\item d
						\end{enumerate}
						Test failed \\ [5pt] \hline
		\end{tabular}
	\end{center}

\subsection{UNIT-07 - Networking (community database)}
	e


\end{document}
