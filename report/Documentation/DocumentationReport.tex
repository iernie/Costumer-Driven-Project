% Activate the following line by filling in the right side. If for example the name of the root file is Main.tex, write
% "...root = Main.tex" if the chapter file is in the same directory, and "...root = ../Main.tex" if the chapter is in a subdirectory.
 
%!TEX root =  

\chapter{Documentation}\label{doc}

\minitoc

This section provides documentation for the Privacy Advisor system. It gives an overview over the available documentation of source code, instructions for how to compile and install the system, and how it is used via. its GUI and command line interfaces.

\section{Source Code Documentation}

The source code is documented using JavaDoc which is 	a tool that generates documentation in HTML format based on source code comments in Java, and is a standard part of the Java SDK. The JavaDoc for the Privacy Advisor system follows Sun Microsystems' style guide for writing JavaDoc comments\footnote{See: http://www.oracle.com/technetwork/java/javase/documentation/index-137868.html}. Source code documentation plays an important role in this project, as it is an early software prototype to be used in research which means that the code is then likely to be modified. The aim of the source code documentation is to supplement UML design documents to facilitate future development.

\section{Installation}

\subsection{Local Installation}

\subsection{Server Installation}

\section{User Interfaces}

\subsection{Graphical User Interface}

\subsection{Command Line Interface}

\subsubsection{Configuration Files}



\subsubsection{Building a Database}

\subsubsection{Loading and Viewing a Database}

\subsubsection{Parsing a P3P Policy}