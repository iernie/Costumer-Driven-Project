% Activate the following line by filling in the right side. If for example the name of the root file is Main.tex, write
% "...root = Main.tex" if the chapter file is in the same directory, and "...root = ../Main.tex" if the chapter is in a subdirectory.
 
%!TEX root =  

\chapter{Planning Phase}

\minitoc 

\section{Purpose}
This document details the different phases in the development process. It presents the development model that has been chosen, relating the choice of model to the particular challenges faced by the nature of the project. It also describes the set of activities included in each phase and a set of activities that are ongoing throughout the lifetime of the project.

\section{Development Model}
As the project is an early implementation phase of a larger research project, there is an inherent uncertainty as to several details regarding both the implementation and the appropriateness of the underlying model. This has several implications for the development process. 

Following the above discussion, a hybrid waterfall model is settled upon. Here the design, implementation and testing phases follow a cyclical pattern. In concrete terms, this means that a first prototype will be developed and then put through some tests, and depending on the results of these tests, there may be revisions in the software design.

Furthermore, there are clearly overlaps between these three phases. A key part of the documentation work consists of documenting the source code, which is basically commenting during development. Furthermore, as implementation is underway, it may turn out that the design needs adjustments, due to unforeseen factors.


\section{Phases}

\subsection{Planning}

\subsection{Prestudy and Research}
In this phase the aim is for each project member to acquire a certain level domain knowledge in the field of Internet privacy and to learn the necessary technology and tools required to implement the model as proposed by the customer. This entails having a working knowledge of the Java programming language, version control using Git and the CBR framework.

\subsection{Requirement Specification}
The requirements specification is a document listing the functional and non-functional requirements of the software to be developed, which is a standard that the results is to be measured against, thus serving as not only a contract between the customer and the project team, but as a basis for developing testing methods.

\subsection{Design/Architecture}
This phase consists of a broad structuring and specification of the overall system. It defines the program structure in terms of program flow, modules, classes and interfaces as well as coding standards and other conventions that will serve as guidelines for the implementation phase.

\subsection{Implementation}
In this phase the design is realized as a working Java program according to the models developed in the Design phase. 

\subsection{Evaluation and Documentation}
This phase consists of testing the system and documenting the structure of the system and how it is operated. From a software engineering perspective, the primary testing grounds are against the standards prescribed by the requirements specification rather than applicability of the underlying model�s performance. As mentioned, among the primary objectives of the project is to provide a testing framework to verify the applicability of the given system in making privacy decisions.

\subsection{Ongoing Activities}

\subsubsection{Reporting and Administrative Tasks}
Under this heading are more project management related activities, such as routine organizational work (ie. arranging meetings and writing status updates), more refined distribution of tasks as the project is underway, and preparation of the project report (this document).

\subsubsection{Study and Lectures}
To solve several of the problems posed by this project, most group members have had to learn new tools and technologies. This includes, but is not limited to Case Based Reasoning, version control (Git), certain features of Java and so on. Lectures on project management  and software development are also subsumed under this heading.

%the project plan:
\section{Measurement of project effects}
The minimal goal of the project is to develop the core functionality for a system that gives users advise when visiting a web-page. The functional requirments The advise is based on: 

\begin{itemize}
\item the user�s previous actions
\item community actions or what similar users have done
\item context of use
\end{itemize}


\section{Risk Report}
In this section we have identified some possible risk for this project. Almost every project should have some kind of risk management. In table \ref{riskTable} below is a description of how the risks would look like and what they mean. Beneath that the actual risk follows.

\begin{table}[htdp]
\caption{Risks explained}
\begin{center}
\begin{tabularx}{\textwidth}{| X | X |}
\hline
\textbf{Risk item} & Unoredered numbering of the risks \\
\hline
\textbf{Activity} & In which activity this risk could happen \\
\hline
\textbf{Risk Factor} & The risk \\
\hline
\textbf{Probability} & The probability of this risk to occur \\
\hline
\textbf{Consequence} & What the consequences of this would be \\
\hline
\textbf{Action taken} & What actions must be taken to avoid this\\ & risk \\
\hline
\textbf{Responsible} & The responsible ones for this risk \\
\hline
\end{tabularx}
\end{center}
\label{riskTable}
\end{table}

The actual risks:

\begin{table}[htdp]
\caption{Changes in requirments}
\begin{center}
\begin{tabularx}{\textwidth}{| X | X |}
\hline
\textbf{Risk item} & 1 \\
\hline
\textbf{Activity} & All \\
\hline
\textbf{Risk Factor} & The requirements might change \\
\hline
\textbf{Probability} & 2 \\
\hline
\textbf{Consequence} & 4 \\
\hline
\textbf{Action taken} & Clarify the requirments and agree on deadlines for any changes that could happen\\
\hline
\textbf{Responsible} & Ulf, customer \\
\hline
\end{tabularx}
\end{center}
\label{risk_1}
\end{table}


\begin{table}[htdp]
\caption{Poorly choosen algorithms}
\begin{center}
\begin{tabularx}{\textwidth}{| X | X |}
\hline
\textbf{Risk item} & 2 \\
\hline
\textbf{Activity} & Design, implementation  \\
\hline
\textbf{Risk Factor} & The implemented algorithms may not work as intended\\
\hline
\textbf{Probability} & 3 \\
\hline
\textbf{Consequence} & 5 \\
\hline
\textbf{Action taken} & Research on similar algorithms and projects \\
\hline
\textbf{Responsible} & Everyone in the group\\
\hline
\end{tabularx}
\end{center}
\label{risk_2}
\end{table}

\begin{table}[htdp]
\caption{Problems with policy retrieving}
\begin{center}
\begin{tabularx}{\textwidth}{| X | X |}
\hline
\textbf{Risk item} & 3 \\
\hline
\textbf{Activity} & Implementation  \\
\hline
\textbf{Risk Factor} & Problems with retreiving data from P3P policies \\
\hline
\textbf{Probability} & 3 \\
\hline
\textbf{Consequence} & 5 \\
\hline
\textbf{Action taken} & Research into P3P, and cooperate with \newline customer \\
\hline
\textbf{Responsible} & Everyone in the group\\
\hline
\end{tabularx}
\end{center}
\label{risk_3}
\end{table}

\begin{table}[htdp]
\caption{Problems with external storage}
\begin{center}
\begin{tabularx}{\textwidth}{| X | X |}
\hline
\textbf{Risk item} & 4 \\
\hline
\textbf{Activity} & Implementation  \\
\hline
\textbf{Risk Factor} & Problems with storing and/or retrieving data \\
\hline
\textbf{Probability} & 2 \\
\hline
\textbf{Consequence} & 3 \\
\hline
\textbf{Action taken} & Look into other methods to save/load data \\
\hline
\textbf{Responsible} & Amanpreet, Dimitry, Nesha \\
\hline
\end{tabularx}
\end{center}
\label{risk_4}
\end{table}

\begin{table}[htdp]
\caption{Remote server problems}
\begin{center}
\begin{tabularx}{\textwidth}{| X | X |}
\hline
\textbf{Risk item} & 5 \\
\hline
\textbf{Activity} & Implementation  \\
\hline
\textbf{Risk Factor} & Obtaining remote server space \\
\hline
\textbf{Probability} & 1 \\
\hline
\textbf{Consequence} & 3 \\
\hline
\textbf{Action taken} & Ask IDI for virtual server \\
\hline
\textbf{Responsible} & Nicholas \\
\hline
\end{tabularx}
\end{center}
\label{risk_5}
\end{table}


\begin{table}[htdp]
\caption{Sickness}
\begin{center}
\begin{tabularx}{\textwidth}{| X | X |}
\hline
\textbf{Risk item} & 6 \\
\hline
\textbf{Activity} & All \\
\hline
\textbf{Risk Factor} & Unable to work due to sickness \\
\hline
\textbf{Probability} & 2 \\
\hline
\textbf{Consequence} & 4 \\
\hline
\textbf{Action taken} & Atleast one person must know what another \newline person is working with, so that person might take over \\
\hline
\textbf{Responsible} & Everyone in the group\\
\hline
\end{tabularx}
\end{center}
\label{risk_6}
\end{table}

\begin{table}[htdp]
\caption{3rd party library problems}
\begin{center}
\begin{tabularx}{\textwidth}{| X | X |}
\hline
\textbf{Risk item} & 7 \\
\hline
\textbf{Activity} & Implementation, testing  \\
\hline
\textbf{Risk Factor} & 3rd party code may be harmful or not work as intended \\
\hline
\textbf{Probability} & 1 \\
\hline
\textbf{Consequence} & 3 \\
\hline
\textbf{Action taken} & Be careful when choosing 3rd party library, and test it throughly \\
\hline
\textbf{Responsible} & Those who implement the 3rd party library and the testers \\
\hline
\end{tabularx}
\end{center}
\label{risk_7}
\end{table}

\begin{table}[htdp]
\begin{center}
\begin{tabularx}{\textwidth}{| X | X |}
\hline
\textbf{Risk item} & 8 \\
\hline
\textbf{Activity} & All  \\
\hline
\textbf{Risk Factor} & Misunderstandings between customer and the group \\
\hline
\textbf{Probability} & 3 \\
\hline
\textbf{Consequence} & 3 \\
\hline
\textbf{Action taken} & Be specific and make sure everyone understands \\
\hline
\textbf{Responsible} & Everyone \\
\hline
\end{tabularx}
\end{center}
\label{risk_8}
\end{table}

\clearpage

\section{Workplan}
The work is separated into six parts, which are Pre-implementation research, Requirement specification, design, implementation and documentation, evaluation, and report writing. The reporting started at the first day and is going to continue until we are finished with the project. We are intending to be finished with the implementation at the end of week 42, and from then on focus more on testing (evaluation) the code/system.
A Gantt diagram over the workplan as it is now:
\begin{center}
\includegraphics[bb=0 0 1568 1171, width=\linewidth, angle=90]{PlanReport/gantt.jpg} %\linewidth
\end{center}
