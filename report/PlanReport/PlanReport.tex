% Activate the following line by filling in the right side. If for example the name of the root file is Main.tex, write
% "...root = Main.tex" if the chapter file is in the same directory, and "...root = ../Main.tex" if the chapter is in a subdirectory.
 
%!TEX root =  

\chapter{Planning Phase}

\minitoc 

\section{Purpose}
This document details the different phases in the development process. It presents the development model that has been chosen, relating the choice of model to the particular challenges faced by the nature of the project. It also describes the set of activities included in each phase and a set of activities that are ongoing throughout the lifetime of the project.

\section{Development Model}
As the project is an early implementation phase of a larger research project, there is an inherent uncertainty as to several details regarding both the implementation and the appropriateness of the underlying model. This has several implications for the development process. 

Following the above discussion, a hybrid waterfall model is settled upon. Here the design, implementation and testing phases follow a cyclical pattern. In concrete terms, this means that a first prototype will be developed and then put through some tests, and depending on the results of these tests, there may be revisions in the software design.

Furthermore, there are clearly overlaps between these three phases. A key part of the documentation work consists of documenting the source code, which is basically commenting during development. Furthermore, as implementation is underway, it may turn out that the design needs adjustments, due to unforeseen factors.


\section{Phases}

\subsection{Planning}

\subsection{Prestudy and Research}
In this phase the aim is for each project member to acquire a certain level domain knowledge in the field of Internet privacy and to learn the necessary technology and tools required to implement the model as proposed by the customer. This entails having a working knowledge of the Java programming language, version control using Git and the CBR framework.

\subsection{Requirement Specification}
The requirements specification is a document listing the functional and non-functional requirements of the software to be developed, which is a standard that the results is to be measured against, thus serving as not only a contract between the customer and the project team, but as a basis for developing testing methods.

\subsection{Design/Architecture}
This phase consists of a broad structuring and specification of the overall system. It defines the program structure in terms of program flow, modules, classes and interfaces as well as coding standards and other conventions that will serve as guidelines for the implementation phase.

\subsection{Implementation}
In this phase the design is realized as a working Java program according to the models developed in the Design phase. 

\subsection{Evaluation and Documentation}
This phase consists of testing the system and documenting the structure of the system and how it is operated. From a software engineering perspective, the primary testing grounds are against the standards prescribed by the requirements specification rather than applicability of the underlying model�s performance. As mentioned, among the primary objectives of the project is to provide a testing framework to verify the applicability of the given system in making privacy decisions.

\subsection{Ongoing Activities}

\subsubsection{Reporting and Administrative Tasks}
Under this heading are more project management related activities, such as routine organizational work (ie. arranging meetings and writing status updates), more refined distribution of tasks as the project is underway, and preparation of the project report (this document).

\subsubsection{Study and Lectures}
To solve several of the problems posed by this project, most group members have had to learn new tools and technologies. This includes, but is not limited to Case Based Reasoning, version control (Git), certain features of Java and so on. Lectures on project management  and software development are also subsumed under this heading.



