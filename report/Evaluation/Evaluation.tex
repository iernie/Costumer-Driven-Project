% Activate the following line by filling in the right side. If for example the name of the root file is Main.tex, write
% "...root = Main.tex" if the chapter file is in the same directory, and "...root = ../Main.tex" if the chapter is in a subdirectory.
 
%!TEX root =  

\chapter{Project Evaluation}\label{eval}

\minitoc

\subsection*{Project Evaluation}
The final phase of the project consists of evaluating the actual outcome of
the project work and the processes that have led to the outcome. It
seeks to answer questions such as:

\begin{itemize}
\item Has the project acheived its major objectives?
\item In what areas could a better result have been achieved?
\item What could be done differently to acheive a better result?
\item How well did planned resource use reflect actual resource use?
\end{itemize}

\section{Software Evaluation}




\section{Group Dynamics and Organization}
The project team was slightly heterogenous, with two international
students and the remaining seven Norwegian. For this reason, English
was chosen as the offical language of the project. This may to some
extent have impacted the degree of participation of some team members
during meetings, and discussions came to be dominated by a minority. 



\section{Tools}

Referring to choices discussed in section~\ref{DevTools}, the key
software tools and languages used are Java, Git, and Google
Docs. 

\subsection{Programming and Implementation}

\subsubsection{Java}
The choice of programming language for the project was very much up to the
project team as the customer had no particular preference, and the
project did not require anything beyond what is catered for by most
general purpose languages. \textbf{Java} was therefore selected 
selected as implementation language based on the following points:

\begin{itemize}
\item All team members had some previous experience in Java
  programming from previous coursework.
\item Good documentation and a large amount of third party libraries
  available on the web.
\end{itemize}

While all team members did have a Java programming background, there
were clearly vast differences in skill levels. This was the cause of
some frustration and extra work. In retrospect, more time should have been spent
on clarifying background and interests of each team member so as to
better allocate workload. Some will also argue that a more
"light-weigh'' programming language, such as Python, would have been a
better choice, even despite the lack of previous experience.

\subsubsection{Git/GitHub}
Git and GitHub were chosen as version control system (VCS) and code
repository. Several team members had only marginal experience with
VCSs and did not commit to learning this in a serious manner, which
led to somewhat slow progress in the start. However, once these problems
were resolved, it can be agreed that Git has served its purpose well.


\subsection{Reporting and Organizational Tools}
Google Docs was used as repository for all "temporary'' documents such
as agendas, status reports, drafts and so forth. All "major''
documents were typeset in \LaTeX and stored at the GitHub repository.

\subsubsection{Google Docs}


\subsubsection{\LaTeX}
Being the de-facto typesetting system for academic writing, and the
only one that some group members had experience with, \LaTeX was
chosen for writing the final report as well as all phase documents. As
with GitHub, though to a lesser extent,  there was a lack of commitment to learning by certain
group members, causing frustration and extra work on a later stage
in the project.


\section{Resource Use}

Figures~\ref{perActivity} and \ref{actualPlannedCuml} illustrate
the discrepancies between actual and planned time use on all
activities, both cumulative over time and over each phase of the project.

\begin{centering}
  \begin{figure}
    \includegraphics[width = \textwidth]{Evaluation/time_per_activity.png}
    \caption{Actual vs. planned time. By project phase.}
    \label{perActivity}
  \end{figure}
\end{centering}

\begin{centering}
  \begin{figure}
    \includegraphics[width = \textwidth]{Evaluation/actual_v_planned_cuml.png}
    \caption{Actual vs. planned time. Cumulative.}
    \label{actualPlannedCuml}
  \end{figure}
\end{centering}

\section{Risks that Occurred}

Referring to the discussion in section~\ref{riskReport},..

\section{Course Evaluation - TDT4290}

Finally, we include some notes on how well the objectives of the
course \emph{TDT4290 - Customer Driven Project} itself have been met.