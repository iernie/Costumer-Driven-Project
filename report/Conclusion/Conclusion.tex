% Activate the following line by filling in the right side. If for example the name of the root file is Main.tex, write
% "...root = Main.tex" if the chapter file is in the same directory, and "...root = ../Main.tex" if the chapter is in a subdirectory.
 
%!TEX root =  

\chapter{Conclusion}\label{conclusion}

\minitoc

\section{Summary and Final Remarks}

This project has designed and implemented a privacy agent based on the
research ideas set forth by SINTEF. The Privacy Advisor software as it
is today, is a case based reasoning engine that can provide advice
with regard to P3P privacy policies given knowledge of previous user
actions. Both the idea and the software remain work in progress,
which is reflected by the system's user interface which is oriented
towards testing. It is built in a modular fashion, so that the
knowledge base, the retrieval algorithm and the similarity metrics, which are likely to be the
parts of the system that require most of the tweaking, can
be substituted without affecting the remainder of the software.



\section{Future Development}

To conclude this report, we look at the some important challenges that
must be addressed in order to further the Privacy Advisor system to a
end-user product.

\subsection{Testing}
For the Privacy Advisor system to be truly valuable, it has to be made
available to end users; the most likely realization being in the
form of a web-browser plugin. For this to occur, much testing is still
required. This testing is a more involved type of testing than the standard
unit testing that the Privacy Advisor system has been through at present.

For the system to actually be successful it has to provide accurate predictions
of users' Internet decisions so that it can provide good privacy protection
without being invasive in the users' day-to-day Internet activities. This type
of testing will require putting together a group of test users and monitoring their
activities over a \emph{prolonged period of time}\footnote{The time aspect is stressed here
as the system will need time to actually learn the user's preferences.}. 

The project team feels that most important and time consuming part of future development lies
in this area. Designing appropriate measures and testing routines for evaluating the performance
of the CBR agent is probably the main obstacle on the way towards realizing the design.

\subsection{Collaborative Filtering} %% TODO: Nicholas: verify
One of the key improvements to the current Privacy Advisor system is the
collaborative filtering/community portion of the system. While the interface
for a networking system is well integrated in the CBR system, the current 
network resource has limited features. Furthermore, this component of the system
will also require a prolonged period of testing, as a substantial amount of data.

% issues with duplicate policies on server, current state, proposed
% fix, sintef solution and problems with storing per user basis

\subsection{Distance Metrics and Algorithms}


\subsection{The Current State of Internet Privacy Policies}
Finally, some words about the current state of Internet privacy policies. While P3P represents
an effort towards a machine readable standard for privacy policies, it is far from being adopted
universally. At present several standards seem to coexist, so a finalized system would have to
be able to handle several different standards.

Also posing significant problems to the effectiveness of privacy enhancement software is fact 
that there are little in the way of legal regulations of privacy protection online\footnote{See for instance:
http://www.nytimes.com/2011/11/20/opinion/sunday/a-push-for-online-privacy.html?_r=1&ref=opinion.}.
At present, businesses are free to arbitrarily change their privacy policies without noticing users,
and very little effort has been made towards verification or enforcement of policies. This means that currently,
the privacy policies cannot be the only input to a \emph{truly} trustworthy privacy enhancement software.


