% Activate the following line by filling in the right side. If for example the name of the root file is Main.tex, write
% "...root = Main.tex" if the chapter file is in the same directory, and "...root = ../Main.tex" if the chapter is in a subdirectory.
 
%!TEX root =  

\chapter{Project Directive}
\label{directive}
\minitoc

\subsection*{Purpose}
This chapter contains the project directive for the project "Privacy Advisor'' (the project) which is developed for SINTEF ICT (the customer) as a part of the course TDT4290 - Customer Driven Project at NTNU during the fall semester of 2011.

The project directive provides a broad overview over the project, defining its objectives, scope, the responsibilities of project participants as well as a few core processes and routines related to project management, reporting and quality assurance. The directive serves as a guideline for project work and later on, project evaluation along with the requirements specification.

The project directive is intended to be dynamic document reflecting the nature of the project and its inherent uncertainties. If a major change of direction occurs during the project lifetime, the project directive is updated accordingly in agreement between the project team and the customer.

\section{Mandate}
The purpose of this project is to implement the key functionality of a privacy agent as described in Nyre and T{\o}ndel (2010), that provides users with advice in making Internet privacy decisions. 

\subsection{Background}
This project is a part of a larger research project at SINTEF ICT that studies approaches to handling Internet privacy related issues. The underlying idea is that while users are often concerned about the way various websites and services handle private information about them, obtaining information about this is very costly in terms of time and effort as privacy policies tend to be very long documents formulated in an inaccessible language. This has led to the suggestion that Internet privacy can be assisted by machine learning techniques, where a particular decision is based on the user's past behavior and the behavior of similar users.

Nyre and T{\o}ndel have proposed a "Privacy Agent" structure that uses the case based reasoning (CBR) method for giving privacy advice. CBR is in many ways similar to the way human experts reason about problems; that is, by looking at what has been done in similar cases previously. Nyre and T{\o}ndel also describes this CBR approach to be complemented by a community database where the same information is stored, allowing for a second lookup that uses a collaborative filtering, that is, making a decision based on the behavior of similar users. 

\section{Objectives}\label{mandateObjectives}
This project identifies three key objective, arranged by order of importance:

\begin{enumerate}
\item Implementing a testing framework of CBR based privacy agent that is able to make privacy decisions based on previous user behavior.
\item Implement the community system/collaborative filtering part of the agent.
\item Extend the system to other standards for machine readable privacy policies.
\item Implement the system as a browser plugin. 
\end{enumerate}

It is important to note that the success of the first objective is by far the most critical to this project. If the core framework and CBR engine is implemented in a robust and extensible fashio that caters for Objectives 2-4, reaching those objectives will be less work. In order to be able to proceed with objectives 3-4, a considerable success from the testing phase would also be required for the system to be viable, that is, it must actually provide good recommendations. Such testing may prove to require more resources than available to this project.

Implementing a browser plugin is considered least important, as it is highly contingent on the success of early testing. It is also given a low priority given the relativly small portion of major websites that implement P3P. 


\section{Resources and Duration}
The system in its complete form is to be demonstrated on November 24 2011. For the project period, a total of 25 hours per week per project member is planned. With seven group members and a project spanning 13 weeks, this adds up to approximately 2300 hours.


\section{Organization}
Project management is based on a standard model where the customer takes on the role of \emph{project owner} or simply "owner". The owner is the actual stakeholder and initiator of the project, and responsible for all executive decisions in the project. 

The \emph{project group} or \emph{team} is responsible for delivering the product in accordance with the wishes of the customer as defined by the \emph{requirements specification} document. Two project management roles are designated, one is responsible for administrative decisions, hereunder planning, reporting, calling meetings, customer contact and so forth. The second management role is that of chief system architect, who has the responsibility and final word in all technical decisions.

The project group organization is based on the modules of the system that is being implemented, this is often referred to as a \emph{functional} structure or organization. One group member is responsible for developing one particular feature. This organization is shown in Table~\ref{orgTable}. In addition to this internal functional organization, two project managers are appointed, one having responsibilities for administrative decisions and reporting and one with responsible for technical decisions.

\begin{table}[htdp]
\begin{center}
\begin{tabularx}{\textwidth}{| X | X | X | X |}
\hline
\textbf{Area} & \textbf{Role} & \textbf{Description} &\textbf{Responsible} \\
\hline
Administrative 		&Project Manager 	&  Customer relations & Ulf Nore \\
				&			 	& Requirements specification & \\
				&				& Planning 			      & \\
				&				& Meeting minutes 		      & \\
				&				& Status reporting		      & \\
				&				& Project report 		      & \\	
\hline
Administrative 	& Head Systems  & Overall design.	 	& Nicholas Gerstle \\
Technical 		& Architect 	   & Design report.		& \\
			&			   & User documentation.	& \\
			&			   & Technical Decisions.	& \\
\hline
\hline
Technical &	Data Storage/Databases  & Flat file data storage system. & Amanpreet Kaur \\
		&						& Database systems	.		& \\
\hline
Technical & CBR - Algorithms and Data structures 	& Data structures for storing privacy policy information.	& Dimitry Kongevold,\\
 		&								& Define and implement similarity metrics.			& Neshahavan Karunakaran \\
 		&								& Retrieval and learning algorithms. 				&	\\
		&								& Parameter storage. 							& \\
\hline
Technical 	&	Testing and Evaluation 	& Design test cases. 	& Henrik Knutsen \\
		&						& Criteria/methodology for model testing. & \\
\hline
Technical & GUI & Implement a simple GUI for testing model framework. & Ulf Nore \\
\hline
Technical & Version control & Set up and maintain code repository. & Einar Afiouni \\
\hline
Technical & XML/P3P Parser & Implement P3P parser that produces inputs to CBR & Einar Afiouni \\
\hline
\end{tabularx}
\caption{Responsibilities.}
\end{center}
\label{orgTable}
\end{table}%



\section{Planning}
A project plan has been developed for the purpose of communicating expectations and progress within the group and to the customer and the advisor. The plan also serves as an aid in identifying problems and project management. For the software development process, a hybrid waterfall model has been chosen.


\section{Limitations and Scope}
The primary focus of this project is on developing a framework that allows for testing the CBR privacy agent framework. This entails building a module for parsing policy documents in XML format, a data structure for holding policy information in memory (henceforth "policy objects''), a set of exchangeable distance metric that compares policy objects, a generic retrieval algorithm (such as k Nearest Neighbors) that works with any distance metric and methods to store and update a knowledge base. Being a part of an ongoing research project, reusability and modularity are important success factors for in evaluating the project. This means that it should for instance be simple to swap P3P with some other privacy policy standard, that different distance metrics should be applicable, new metrics could easily be implemented and so forth.


\section{Quality Assurance}

This section describes measures to be done to assure that the project is able to reach the quality level expected by establishing internal routines as well as reporting from the project group to ouside interests, that is, the customer and the project advisor.

\subsection{Status Reporting}

\subsubsection{External Reporting}
The project manager is responsible for producing weekly status reports detailing the progress towards the objectives that are established in this and other planning documents. The status report forms the basis for discussions in weekly advisor meetings held on Wednesdays at 14.15 in ITS464 at NTNU and in on a semi-weekly basis in customer meetings with SINTEF.

\subsubsection{Internal Reporting}
For internal reporting, each team member is to keep a time sheet tracking the amount of time spent on different project activities. These time sheets are to be updated on a weekly basis so as to keep track of progress in accordance with the project plan. Furthermore, internal project team meetings are held prior to advisor and customer meetings in order to keep the status report up to date.

\subsubsection{Templates}
For the abovementioned reporting procedures, a set of templates have been worked out. The templates encourage reporting according to a particular standards, which makes document preparation and reading easier. Examples of these templates are those for meeting minutes and status reports. All templates are stored in the document respository and are available for all project members.

Finally, a simple Java code template has been agreed on. This template illustrates key points from Sun Microsystems' coding standards. Strict adherence to a proper standard facilitates among other things the generation of JavaDoc documentation.

\subsection{Response Timelines}
To ensure an efficient decision making process and ensure that important tasks receive proper attention, some simple routines have been established:
\begin{description}
\item[Meetings:] Meetings are to be called by e-mail 24 hours prior to the meeting. The invitation should include an agenda as well as any other documents relevant to the discussion.
\item[Meeting minutess:] The project leader is responsible for assigning the task of taking minutes at meetings. The minutes are to be made available at the document repository within a 24 hours of the meeting.
\end{description}

\subsection{Risk Management}
A risk report is to be produced as a part of the project plan. The risks are to be identified by severity and probability as well as the activities that are touched by the risk. The risk reporting phase will also identify a group member responsible for managing and mitigating the particular risk, and serve as an input to planning so as to allow for a certain degree of slack.

\subsection{Document Respository}
\textbf{Google Docs} has been chosen to serve as repository for documents such as status reports, meeting minutes, time sheets, and other frequently edited documents. Google Docs allows for simultaneous real-time editing and collaboration and has basic spreadsheet and word-processing functionality. All project interests, including customer and project advisor, are granted read-write access to the Google Docs repository to review plans, notes and comment on these.

