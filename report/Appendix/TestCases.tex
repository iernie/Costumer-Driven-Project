% Activate the following line by filling in the right side. If for example the name of the root file is Main.tex, write
% "...root = Main.tex" if the chapter file is in the same directory, and "...root = ../Main.tex" if the chapter is in a subdirectory.
 
%!TEX root =  

\chapter{Test cases}

\begin{center}
	\begin{longtable}{ | p{4cm} | p{10cm} | }
	\caption{UNIT-01}\\ \hline
	\textbf{Item} & \textbf{Description} \\ [3pt] \hline \hline
	\endfirsthead
	\multicolumn{2}{c}%
	{\tablename\ \thetable\ -- \textit{Continued from previous page}} \\ \hline
	\textbf{Item} & \textbf{Description}\\ \hline
	\endhead \hline \hline 
	\multicolumn{2}{r}{\textit{Continued on next page}} \\
	\endfoot \hline
	\endlastfoot
				Name & Command line interface (CLI) functionality \\  [3pt] \hline
				Test identifier & UNIT-01 \\  [3pt] \hline
				Person responsible & Henrik Knutsen \\  [3pt] \hline
				Feature(s) to be tested & That all possible commands are working correctly when using the CLI. That the program runs without input \\ [3pt] \hline
				Pre-conditions & Code for input handling for all possible commands. Code for error handling for invalid inputs. \\  [3pt] \hline
				
				Execution steps & 	\begin{enumerate}
							\item Run the program for every type of argument
							\item Run the program without arguments
						\end{enumerate} \\ [3pt] \hline

				Expected results & 	\begin{enumerate}
								\item The specified variable is set to the specified value
								\item All the values are loaded from the config file
							\end{enumerate} \\ [3pt] \hline
	\end{longtable}
\end{center}

\newpage
\begin{center}
	\begin{longtable}{ | p{4cm} | p{10cm} | }
	\caption{UNIT-02}\\ \hline
	\textbf{Item} & \textbf{Description} \\ [3pt] \hline \hline
	\endfirsthead
	\multicolumn{2}{c}%
	{\tablename\ \thetable\ -- \textit{Continued from previous page}} \\ \hline
	\textbf{Item} & \textbf{Description}\\ \hline
	\endhead \hline \hline 
	\multicolumn{2}{r}{\textit{Continued on next page}} \\
	\endfoot \hline
	\endlastfoot
				Name & P3P parser \\  [3pt] \hline
				Test identifier & UNIT-02 \\  [3pt] \hline
				Person responsible & Henrik Knutsen \\  [3pt] \hline
				Feature(s) to be tested & That the P3P parser correctly parses all the required fields and their values. \\  [3pt] \hline
				Pre-conditions & The P3P parser must be implemented. Need to have P3P policies with a wide range of cases. \\  [3pt] \hline
				
				Execution steps & 	\begin{enumerate}
								\item Run a P3P xml in the P3P parser and print the parsed fields and their values to console
								\item Manually compare the printed fields and values with the contents of the P3P xml
				\end{enumerate} \\ [3pt] \hline
	
				Expected results &	\begin{enumerate}	
								\item The P3P xml is parsed successfully. It's content is printed to console
								\item The printed output have the same fields, each having the same value as those in the xml
							\end{enumerate}
							 \\  [3pt] \hline
	\end{longtable}
\end{center}
		
\newpage
\begin{center}
	\begin{longtable}{ | p{4cm} | p{10cm} | }
	\caption{UNIT-03}\\ \hline
	\textbf{Item} & \textbf{Description} \\ [3pt] \hline \hline
	\endfirsthead
	\multicolumn{2}{c}%
	{\tablename\ \thetable\ -- \textit{Continued from previous page}} \\ \hline
	\textbf{Item} & \textbf{Description}\\ \hline
	\endhead \hline \hline 
	\multicolumn{2}{r}{\textit{Continued on next page}} \\
	\endfoot \hline
	\endlastfoot
				Name & Local database \\  [3pt] \hline
				Test identifier & UNIT-03 \\  [3pt] \hline
				Person responsible & Henrik Knutsen \\  [3pt] \hline
				Feature(s) to be tested & Writing to and reading from the local database. That the serialization of the database is working. \\  [3pt] \hline
				Pre-conditions & Code for writing to and reading from the database file. Need to have two different P3P policies. \\  [3pt] \hline
				
			Execution steps & 	\begin{enumerate}
							\item Write policy A to the local database
							\item Write policy B to the local database
							\item Read and print policy A from the local database
							\item Read and print policy A from the local database
							\item Compare the written policy A and the read policy A
							\item Compare the written policy B and the read policy B
						\end{enumerate} \\ [3pt] \hline
			
			Expected results &	\begin{enumerate}
							\item Policy A is successfully written to the database file
							\item Policy B is successfully written to the database file
							\item Policy A is successfully read from the database file and printed
							\item Policy B is successfully read from the database file and printed
							\item The written policy A and the read policy A are identical. They both have the same fields, with the same values
							\item The written policy B and the read policy B are identical. They both have the same fields, with the same values
						\end{enumerate}
							 \\  [3pt] \hline
	\end{longtable}
\end{center}

\newpage
\begin{center}
	\begin{longtable}{ | p{4cm} | p{10cm} | }
	\caption{UNIT-04}\\ \hline
	\textbf{Item} & \textbf{Description} \\ [3pt] \hline \hline
	\endfirsthead
	\multicolumn{2}{c}%
	{\tablename\ \thetable\ -- \textit{Continued from previous page}} \\ \hline
	\textbf{Item} & \textbf{Description}\\ \hline
	\endhead \hline \hline 
	\multicolumn{2}{r}{\textit{Continued on next page}} \\
	\endfoot \hline
	\endlastfoot
				Name & Graphical user interface (GUI) functionality \\  [3pt] \hline
				Test identifier & UNIT-04 \\  [3pt] \hline
				Person responsible & Henrik Knutsen \\  [3pt] \hline
				Feature(s) to be tested & That all the elements - buttons, lists etc. - are working as intended. \\  [3pt] \hline
				Pre-conditions & GUI with all the necessary listeners must be implemented. Code for running the program with the GUI. \\  [3pt] \hline
				
			Execution steps & 	\begin{enumerate}
							\item Run the program using the GUI
							\item Test every option in the menu bar
							\item Test every button in the configuration menu
							\item Test every scroll bar
							\item Resize the window
						\end{enumerate} \\ [3pt] \hline
			
			Expected results &	\begin{enumerate}
							\item The program starts and loads the graphical user interface
							\item The code connected to the option is executed
							\item Every button is running the connected code															\item Every scroll bar scrolls through the list, up and down, from end to end, successfully
							\item Window can be resized without having elements of the GUI overlapping. The elements and panes scales with the main window
						\end{enumerate}
							 \\  [3pt] \hline
	\end{longtable}
\end{center}

\newpage
\begin{center}
	\begin{longtable}{ | p{4cm} | p{10cm} | }
	\caption{UNIT-05}\\ \hline
	\textbf{Item} & \textbf{Description} \\ [3pt] \hline \hline
	\endfirsthead
	\multicolumn{2}{c}%
	{\tablename\ \thetable\ -- \textit{Continued from previous page}} \\ \hline
	\textbf{Item} & \textbf{Description}\\ \hline
	\endhead \hline \hline 
	\multicolumn{2}{r}{\textit{Continued on next page}} \\
	\endfoot \hline
	\endlastfoot
				Name & Algorithm classification \\  [3pt] \hline
				Test identifier & UNIT-05 \\  [3pt] \hline
				Person responsible & Henrik Knutsen, Dimitry Kongevold \\  [3pt] \hline
				Feature(s) to be tested & That the k-nearest neighbor algorithm bases its decision on the k most similar policies \\  [3pt] \hline
				Pre-conditions & Code for reading from the weights file must be implemented. A working k-nearest neighbor algorithm that uses the weights must be implemented. Need one policy to test on, and a set of policies to be used as history. \\  [3pt] \hline
				
Execution steps & 	\begin{enumerate}
							\item Load a set of policies into the database file
							\item Manually calculate and write down the distances between the single policy and each of the policies in the history
							\item Run the distance algorithm on a single policy and the history and compare the distances that are returned by the algorithm with the manually calculated distances from step 2
							\item Manually find the k policies with the lowest distances
							\item Run the reduction algorithm with necessary input to find the k nearest policies and compare the k policies returned by the reduction algorithm with those found in step 4
							\item Run the conclusion algorithm and verify the results returned by the algorithm
						\end{enumerate} \\ [3pt] \hline
			
			Expected results &	\begin{enumerate}
							\item The policies are added to the database file
							\item The six distances are obtained
							\item The algorithm returns the same distances as those found in step 2
							\item The k policies are obtained
							\item The algorithm returns the same k policies as those found in step 4
							\item The algorithm is giving the correct recommendation and confidence value
						\end{enumerate}
							 \\  [3pt] \hline
	\end{longtable}
\end{center}

\newpage
\begin{center}
	\begin{longtable}{ | p{4cm} | p{10cm} | }
	\caption{UNIT-06}\\ \hline
	\textbf{Item} & \textbf{Description} \\ [3pt] \hline \hline
	\endfirsthead
	\multicolumn{2}{c}%
	{\tablename\ \thetable\ -- \textit{Continued from previous page}} \\ \hline
	\textbf{Item} & \textbf{Description}\\ \hline
	\endhead \hline \hline 
	\multicolumn{2}{r}{\textit{Continued on next page}} \\
	\endfoot \hline
	\endlastfoot
				Name & Algorithm learning \\  [3pt] \hline
				Test identifier & UNIT-06 \\  [3pt] \hline
				Person responsible & Henrik Knutsen, Neshahavan Karunakaran\\  [3pt] \hline
				Feature(s) to be tested & That the weights file is updated when a new policy is added to history. \\  [3pt] \hline
				Pre-conditions & Code for reading from and writing to the weights file. Code for writing to the database. Algorithms for classification and learning must be implemented. \\  [3pt] \hline

Execution steps & 	\begin{enumerate}
							\item Make a set of policies and load the set into the history
							\item Read the weights from the weights file
							\item Run the classification and learning algorithms on the policy to be classified and the history, with the weights from step 2
							\item Read the weights from the weights file
							\item Compare the contents of the weights files obtained in steps 2 and 4
						\end{enumerate} \\ [3pt] \hline
			
			Expected results &	\begin{enumerate}
							\item The policies are loaded into the history successfully
							\item The weights are written down
							\item The classification and learning algorithm runs successfully on the policy to be classified and the history
							\item The weights are loaded
							\item The weights loaded in step 4 are different from the weights written down in step 2
						\end{enumerate}
							 \\  [3pt] \hline
	\end{longtable}
\end{center}

\newpage
\begin{center}
	\begin{longtable}{ | p{4cm} | p{10cm} | }
	\caption{UNIT-07}\\ \hline
	\textbf{Item} & \textbf{Description} \\ [3pt] \hline \hline
	\endfirsthead
	\multicolumn{2}{c}%
	{\tablename\ \thetable\ -- \textit{Continued from previous page}} \\ \hline
	\textbf{Item} & \textbf{Description}\\ \hline
	\endhead \hline \hline 
	\multicolumn{2}{r}{\textit{Continued on next page}} \\
	\endfoot \hline
	\endlastfoot
				Name & Interaction with community databas \\  [3pt] \hline
				Test identifier & UNIT-07 \\  [3pt] \hline
				Person responsible & Henrik Knutsen \\  [3pt] \hline
				Feature(s) to be tested & That policies can be sent to and saved on the community database. That the community database returns a recommendation \\  [3pt] \hline
				Pre-conditions & A running local client and a server. Code for sending policies and must be implemented locally, and code for calculating and returning a recommendation must be implemented on the server. \\  [3pt] \hline
				
			Execution steps & 	\begin{enumerate}
							\item Upload a policy to the community database
							\item Check that the database returns a recommendation
						\end{enumerate} \\ [3pt] \hline
			
			Expected results &	\begin{enumerate}
							\item The policy to be classified is saved on the community database
							\item The community database returns a recommendation based on the policy to be classified and the database history
						\end{enumerate}
							 \\  [3pt] \hline
	\end{longtable}
\end{center}