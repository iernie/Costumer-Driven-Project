% Activate the following line by filling in the right side. If for example the name of the root file is Main.tex, write
% "...root = Main.tex" if the chapter file is in the same directory, and "...root = ../Main.tex" if the chapter is in a subdirectory.
 
%!TEX root =  

\chapter{Test cases}
	\setcounter{tocdepth}{1}

	\section{Test cases}
		\vspace{8 mm}		

		\begin{center}
			\begin{tabular}{ |  p{3.5cm} | p{10cm} | }
				\hline
				Item & Description \\ [5pt] \hline \hline
				Name & Command line interface (CLI) functionality \\  [5pt] \hline
				Test identifier & UNIT-01 \\  [5pt] \hline
				Person responsible & Henrik Knutsen \\  [5pt] \hline
				Feature(s) to be tested & All possible commands when running the program with the CLI. That input with incompatible commands does not run. \\  [5pt] \hline
				Pre-conditions & Code for input handling for all possible commands. Error handling for invalid inputs. \\  [5pt] \hline
				Execution steps & 1. Test all the commands one by one. \newline 2. Test the combinations of invalid inputs. \\  [5pt] \hline
				Expected result & 1. The program runs using the input variables. \newline 2. Error warning: Invalid arguments. Abort startup. \\  [5pt] \hline
			\end{tabular}
		\end{center}

		\begin{center}
			\begin{tabular}{ |  p{3.5cm} | p{10cm} | }
				\hline
				Item & Description \\ [5pt] \hline \hline
				Name & P3P parser \\  [5pt] \hline
				Test identifier & UNIT-02 \\  [5pt] \hline
				Person responsible & Henrik Knutsen \\  [5pt] \hline
				Feature(s) to be tested & That the P3P parser correctly parses all the required fields and their values. \\  [5pt] \hline
				Pre-conditions & The P3P parser must be implemented. Need to have P3P policies with a wide range of cases. \\  [5pt] \hline
				Execution steps & 1. Manually go through a P3P XML and obtain all the required fields and the values. \newline 2. Run the same P3P XML in the P3P parser and print the parsed elements and their values to console.
					\newline 3. Compare the results from the two parsing methods. \\  [5pt] \hline
				Expected result & The two parsing methods give identical output. They must both have the same fields, each containing the same values \\  [5pt] \hline
			\end{tabular}
		\end{center}
		
		\begin{center}
			\begin{tabular}{ |  p{3.5cm} | p{10cm} | }
				\hline
				Item & Description \\ [5pt] \hline \hline
				Name & Local database \\  [5pt] \hline
				Test identifier & UNIT-03 \\  [5pt] \hline
				Person responsible & Henrik Knutsen \\  [5pt] \hline
				Feature(s) to be tested & Writing to and reading from the local database. That the serialization of the database is working. \\  [5pt] \hline
				Pre-conditions & Code for writing to and reading from the database file must be implemented. Need to have two different P3P policies. \\  [5pt] \hline
				Execution steps & 1. Write policy A to the local database. \newline 2. Write policy B to the local database. \newline 3. Read policy A from the local database. \newline 4. Read policy B from the local database. \\  [5pt] \hline
				Expected result & The written policy A and the read policy A must be identical. \newline The written policy B and the read policy B must be identical. \\  [5pt] \hline
			\end{tabular}
		\end{center}

		\begin{center}
			\begin{tabular}{ |  p{3.5cm} | p{10cm} | }
				\hline
				Item & Description \\ [5pt] \hline \hline
				Name & Graphical user interface (GUI) functionality \\  [5pt] \hline
				Test identifier & UNIT-04 \\  [5pt] \hline
				Person responsible & Henrik Knutsen \\  [5pt] \hline
				Feature(s) to be tested & That all the interactable elements, buttons, lists etc., is working as intended. \\  [5pt] \hline
				Pre-conditions & GUI with all the necessary listeners must be implemented. Code for running the program with the GUI must be implemented. \\  [5pt] \hline
				Execution steps & 1. Run the program using the GUI. \newline 2. Test all the interactable elements. \\  [5pt] \hline
				Expected result & All the interactable elements is triggering the right methods when used. \\  [5pt] \hline
			\end{tabular}
		\end{center}

		\begin{center}
			\begin{tabular}{ |  p{3.5cm} | p{10cm} | }
				\hline
				Item & Description \\ [5pt] \hline \hline
				Name & Algorithm classification \\  [5pt] \hline
				Test identifier & UNIT-05 \\  [5pt] \hline
				Person responsible & Henrik Knutsen \\  [5pt] \hline
				Feature(s) to be tested & That the k-nearest neighbor algorithm bases its decision on the k most similar policies \\  [5pt] \hline
				Pre-conditions & Code for reading from the weights file must be implemented. A working k-nearest neighbor algorithm that uses the weights must be implemented. Need one policy to test on, and a set of policies to be used as history. \\  [5pt] \hline
				Execution steps & 1. Load a set of policies into the history. \newline 2. Run the k-nn algorithm on the policy to be classified and the history. \newline 3. Manually go through the policies and verify the output of the algorithm. \\  [5pt] \hline
				Expected result & The algorithm finds the most similar policy. \\  [5pt] \hline
			\end{tabular}
		\end{center}

		\begin{center}
			\begin{tabular}{ |  p{3.5cm} | p{10cm} | }
				\hline
				Item & Description \\ [5pt] \hline \hline
				Name & Algorithm learning \\  [5pt] \hline
				Test identifier & UNIT-06 \\  [5pt] \hline
				Person responsible & Henrik Knutsen \\  [5pt] \hline
				Feature(s) to be tested & That the weights file is updated when a new policy is added to history. \\  [5pt] \hline
				Pre-conditions & Code for reading from and writing to the weights file must be implemented. Algorithms for classification and learning must be implemented. \\  [5pt] \hline
				Execution steps & 1. Get the contents of the weights file. \newline 2. Load a set of policies into the history. \newline 3. Run the classification algorithm on the single policy and the history. \newline 4. Choose to store the new policy, the context and the action.
					\newline 5. Get the contents of the weights file. \newline 6. Compare the contents of the weights files obtained in steps 1. and 5. \\  [5pt] \hline
				Expected result & The two weights files obtained in steps 1. and 5. are different \\  [5pt] \hline
			\end{tabular}
		\end{center}

		\begin{center}
			\begin{tabular}{ |  p{3.5cm} | p{10cm} | }
				\hline
				Item & Description \\ [5pt] \hline \hline
				Name & Packet passing through network to community database \\  [5pt] \hline
				Test identifier & UNIT-07 \\  [5pt] \hline
				Person responsible & Henrik Knutsen \\  [5pt] \hline
				Feature(s) to be tested & That packets can be sent between the client program and the community database.  \\  [5pt] \hline
				Pre-conditions & A running local client. A (virtual) server. Code for sending and receiving packets must be implemented. \\  [5pt] \hline
				Execution steps & 1. Start the program locally. \newline 2. Start the (virtual) server. \newline 3. Send packet A from the local client. \newline 4. Receive packet A at the (virtual) server. \newline 5. Send packet B from the (virtual) server.
					\newline 6. Receive packet B at the local client. \\  [5pt] \hline
				Expected result & The received packet A is identical to the sent packet A. The received packet B is identical to the sent packet B. \\  [5pt] \hline
			\end{tabular}
		\end{center}
