\documentclass[11pt]{book}
%\usepackage{utf8}{inputenc} % For Norwegian characters.
\usepackage{geometry}        
\geometry{letterpaper}    
\usepackage[parfill]{parskip}  
\usepackage{graphicx}
\usepackage{amssymb}
\usepackage{epstopdf}
\usepackage{tabularx}
\usepackage{minitoc}
\usepackage{wrapfig}
\usepackage{pdfpages}
\usepackage{longtable}
\usepackage{subfig}
\usepackage{makeidx}
\usepackage{fancyhdr}

\DeclareGraphicsRule{.tif}{png}{.png}{`convert #1 `dirname #1`/`basename #1 .tif`.png}

\usepackage[colorlinks=true, pdfstartview=FitV, linkcolor=blue, 
            citecolor=blue, urlcolor=blue]{hyperref}
            
\usepackage[T1]{fontenc}
\usepackage[ansinew]{inputenc}
\usepackage[english]{babel}
\usepackage{csquotes}
\usepackage[subsection,toc3]{javadoc}


%\pagestyle{fancy}

\pagestyle{fancyplain}
\setlength{\headheight}{15pt} 
\fancyhf{} 
\lhead{\fancyplain{}{\footnotesize\leftmark}}
%\chead{\fancyplain{}{\footnotesize\chaptername}}
\rhead{\fancyplain{}{\footnotesize\rightmark}}

%\lfoot{\fancyplain{}{\footnotesize\thechapter}}
\cfoot{\fancyplain{}{\footnotesize\thepage}}
%\rfoot{\fancyplain{}{\footnotesize\thesection}}

\newtheorem{theorem}{Theorem}
\newtheorem{corollary}[theorem]{Corollary}
\newtheorem{definition}{Definition}
\newtheorem{lemma}{Lemma}
\newtheorem{exercise}{Exercise}
\newtheorem{remark}{Remark}
\newtheorem{example}{Example}
\newtheorem{warning}{Warning}

\def\grad{ \mbox{grad}}
\def\curl{ \mbox{curl}}
\def\div{ \mbox{div}}
\def\U{\ensuremath {\cal U}}
\def\S{\ensuremath {\cal S}}
\def\V{\ensuremath {\cal V}}
\def\R{\ensuremath {\cal R}}
\def\tr{\ensuremath {\mbox{tr}}}

% ------------------- Title and Author -----------------------------
\title{Project Report\\
TDT4290 - Customer Driven Project \\ 
"Privacy Advisor"}

\author{GROUP 4\\
Ulf Nore, Nicholas Gerstle, Henrik Knutsen, Dimitry Kongevold,\\ 
Einar Afiouni, Neshahavan Karunakaran, Amanpreet Kaur}

\date{NTNU, Fall 2011}


\makeindex
\begin{document}

\dominitoc 

\frontmatter
\maketitle

\chapter*{\centering Abstract}
%background
This report details the design, development and implementation of the
"Privacy Advisor" software in fulfillment of course TDT4290-Customer
Driven Project. Privacy Advisor is developed as part of a larger project for SINTEF ICT, in
order to investigate the applicability of machine learning to aid users in
making internet privacy decisions. 

% method
Previous research at SINTEF has pointed
out several advantages of using a Case Based Reasoning (CBR) agent as
a promising machine learning technique in the field of privacy advice,
which is also what the Privacy Advisor is based on. SINTEF ICT
has indicated the use of the project software as part of future
research and therefore emphasis is placed on implementing a broad
and modular testing framework, rather than actual algorithms.

% result
The resulting product is a Java based framework with a central CBR
engine that is extendable with respect to key algorithms, databases
and user interfaces. It allows for network communication with a
community server that could implement a collaborative filtering approach 
similar to the CBR.

% conclusion
Privacy Advisor is aimed towards future research in privacy
enhancement techonology and is therefore build as a testing
framework. The core engine however, is flexible enough to be expanded
on for future development.



%more details in results?

%\dominitoc

\listoftables

\listoffigures

\tableofcontents \label{toc}

\pagenumbering{arabic}


\chapter{Introduction}

This report describes the development of a machine learning systems for aiding users in Internet privacy decisions. The software is titled "Privacy Advisor" and uses a case based reasoning (CBR) approach, which is a learning method that seeks to predict preferences based on previous user choices. The project is part of the course TDT4290 - Customer Driven Project and is part of a SINTEF ICT research project in privacy agents. The report is written in a chronological order where each chapter represents a distinct "phase" in the development process. In reality, of course, there are no crisp boundaries, but the structure here provides a useful structure for reasoning about the process. 

The structure of the report is as follows. Part~\ref{p1} describes the initial phase of the project; the projects directive, the planning phase, and the preliminary study phase. The project directive gives a high level overview over the project, its objectives, how they are reached, project scope, resources available etc. The preliminary study comprises the first weeks of the project, and is a consistent effort to better grasp the problem at hand, identify how it can be broken into subproblems and what tools are required to solve these problems. It also seeks to identify to some extent what the priorities of the project are; that is, given scarce resources, which objectives are prioritized. The project planning phase seeks to make more fine grained decisions on how resources are best allocated over time and to the various tasks that comprise the project.

Part~\ref{p2} describes the requirements specification, the design phase, implementation and documentation. The requirements specification is a contract between the customer and the project team where the the requirements to be satisfied by the software are stated explicitly. Based on the requirements, a design is made, which henceforth serves as a guideline for implementing the software system.

% TODO
Finally, Part~\ref{p3} describes the software testing stage. It also
contains a chapter evaluating the project as a whole and a summary of
the project work. It also looks at future extensions and some
challenges that need to be addressed in future development.

\part{Initial Phase - Planning and Research}\label{p1}
\addtocounter{chapter}{1}
\setcounter{section}{0}
% Activate the following line by filling in the right side. If for example the name of the root file is Main.tex, write
% "...root = Main.tex" if the chapter file is in the same directory, and "...root = ../Main.tex" if the chapter is in a subdirectory.
 
%!TEX root =  

\chapter{Project Directive}

\minitoc

\section{Purpose}
This document describes the mandate, background, resources available and organizational structure of the project �A Privacy Advisor�, henceforth referred to as �the project�.

\section{Mandate}
The purpose of this project is to implement the key functionality of a privacy agent as described in Nyre and T�ndel (2010), that provides users with advice in making Internet privacy decision. 


\subsection{Background}
This project is a part of a larger research project at SINTEF ICT that studies approaches to handling Internet privacy related issues. The underlying idea is that while users are often concerned about the way various websites and services handle private information about them, obtaining information about this is very costly as privacy policies tend to be very long documents formulated in an inaccessible language. This has led to the idea that Internet privacy can be handled by machine learning techniques, where a particular decision is based on the user�s past behavior and the behavior of similar users.

Nyre and T�ndel has then proposed a �Privacy Agent� structure that uses the case based reasoning (CBR) method for giving privacy advice. CBR is in many ways similar to the way human experts reason about problems, and is usually described as a process in four stages (here related to the privacy decision problem):
\begin{itemize}
\item Retrieve: 
Given a new site, the agent will retrieve from its knowledge base, the set of cases deemed the most similar to the one at hand. This means, that if presented with the site Facebook, for instance, the agent finds Twitter, Google and LinkedIn to be the sites that have the most similar policy to Facebook.
\item Reuse: 
Look at the decisions made about the cases that were found and adapt this decision to the problem at hand. In this case, the agent needs to see if there are strong enough indications toward a particular behavior with respect to the type of site that is at hand. If for instance, the user has accepted the policies of all the similar sites found, he is also likely to accept that of Facebook.
\item Revise: 
Once a conclusion is reached, it is presented to the user (along with the background for why it is reached). The user may then choose to accept the conclusion, or to overrule it, providing the system with directions as to why it was wrong. This may in turn cause the agent to update its parameters accordingly. 
\item Retain: Finally, the new case is stored to the database along with the user�s decision as a reference for the next time the same site is opened, and as a case to employ when evaluating new sites.
\end{itemize}

Nyre and T�ndel also describes this local CBR approach to be complemented by a community database where the same information is stored, allowing for a second lookup that uses a collaborative filtering, that is, making a decision based on the behavior of similar users.

\section{Objectives}
This project identifies three key objective, arranged by order of importance:
Implementing a testing framework of CBR based privacy agent that is able to make privacy decisions based on previous user behavior.
Implement the community system/collaborative filtering part of the agent.
Extend the system to other standards for machine readable privacy policies.
Implement the system as a browser plugin. This is considered least important, and is contingent on the success of early testing. It is also given a low priority given the relative small portion of major websites that implement P3P.


\section{Resources and Duration}
The system in its complete form is to be demonstrated on November 24 2011. For the project period, a total of 25 hours per week per project member is planned. With seven group members and a project spanning 13 weeks, this adds up to approximately 2300 hours.

\section{Organization}
The project group organization is based on the modules of the system that is being implemented. One group member is responsible for developing one particular feature. This organization is shown in Table~\ref{orgTable}.

\begin{table}[htdp]
\caption{Responsibilities}
\begin{center}
\begin{tabularx}{\textwidth}{| X | X | X |}
\hline
\textbf{Role} & \textbf{Description} &\textbf{Responsible} \\
\hline
Administrative 	&  Customer relations  & Ulf Nore \\
			& Requirement specification & \\
			& Requirement specification & \\
			& Planning documents & \\
			& Meeting minutiae & \\
			& Project report & \\	
\hline					
Software design and Architecture   	&  UML modeling.	 	& Nicholas Gerstle \\ 
						      	& Design report.		& \\
							& User documentation.	& \\
							& Technical Decisions.	& \\
\hline
Data Storage/Databases	& Flat file data storage system. 	& Amanpreet Kaur \\
					& Database systems	.			& \\
\hline
CBR - Algorithms and Data structures 	& Data structures for storing privacy policy information.	& Dimitry Kongevold,\\
								& Define and implement similarity metrics.			& Neshahavan Karunakaran \\
								& Retrieval and learning algorithms. &	\\
								& Parameter storage. & \\
\hline
Testing and Evaluation 	& Design test cases. & Henrik Knutsen \\
					& Criteria/methodology for model testing. & \\
\hline
GUI & Implement a simple GUI for testing model framework. & Ulf Nore \\
\hline
Version control & Set up and maintain code repository. & Einar Afiouni \\
\hline
XML/P3P Parser & Implement P3P parser that produces inputs to CBR & Einar Afiouni \\
\hline
Version control & Set up and maintain code repository. & Einar Afiouni \\								 	
\hline
\end{tabularx}
\end{center}
\label{orgTable}
\end{table}%



\section{Planning}
A project plan has been developed for the purpose of communicating expectations and progress within the group and to the customer and the advisor. The plan also serves as an aid in identifying problems and project management. For the software development process, a hybrid waterfall model has been chosen.


\section{Limitations and Scope}
The primary focus of this project is on developing a framework that allows for testing the CBR privacy agent framework. This entails building a module for parsing policy documents in XML format, a data structure for holding policy information in memory (henceforth �policy objects�), a set of exchangeable distance metric that compares policy objects, a generic retrieval algorithm (such as k Nearest Neighbors) that works with any distance metric and methods to store and update a knowledge base. Being a part of an ongoing research project, reusability and modularity are important success factors for in evaluating the project. This means that it should for instance be simple to swap P3P with some other privacy policy standard, that different distance metrics should be applicable, new metrics could easily be implemented and so forth.


\setcounter{section}{0}
\addtocounter{chapter}{1}
% Activate the following line by filling in the right side. If for example the name of the root file is Main.tex, write
% "...root = Main.tex" if the chapter file is in the same directory, and "...root = ../Main.tex" if the chapter is in a subdirectory.
 
%!TEX root =  

\chapter{Preliminary Study\label{prelim}}


\minitoc

This section describes the preliminary study phase of the project which is the time spent by the project team gaining insight into the problem the software system is to solve. For this project, the problem is very well defined from the customer's perspective, so little time is spent looking into similar products and various possible solutions to the overarching problem of computer aided privacy advice. A large portion of this time was spent on investigating the proposed solution and the technologies involved, that is Case Based Reasoning (CBR) and the technologies used by it. Another critical work laid down in this phase were choices with respect to project scope.

\section{Privacy Technology}\label{privTech}
This section briefly describes the situation today with respect to Internet privacy tools and which needs the project seeks to address.

As discussed in T�ndel and Nyre (2010), there is a disproportion between the average Internet user's concern for privacy and the actual control he has over private information. While most websites provide a privacy policy, these tend to very long documents written in a obscure "legalesque" language meant more for protecting the websites' interests than those of users. To aid this problem, machine readable standards such as P3P have been introduced. P3P seeks to compress the information contained in the privacy policy in an XML document that can be parsed and summarized by computer programs. 

The AT\&T \emph{Privacy Bird} is mentioned in T�ndel and Nyre (2010) as an example of current Internet privacy software. Privacy Bird can parse P3P documents and match a site's privacy policy with the user's preferences. The problem with this however, is that the user has to explicitly state his preferences, which is a rather time-consuming endeavor. The novel feature of this project is to introduce an intelligent system that \emph{learns} the user's preferences, seeking to limit the amount of user interference. This is to be acheived by the use of CBR agent, that can look at previous examples of user choices in similar situations. A particular advantage of the CBR approach is the feedback loop where the system can actually \emph{explain} its choice in terms of similar cases which sets CBR apart from alternative reasoning models such as artificial neural networks. It also allows for better tuning to user response in those cases that the user disagrees with the recommendation.

\section{Project Scope}
The research nature of this project makes it a very open-ended one. The clearly most important part of the system envisioned by the customer is the CBR engine, that classifies websites based on the user's previous decisions. Implementing this is clearly necessary, regardless of which direction is followed and what limitations are made.

However, once the local CBR engine and auxiliary modules such as parsing P3P documents and so forth are in place, there are several directions that the project can take, and pursuing all of them is an unlikely scenario given the resource limitations. To some extent, the direction of the project also depends in a large extent on how well preliminary testing goes; that is, how well does the CBR system actually predict user preferences. Depending on the results of these tests, several possible directions were deemed possible:

\begin{enumerate}
\item Given test failure, making improvements to the algorithm, hereunder, the retrieval methods, the amount of data stored per case, the distance metric used to compare cases, and the weighting of the different features of a case/policy.
\item Extending the system by implementing the community portion/collaborative filtering part of the proposed system.
\item Given a test success, implementing a working browser plugin.
\item Closely related to the previous point, extending the system to work with other privacy policy standards.
\end{enumerate}

Direction 1 above takes the project from a system engineering direction more over to a scientific and statistical analysis type project, and while placing some emphasis on tuning the algorithms, this is not our primary focus\footnote{This would require gathering a dataset of some size as well as setting up testing scenarios, which not only requires a sophisticated statistics background, but also likely group of test users. It was decided that this should not be prioritized in a software engineering project of limited scope such as this.}. Item 3 relies heavily on the effectiveness of the algorithm and may require exactly the type of statistical analysis discussed. We therefore ended up pursuing the implementation of the community portion of the system. 

\section{Choice of Development Method}
Having briefly identified the project "flow", a development method or model should be chosen. We settled on a hybrid model, which is basically a waterfall where the design-implementation-testing phase is repeated two times allowing for some flexibility with respect to the uncertainties discussed in Section~\ref{privTech}. The workflow is illustrated in Figure~\ref{workflow}.

\begin{figure}[htbp]
\begin{center}
\caption{Development Model}
\includegraphics[width = 1.0\textwidth]{PrelimStudy/workflow}
\label{workflow}
\end{center}
\end{figure}


\setcounter{section}{0}
\addtocounter{chapter}{1}
% Activate the following line by filling in the right side. If for example the name of the root file is Main.tex, write
% "...root = Main.tex" if the chapter file is in the same directory, and "...root = ../Main.tex" if the chapter is in a subdirectory.
 
%!TEX root =  

\chapter{Planning Phase}
\label{plan}

\minitoc 

\subsection*{Purpose}
This chapter details the planning process that seeks to identify the different phases in the development process and allocate resources over time to the various activities that comprise each phase. Planning also needs to account for a number of risk factors that may impact the process, either by allowing for enough preventive measures or by allocating extra time to activities that may be affected. Thus a risk report enumerating several potential risks has been worked out.

\section{Project Phases}\label{phases}
The choice of development model is detailed in Chapter \ref{prelim}. Here the various phases of the project are described.

\subsection{Preliminary Study and Research}
In this phase the aim is for each project member to acquire a certain level domain knowledge in the field of Internet privacy and to learn the necessary technology and tools required to implement the model as proposed by the customer. This entails having a working knowledge of the Java programming language, version control using Git and the CBR framework.

\subsection{Planning}
Planning seeks to identify the activities needed to reach the project objective. This entails breaking down the objectives into sub-problems, identifying the relationship between these, and allocating time for each of them. 

\subsection{Requirement Specification}
The requirements specification is a document listing the functional and non-functional requirements of the software to be developed, which is a standard that the results is to be measured against, thus serving as not only a contract between the customer and the project team, but as a basis for developing testing methods.

\subsection{Design/Architecture}
This phase consists of a broad structuring and specification of the overall system. It defines the program structure in terms of program flow, modules, classes and interfaces as well as coding standards and other conventions that will serve as guidelines for the implementation phase.

\subsection{Implementation}
In this phase the design is realized as a working Java program according to the models developed in the Design phase. 

\subsection{Evaluation and Documentation}
This phase consists of testing the system and documenting the structure of the system and how it is operated. From a software engineering perspective, the primary testing grounds are against the standards prescribed by the requirements specification rather than applicability of the applicability of the CBR agent model to privacy enhancement. As mentioned, among the primary objectives of the project is to provide a testing framework to verify the applicability of the given system in making privacy decisions.

\subsection{Ongoing Activities}

\subsubsection{Reporting and Administrative Tasks}
Under this heading are more project management related activities, such as routine organizational work (ie. arranging meetings and writing status updates), more refined distribution of tasks as the project is underway, and preparation of the project report (this document).

\subsubsection{Study and Lectures}
To solve several of the problems posed by this project, most group members have had to learn new tools and technologies. This includes, but is not limited to Case Based Reasoning, version control (Git), certain features of Java and so on. Lectures on project management  and software development are also subsumed under this heading.

\section{Risk Report}\label{riskReport}
The term \emph{risk} is usually defined as the possibility of an undesirable outcome (loss) as a consequence of a choice or an action made. 


\subsection{Overview and Risk Management}
In this section we have identified some risk factors that can impact the project. Every project does risk management at some level, wether explicit stated or not. By identifying and quantifying the \emph{likelihood} and \emph{consequence} of undesirable events, the project plan can be adapted so as to allow for certain contingencies. Risks are quantified in two dimensions on a scale from 1-5 in severity, both in terms of the probability of occurrence and in terms of the consequence for the project.

The table below contains a sample risk table depicting how risks are described, quantified and which actions are taken to mitigate the particular risk. 


\begin{table}[htdp]
\begin{center}
\begin{tabularx}{\textwidth}{| X | X |}
\hline
\textbf{Risk item} & An arbitrary number identifying the risk factor. \\
\hline
\textbf{Activity} & The activity affected by this risk. \\
\hline
\textbf{Risk Factor} & A short textual description of the risk factor. \\
\hline
\textbf{Probability} & The probability of the event occurring. Measured on a scale from 1(unlikely) to 5(almost certain).\\
\hline
\textbf{Consequence} & What the consequences of the event occurring. Measured on a scale from 1(not critical) to 5(disastrous).\\
\hline
\textbf{Risk} & Probability * Consequence\\
\hline
\textbf{Action taken} & Actions that can be taken to avoid this\\ & event occurring. \\
\hline
\textbf{Deadline} & An optional date set for taking precautions to deal with the risk. \\
\hline
\textbf{Responsible} & The group member responsible for the risk. \\
\hline
\end{tabularx}
\caption{Risk characterization.}
\end{center}
\label{riskTable}
\end{table}



\subsection{Discussion}

As shown in Appendix \ref{riskAppendix}, three broad risk categories were identified:

\begin{enumerate}
\item \textbf{Technical}: These risks pertain mainly to the  implementation; that critical success factors are not met by the implemented software.
\item \textbf{Communication}: These are risks related to miscommunication, either within the project team or between the customer and the team.
\item \textbf{Planning}: The final category has to do with planning and decisions made early on in the project phase.
\end{enumerate}

Among the more critical risks identified were a potential failure to
correctly parse policy documents as well as the risk of an improper
choice of algorithms. Failure to properly parse P3P policies could
impact progress as it would slow down and push the testing phase back
in time. This will also impact the second risk factor above, as an
early testing would reveal the weaknesses of algorithms in time.

An important issue to remedy these problems is a modular design, that
is that these two pieces of functionality are to be isolated in
separate code modules. This implies that if the initial choice of algorithms
made by the project team, while important, are not critical, as they
can be replaced during later development stages.

Other risk factors identified include database and data storage
issues, in particular pertaining to the collaborative filetering
system, and disagreements within the project team. A full listing
of the risk factors identified, and proposed measures to mitigate
these is given in Appendix~\ref{riskAppendix}.


\section{Measurement of project effects}
The primary objective of this project is to build a research prototype that allows for parsing P3P policies and provide advice using CBR given a particular knowledge base. The advice is based on: 

\begin{itemize}
\item the user's previous actions.
\item community actions or what similar users have done.
\item context of use.
\end{itemize}
   

\section{Project Plan}
As discussed in Section~\ref{phases}, the sequential part of the project is separated into six phases; pre-implementation research, requirement specification, design, implementation and documentation, evaluation, and report writing. The reporting started at the first day of the project and continues until project completion. 

Implementation is scheduled to be complete at the end of week 42, which marks a shifting of focus to testing and evaluation. The project plan was initially laid out based on \emph{evidence based scheduling}, starting out with rough estimates of each particular task pertaining to each of the project phases. 

The different components of the system were identified early in the process, and each componentsuch as core CBR and algorithms, networking, GUI and so forth were assigned as the responsibility of a team member. The responsible for each component then provides an estimate over the required time to design, implement and document the component. These estimates will in turn be based on a breakdown of the required tasks for each system. These numbers are then aggregated to form the total plan.

\subsection{Project Milestones}

In the initial plan, four important milestones were laid down to provide a clear measure of progress. The milestones concern the key project phases related to designing, implementing and documenting the system.

\begin{enumerate}
\item \textbf{Requirements specification}: September 21 (Week 38).
\item \textbf{Design}: September 28 (Week 39).
\item \textbf{Implementation and documentation}: October 19 (Week 42).
\item \textbf{Testing and evaluation}: November 2 (Week 44).
\end{enumerate}

A short span between the completion of the requirements specification and the design phases was set as these phases were deemed to be largely overlapping. Many requirements were deemed to be clear from the outset and could therefore be passed on to the design phase without waiting for the formal recognition by the customer. This improves project flow as several key activities occur at the same time.

Similar considerations were made with regard to the implementation and testing milestones. The preparatory steps for the testing work can be conducted in parallell with implementation.

\includepdf{PlanReport/DetailedPlan}


\part{Design and Implementation Phase}\label{p2}
\setcounter{section}{0}
\addtocounter{chapter}{1}
% Activate the following line by filling in the right side. If for example the name of the root file is Main.tex, write
% "...root = Main.tex" if the chapter file is in the same directory, and "...root = ../Main.tex" if the chapter is in a subdirectory.
 
%!TEX root =  

\chapter{Requirements Specification}\label{reqspec}

\minitoc

\subsection*{Purpose}
This requirements specification has been prepared for and accepted by SINTEF ICT (the customer) stating the requirements for the software system to be developed for the course TDT4290: Customer Driven Project. The requirements span two categories; functional requirements, describing the functionality the software needs to supply, and non-functional requirements, describing the development process.

The requirement specification, once accepted by the customer, will serve as a contract between the parties involved, being a guideline for design and implementation and the standard against which the product is evaluated. It could also serve as a basis for further development of the Privacy Advisor system.

\section{Introduction}
\subsection{Background and Similar Software}
SINTEF ICT is currently investigating new approaches to privacy protection of end-users. T{\o}ndel et al. (2011) proposes a specific agent design for a machine learning approach to advice users on privacy actions based on:

\begin{itemize}
\item Past behavior using case based reasoning (CBR)
\item Similar users' behavior in similar situations using collaborative filtering (CF)
\end{itemize}

While there are systems for privacy protection, and more specifically aiding users in making privacy related decisions, the majority of these systems rely in a large extent on the user pre-specifying his preferences and being prompted with messages about where the policy of a given site conflicts with the user's preferences. Our design aims at being "low profile" or "non invasive'', that is able to make sensible decisions with as little interference as possible, and at the same time, given as little feedback as possible, able to cater for the dynamic nature of both web sites' privacy policies and user preferences with respect to privacy.

\subsection{Scope}\label{reqScope}

The primary aim of this project is the implementation of the core classification system described in T{\o}ndel et al. (2011) to allow for testing the applicability of the suggested approach to predicting privacy preferences. Since the software is intended to be a part of a research project, a design that allows for testing of various hypotheses and models is required. This implies a highly modular design where the various components of the core system can be replaced for the purpose of more detailed research. 

Furthermore, given the research nature of the project, less emphasis is placed on developing a complete stand-alone application. The core focus for our project will be on developing the underlying system and an interface for testing and parameter estimation. Hence development can take two directions:

\begin{enumerate}
\item A testing system that can be fed a knowledge base consisting of input-output mappings (P3p + context -> decision), and run interactive tests on a sample where the user is allowed to give feedback to the system and see the explanation for the recommendation. We envision a dual CLI/GUI (command line interface and graphical user interface) solution for this. In a final product, this testing system can also be used for the purpose of calibrating the model.
\item An end-user system that can run as a browser plug-in giving real time advice to the user as he browses the web.
\end{enumerate}

While theoretically appealing, there is little empirical research documenting the applicability of CBR to the task at hand, which in turn implies that there is likely a large research of work to be done for the system to be able to successfully predict user preferences. Based on this observation, the emphasis of this project will be on the first of the above directions, namely providing a research framework.

\subsection{Overview}
This document is organized as follows:
�	Section~\ref{sysDesc} gives an overview over the system; its requirements and user characteristics.
�	Section~\ref{useCase} presents four different use-case scenarios.
�	Section~\ref{specRec} presents specific functional and non-functional requirements.
\section{Overall Description}

\section{System Description}\label{sysDesc}
The overall structure of the system is detailed in T{\o}ndel et al. (2011), and consists of the local CBR reasoning system, the remote/community collaborative filtering, both with their respective databases for storing information. This is in turn linked to an interface that is able to read and parse P3P policy files that are retrieved either from a local file (for the testing system) or by retrieving from the web.

\begin{figure}[htbp]
\begin{center}
\includegraphics[width = \textwidth]{DesignReport/uml/flowchart.png}
\caption{Program Flow.}
\label{ReqSpecFlow}
\end{center}
\end{figure}

\subsection{User Interface}
Because of the research nature of the project, the customer considers the user interface to be of small importance. As the underlying algorithm/methodology is in an early development phase, the core focus is placed on producing a system for model testing and evaluation rather than an end user interface. 

\subsection{Hardware and Software Interface}
Being written in Java, the software requires a local copy of the Java Runtime Environment (JRE) installed on the computer. 

For the community functionality (collaborative filtering), a dedicated server running the filtering engine must also be available. Since this is basically a modified version of the local server, it has similar requirements, but as it presumably will hold a larger knowledge base, its hardware requirements will be greater, as both lookup time (computational demands) and storage demands will increase with the number of users. It may also require additional server/database software such as mySQL, CouchDB etc. 

\subsection{User Characteristic}

For this project we distinguish between two groups constituting the users of the product. 
\subsubsection{Developers/Researchers}
Firstly, developers/researchers that will be working on the testing and calibration of the underlying model and extending it to other policy types beside P3P etc. These users are the primary focus of our work. A research/developer is an expert user, and needs to be familiar with how privacy policies are coded in machine-readable form such as P3P, but also the software source codes in order to modify, extend and optimize the algorithms. 

\subsubsection{End Users}
Secondly, the end user who will be using the software in the form of a browser plugin that provides advice with respect to the users behavior on the Internet. A key objective for the project is that the agent is to be able to make good decisions and require as little feedback as possible from the user. To the extent interaction is needed, it should be able to clearly state an explanation for its decisions and allow the user to override in a simple manner.


\section{Use Cases}\label{useCase}

The first use case illustrates a research setting where calibration/testing interface allows the user to load in a dataset of P3P policies and test the performance of the underlying model. The last three use cases illustrate the potential application of the system as a browser plugin that runs in the background monitoring the users activities and the web sites he is visiting. As previously stressed, the success of testing according to the first use case determines the extent to which the system described in cases 2-4 is implemented.

\subsection*{Case 1: Research/calibration}
In this case a researcher wants to test the properties of the underlying model. Using the Calibration GUI, he imports 50 P3P policies that are parsed. Further he designates that 40 of these are to be stored immediately in the knowledge base along with a corresponding action for each policy.

The user now specifies the distance metric he wants to apply to each of the different components. Finally, he can either set the (importance) weights assigned to each of the policy components, or he can load the weights from a flat text file. Now that the configuration is complete, the ten policies withheld earlier from the sample can be classified. For each of the ten policies, the user can choose either to accept, or reject, and provide a reason for his rejection before proceeding to the next policy. 

\subsection*{Case 2: End user - local query, recommendation accepted, site rejected}
A user visits a previously unvisited website. The privacy agent tries to retrieve machine-readable privacy information from the site. When the policy is obtained it is parsed and a context object, consisting of the policy, domain, time of visit, and other contextual information, is created. The context object is compared to the local database for similar contexts. Since the user has visited sites with a similar policy previously, the comparison succeeds and the site is blocked based on data from the local database. The user agrees with this decision and navigates away from the site.

\subsection*{Case 3: End user - local query, site approved by recommender, recommendation accepted}
A user visits a previously unvisited website. As before, the system the system fetches the necessary data to do a local query. This query indicates, with sufficient confidence, that the site's policy is acceptable. The user is then allowed to continue browsing with no intervention from the Privacy Agent.

\subsection*{Case 4: End user - global query, recommendation overridden}
As before, but in this case, no sufficiently similar cases are found locally. In this case the system will query the global server for similar users that have visited the same site to base its decision on this. In this case, site is blocked, but the user disagrees. He selects an override feature and gives a reason for why he overrides.

\section{Specific Requirements}\label{specRec}

\subsection{Product perspective}
As described in Section~\ref{reqScope}, as the main goal of the project is to develop a testing framework for the core reasoning system. The secondary goal is to implement a user interface that can work as a stand-alone application to allow for actual user testing.

\subsection{Functional Requirements}
\begin{itemize}

\item The system should be able to parse a P3P file to instantiate the data as a privacy case/event/instance.
\item Based on past history (knowledge base), it should retrieve the cases most similar to the one presented.
\item Given the degree of similarity to past cases and the uniformity of action taken in the past, the system can either 
  \begin{itemize}
  \item Give the user a recommendation a recommendation or 
  \item Pass the recommendation decision on to the community/CF system.
  \end{itemize}
\item If passed on to the CF, the system will query a server for the most similar users and use the data on their decisions in similar cases to make a recommendation (along with local/CBR recommendation) 
\item Update the database with the recommendation.
\item Allow the user to view the explanation for the recommendation
\item Allow the user to overrule a recommendation. 
\item When overruling a recommendation, the user must be allowed to explain why the decision is made, e.g. one time occurrence, permanent rule, etc.	
\item Allow the user, if making a new general rule, to backtrack and alter previous cases
\end{itemize}

\subsection{Non-Functional Requirements}
\begin{itemize}
\item Implementation
  \begin{itemize}
  \item Code is written in Java following Sun Microsystems' conventions\footnote{http://www.oracle.com/technetwork/java/codeconvtoc-136057.html}.
  \item Third party libraries are to be documented with version numbers and to be included in the installation package.
\end{itemize}

\item Maintainability:
  \begin{itemize}
  \item Code repositories and version control: github is used as code repository and for version control.
  \item User documentation is to be produced.
  \item A well documented API is to be designed	
  \item English (US) is to be used as language for naming convention for source code and filenames, and in code comments and documentation.
  \item The code is to be designed in a modular fashion.
  \end{itemize}

\item Performance: 
  \begin{itemize}
  \item For the final end-user product that will run as a browser plug-in, performance will be important, as the program should not be seen as a nuisance in getting work done.
  \end{itemize}

\item Portability: 
  \begin{itemize}
  \item The testing/design system should be portable to any system with a JRE.
  \end{itemize}

\item User interface:
  \begin{itemize}
  \item Two UIs are to be implemented: A command line interface (CLI) as well as a GUI is to be designed using Java/swing.
  \item These interfaces are meant to facilitate testing the model framework.
  \end{itemize}


\end{itemize}


\setcounter{section}{0}
\addtocounter{chapter}{1}
% Activate the following line by filling in the right side. If for example the name of the root file is Main.tex, write
% "...root = Main.tex" if the chapter file is in the same directory, and "...root = ../Main.tex" if the chapter is in a subdirectory.
 
%!TEX root =  

\chapter{Design}
\label{design}

\minitoc

This chapter describes the design phase of the program, where the program architecture is established. Several critical decisions are made in this phase and the design and architecture decisions impacts the way the implementation phase proceeds as it defines how the final software system is decomposed into modules, and how these modules behave and interact with each other. 

In implementing Privacy Advisor, a class structure is built around the CBR agent model discussed in the paper by T{\o}ndel and Nyre. Given the broad structure in the CBR agent model, several details need to be fleshed out, including data structures for storing policies, databases, choosing actual algorithms, a user interface, and so forth. This chapter lays out the broad structure of the Privacy Advisor system. Implementation details are discussed in Chapter~\ref{Implementation}.

%%
%% This chapter should be organized as follows: 
%% - overview
%% - io/gui/cli
%% - cbr engine
%% - data objects storage


\begin{figure}[htbp]
\begin{center}
\includegraphics[width = \textwidth]{DesignReport/uml/flowchart.png}
\caption{Program Flow.}
\label{DesignFlowChart}
\end{center}
\end{figure}

\section{Design Overview}
This section describes the architecture of the local CBR based system. The next section gives an overview over the design of the server component using collaborative filtering and how it interfaces with the local system.  

\begin{figure}[htbp]
\begin{center}
\includegraphics[width = \textwidth]{DesignReport/uml/Case.png}
\caption{System Overview.}
\label{SystemOverview}
\end{center}
\end{figure}

\subsection{Program Flow}
Figures~\ref{DesignFlowChart} and \ref{SystemOverview} seek to provide a broad overview of the structure of the Privacy Advisor system. The program flow is given by Figure~\ref{DesignFlowChart}: a policy object is passed to the CBR system from the user, either through the CLI or GUI. The CBR does a lookup on similar cases in the local database and computes an initial recommendation based on the similar cases retrieved. If the confidence in this recommendation is above a given threshold, it provides the user with the advice and the background for the advice through the user interface. If the confidence is not sufficient, the system will query the community system for an advice, which then is combined with the initial recommendation for a final advice which is presented to the user. The user can then give a feedback on the advice choosing to accept or reject it. The user feedback is then stored back to the database.



\begin{figure}[htbp]
\begin{center}
\includegraphics[width = \textwidth]{DesignReport/uml/gio.png}
\caption{Input/output and user interfaces.}
\label{UserIO}
\end{center}
\end{figure}


\subsection{Top Level Structure}
[Discuss figure \ref{gioFig}.]

\section{User Interfaces and Input/Output}
Privacy Advisor can be run using either a command line interface (CLI) or a graphical user interface (GUI). Both the CLI and the GUI are built on top of a "General Input/Output'' module, GIO. GIO creates the database objects and issues the proper commands to the CBR framework based on user input. The GIO class is shown in Figure~\ref{gioFig}. 

\begin{figure}[htbp]
\begin{center}
\includegraphics[width = \textwidth]{DesignReport/uml/gio.png}
\caption{working of system.}
\label{gioFig}
\end{center}
\end{figure}


\subsubsection{Graphical User Interface} 
%TODO Einar

\begin{figure}[htbp]
\begin{center}
\includegraphics[width = \textwidth]{DesignReport/uml/policyadvisorgui}
\caption{GUI interfaces.}
\label{GUI_interface}
\end{center}
\end{figure}

\subsubsection{Weights and Configuration Files}
As an addition to passing command line arguments, GIO also reads a text based configuration file containing CBR and database settings. The configuration files are detailed in the User Documentation in table~\ref{configTable}. 

\subsubsection{CBR} % TODO Nicholas: Why PDatabase is not connected directly to CBR
Input from the UI is passed on to the CBR framework. CBR in turn references three other key modules, a \emph{reduction} algorithm, a \emph{conclusion} algorithm, and finally, a \emph{learning} algorithm.

The reduction algorithm searches the database to find the most similar cases to the new case presented. The canonical reduction algorithm is k Nearest Neighbors, discussed in section~\ref{kNN}. The conclusion algorithm looks at the set of cases returned by the reduction, and decides on the most appropriate action for the novel case. It also returns a measure of confidence in the conclusion reached.

Finally, a learning algorithm allows for automatically tuning the parameters used for distance calculations. This is discussed further in section~\ref{learnAlgos}.

An overview of the CBR system is given in Figure~\ref{cbr_fig}...
% comment reduction, learning, conclusion, KNN


\begin{figure}[htbp]
\begin{center}
\includegraphics[width = \textwidth]{DesignReport/uml/CBR.png}
\caption{CBR System.}
\label{cbr_fig}
\end{center}
\end{figure}




\section{Data Objects and Storage}

\subsection{P3P Policy Objects}\label{p3pPolObj}
An overview of the Policy Object is given in Figure~\ref{po_fig}...

A P3P Policy Object consists of an \texttt{Action}, a \texttt{Context} and a list of different \texttt{Case} objects. The action object consists of the result from the comparison algorithm, stating if the policy is accepted as a good match, the reasons for this statement and with how much confidence this statement is correct.

The context objects holds most of the objects context, that is what domain the policy belongs to as well as when it was created, last accessed and when it will expire. The list of cases contains one case for each datatype within the policy. A datatype is what kind of information is collected, for example name or date of birth. Each case contains what the purpose for this information is, who are the recipients and the retention for this information. Each datatype has its own case as it simplified the comparison algorithm.

The last thing a policy object contains is a hashmap of the entity data. This is the data that is included at the beginning of every policy document and contains information about the company in question.

\begin{figure}[htbp]
\begin{center}
\includegraphics[width = \textwidth]{DesignReport/uml/po.png}
\caption{Policy Object.}
\label{po_fig}
\end{center}
\end{figure}

\subsection{P3P Policy Database}

An overview of the Policy Database is given in Figure~\ref{pd_fig}...

The local case history, maintained in a local database, is stored via a concrete class implementing 'PolicyDatabase'. This abstract class details the required methods for a local policy database: a singleton constructor for the database object; a call to load the database once constructed, from disk; a method adding a single policy to the database; a method returning a Java iterator over the stored PolicyObjects, and a call to return all policies from a given domain.

In order to ensure consistency, the local policy database enforces singletonness. The database object itself is constructed without the actual history, requiring a seperate parameter-less 'loadDB' call on it to load policies from disk to the class, if necessary.

During the CBR cycle, it becomes necessary to check past cases for relevancy during the 'retrieve' phase. This is accomplished by using the standard java Iterator return by \texttt{getiterator()}.
Finally, the CBR cycle concludes by saving the new case (using 'addpolicy(newpolicy)'), and closing the database using 'closeDB()' (which is when the cases would be saved to disk).

An overview of the Policy Database is given in Figure~\ref{pd_fig}...



\begin{figure}[htbp]
\begin{center}
\includegraphics[width = \textwidth]{DesignReport/uml/pd.png}
\caption{Policy Database.}
\label{pd_fig}
\end{center}
\end{figure}

\subsection{Interfaces}







\subsection{Community Server} %%TODO needs rewrite- focuse here is on design not implementation. mention query should be server-side
The community knowledge repository is implemented using a public CouchDB (no-SQL server), which is accessed using standard Java to JSON java libraries. The client program (end-user java application) communicates with the server at two points- when the application has insufficient knowledge, or confidence in its knowledge, to make a suggestion as to the acceptance of a new P3P policy; and after the user has confirmed or overridden the policy.

In the first instance, the new policy under consideration is converted to JSON using GSON (the Google Json libraries), and transmitted to the database, which parses the new policy and replies with a JSON encoded suggested Action.

In the second instance, the final policy (including the action taken on it) is sent to the CouchDB server, and the server proceeds to store the object.
On the database end, there are two essential interfaces (beyond any standard initialization and shutdown procedures). As seem above, these two interfaces are the suggestion provider, which includes a query to find the most similar policies and actions on them by the community, and a interface to simply save the new policy to the appropriate database.
The database is easily replaceable, requiring only the construction of a new class implementing 'NetworkR', the abstract class detailing the methods called by the PrivacyAdvisor framework. The selection between available 'NetworkR' implementations in made by setting the 'NetworkRType' configuration variable during initialization to the full classname.


\setcounter{section}{0}
\addtocounter{chapter}{1}
% Activate the following line by filling in the right side. If for example the name of the root file is Main.tex, write
% "...root = Main.tex" if the chapter file is in the same directory, and "...root = ../Main.tex" if the chapter is in a subdirectory.
 
%!TEX root =  

\chapter{Implementation}\label{impl}

\minitoc

\section*{Purpose}
This document explains the implementation phase of the project providing a more detailed description of particular key
details that were not decided on in the design phase. This relates in particular to choices regarding the particular CBR algorithms
(k-Nearest Neighbors) and the similarity measures describing how "equal'' two cases are. It also details the datastructures that
are used to represent policies and how comparisons are done on these datastructures.

\section{Algorithms}

\subsection{K-Nearest Neighbors}

\emph{K-Nearest Neighbors} (k-NN) is a \emph{lazy, non-parametric} algorithm for classifying objects based on the classification of the nearest examples in a given feature space. k-NN is one of the simplest machine learning algorithms as it decides the classification based on a majority vote of, that is, an object is classified according to the most common classification of its $k$ nearest neighbors. The most critical component for the success of the k-NN algorithms is the definition of distance. This is discussed in Section~\ref{SimilarityMeasures}. 

For testing purposes, we have implemented a very simple kNN that sorts the example set (knowledge base) by distance from the object to be classified and returns the k nearest objects. This is obviously not an optimal approach being $O(n lg(n))$ where $n$ denotes the size of the knowledge base. This is not problematic for a small scale application such as ours where the knowledge contains less than 200 objects. If the system is to be scaled up, it would require a new kNN implementation, of which there are several available.

\subsubsection{Learning algorithms}
The learning algorithm updates the weighs that each property field is assigned in computing distances. The learning algorithm goes through the policies in the database and computes updated weights. The implemented learning algorithm is a simple one that for each weight goes through every policy and checks if it have been accepted or not. Then it returns the number of accepts divided by the number of policies for each weight.

By updating the weights like this we make sure that the system to some degree learns what the user wants. Thus, hopefully the system would be able to make a different conclusion next time if the user didn't agree on the conclusion the system came up with this time.

\subsection{Distance Measures}\label{SimilarityMeasures}

\subsubsection{Definition}

Mathematically, a \emph{metric} or \emph{distance function} is a function defines the \emph{distance} between two objects in a set are, that is, it defines a notion of how far apart two objects are. In a purely mathematical sense, a distance function defined over a set $X$, $X\times X\longrightarrow \mathbb{R}$ that is required to obey the conditions of non-negativity, symmetry, sub-additivity (the triangle inequality) and identity of indiscernibles. 

Some examples of commonly used metrics are the Euclidian, Mahalanobis, and the Manhattan distance measures. These along with a few others are defined in the next section. These metrics have all in common that $\mathbb{R}^n\times \mathbb{R}^n\longrightarrow \mathbb{R}$, which in the case of comparing privacy policies and corresponding context information, is problematic as these, in their raw form contain large amounts of textual data. Two remedies could be proposed for this situation:

\begin{enumerate}
\item Provide a function to map privacy objects (P3P policies and context info) to real vectors.
\item Define a new metric that operates directly on privacy objects.
\end{enumerate}
 
%% NBNBNBNB!!!!
%% Formulae must be inserted!!
%%

\subsubsection{Common Metrics}

\begin{itemize}
\item The\emph{Manhattan distance} function computes the distance that would be traveled to get from one data point to the other if a grid-like path is followed. It can be written as  where d is the dimensionality of the data objects. 
\item The \emph{Hamming distance} is defined as number of positions in which a source and target vector disagrees. If data are bitstrings then Hamming distance can be calculated as.
 
\item \emph{Levenshtein distance} is based on Hamming distance but adds also operands as insertion, deletion and substitution. And can be written... 
 
\item \emph{Ontology distances} are more based on compute semantic similarity of objects rather than their textual representation. For example distance between apple and orange is less then between apple and house. To calculate this distance you need some sort of logical tool like ontological tree. Where every leaf has a logical ancestor for example ancestor for apple will be fruit.
\end{itemize}
 
\subsubsection{Customer Advise}

In the paper "Towards a Similarity Metric for Comparing Machine Readable Privacy Policies", the some of the problems of defining a similarity metric for privacy policies is discussed. A key topic is how the calculation of similarity between online 3P3 policies can be subdivided in two parts, \emph{local similarity} and \emph{global similarity}. This way known metrics such Levenshtein distance  can be applied to local distance. And for global similarity we can calculate a simple or weighted average of local distances, where the second one allows for amplifying the importance of particular attributes.

Another important topic is how system designers can apply domain knowledge to improve distance calculation. For example, for the recipient field identifying who data is shared with, there are certain values revealing that significantly more private information is exposed than others. Having private information retained by the website (recipient = "ours") is in a sense less critical than it being given away to third parties for commercial purposes (recipient = "other) or being public (recipient= "public"). So distance between unrelated or public is less then between ours and unrelated.

\subsection{Implementation}

\subsubsection{Bitmap Representation}
A bitmap (or bit string) is a way of representing a set of objects. It simply translates values from a set to a vector of fixed length, where each value has a specific place. For example over language $L=\{a,b,c,d\}$ for a sets $\{a,b,c\}$ a bit-map will look like $[1 1 1 0]$ where 1st integer represents $a$ and last integer represents value $d$. This way it doesn�t matter what order the values are arranged and how many values are so set $\{d,b\}$ over same langue $L$ will be $[0 1 0 1]$ where 1st value still representing value $a$, or more correctly said, absence of value $a$. Calculating intersections or unions over bitmaps $A$ and $B$ uses bitwise Boolean operators. Where union can be easily written as $(A_i \vee B_i)$ where i is the position in the vector, and the intersection $(A_i \wedge B_i)$.

Weighed bitmap is when each of the values is multiplied by its corresponding weight. Let us say on the language $L$ the weights are $a=1 b=2 c=4 d=3$, since bit-map over set $\{a,b,c\}$ is $[1 1 1 0]$ the weighed bit-map will be $[1*w_a 1*_b 1*_c 0*w_d]=[1 2 4 0]$. And for set $\{d,b\}$ it will be $[0 2 0 3]$.

\subsubsection{Privacy Policy Representation}

This section describes the data structures used to represent policies. On the top level, \texttt{policyObject}s contain P3P policy and context information (URL + time etc.). A P3P policy can have a varying number of \emph{statements}, some of them greatly differ from each other, and others are very similar. Each statement is build of four fields: data-collected, purpose, recipient and retention. Purpose, recipient and retention can contain one or many given values, the combination of witch describe how the given data-collected may be used in the future. Data-collected is divided in four major fields: dynamic, user, third-party and business. Many different data-types can be collected in one statement.

For every data object that being described in a statement, we create a \texttt{Case}. A \texttt{Case} in a \texttt{policyObject} corresponds to a statement in P3P policy. It contains the purpose, recipient and retention, but a Case can only have one unique data type; this results that one statement can be translated to many cases. \texttt{policyObject} has a list of all its \texttt{Case}s and additional information that we find useful like time of the visit, location of the domain, and action decided upon by the CBR system or the user. Based on SINTEFs proposal the two levels of similarity, local and global are accounted for in implementation. %Though for easy switching between algorithms a distance Metric class have to contain bought. !!! ????

\subsubsection{Bitmap Distance}

The distance function is to be a highly modular component of the CBR system. The default distance function implemented uses the bitmap data structure mentioned above, and is henceforth referred to as "Bitmap Distance". For the local similarity Bitmap Distance generates bitmaps for fields, retention, purpose, and recipient. This eliminates the problem that these fields can have a differing number of attributes.  Prior to computing the Hamming distance the bit-map is transformed into weighed bit-map by multiplying every attribute by its weight. %The bitmap representation would be bad one for data-types. We implemented a sort of string-comparison/ontological-distance that will be explained later.  !!! ???????


%%% BEGIN NON-EDITED TEXT

% The Hamming distance for weighed bitmap is easy to implement like we said before it can be represented by Boolean expressions. For practical reasons we first apply weights to bitmap and then calculating distance, this way you can say we take difference between cases because for $d_(h_i ) (s,t)$ to be equals $1$ in $formula$. One of $s_i$  or $t_i$ must be $0$ and other one $1$. If we apply weights to $d_(h_i ) (s,t)$ it will be $1*weight$. So if $d_(h_i ) (s,t)=1*weight$ that absolute value of difference between $s_i$  or $t_i$ will be $1*weight$ as well. Now we just showed that Hamming distance on binary vectors (also called bitmap) equals Manhattan distance on same sets. Hamming distance with creation of bitmaps is a fast algorithm with a linear run-time. The easy ways predicting the result and fast run-time is the reasons we choose this for implementation. With the help of the way data type is structured in P3P policies we can calculate a data-type distance by just parsing data-type-string. Every data-type string has to start with one of the four previously mentioned fields, then after a dot a sub-field is followed after other dot a sub-sub-field. Let us say we have your simplified ontological tree just within the syntax of data-type-string with an "invisible" root node over those four fields. This way when we have to data-type-strings for comparison we can easily count number of ancestors from string A to string B. For instance String a is �user.home-info.postal� and B �user.bdate�. Number of nodes we need to take from stringA to stringB is 3. 

The weakness is that without weights it�s more string comparison then ontological tree.

We decided that weighting data-types values was outside of scope of this project. We simply gave static values to 1st nodes from the root(the four main fields) based on our knowledge and interpretation of private policy.

It is possible to create a weight system so this algorithm will work like a full ontological tree. For example if we add a value for each possible node value/weight so we can simply take difference between StringA�s 1st 2nd and 3rd field and respectfully StringB�s fields. If the field 1 in A same as B�s then distance is of course 0. But this way we can say that some of the fields or sub-fields based on the same depth in your tree can have same values and distance between them will be 0. For example if we want to say that bdate and name are equally important for user we can give then same value making difference between them 0, but if we want to increase distance between name and home-info we have give them values that have a greater difference like 1 for name and 10 for home-info, creating distance 9 between them.

Weights are the key to learning and adjusting this algorithm. They are greatly used in our implementation of Hamming distance and can give great results in data-type analysis.

The global similarity part was not as easy as creating a bitmap. The number of cases a policy can is undefined, and the similarity of those cases can differ a great deal. Your solution was sum of minimal distance of each Case to a case in the other policy. For instance a caseA from policyA have distance 2, 3, 1 to cases in policyB that mean the distance caseA will have distance 1 to policy. 

% Pseudo code will look like:

% \begin{verbatim}
Sum=0
For caseA in PolicyA.Cases
	Min=infinity
	For caseB in PolicyB.Cases
		Dist=Compare caseA to caseB
		If dist<min then
			Min=dist
		End
	End
	Sum=sum+min
End
% Return sum
% 
% \end{verbatim}
% 

By choosing minimal distance between cases we guaranty that if two cases are identical the distance between them will be 0. But it also creates a problem, consider to policies policyA with {caseA1 caseA2 caseA3} and policyB with {caseB1 caseB2} where every case in A has minimal distance with caseB1. This way the properties of caseB2 will go �unnoticed� in the sum of distances between cases. We solved this problem by running algorithm twice but changing places of policy A and B the 2nd time and simply summing results.

The weakness of computing this way is that some of the distances will be counted twice, but user-privacy-safety wise is better to count some cases of minimal distance twice then leave a �dangerous� or most distant case out.

There is some variations in this method. You can use maximum/minimum distance between cases or average of the sum. We choose minimum because this way we always try to find a best match between cases and policies. With an algorithm that will minimize error from computing twice, from a to b and b to a, minimum distance with give the best results. But in total picture if you use same algorithm that considers every case for every policy in your database the results will be proportional.

%%% END NON-EDITED TEXT



\section{Data Storage}

\subsection{Flat File Storage}

\subsection{Databases}

\section{User Interface}

\subsection{Model View Controller Architecture}
[Describe theoretically]

\subsection{Privacy Advisor Interface}
[Describe choices that have been made and reasons for differences from standard architecture]

\subsection{CLI}

\subsection{GUI}




\part{Testing, Evaluation and Summary}\label{p3}

\setcounter{section}{0}
\addtocounter{chapter}{1}
% Activate the following line by filling in the right side. If for example the name of the root file is Main.tex, write
% "...root = Main.tex" if the chapter file is in the same directory, and "...root = ../Main.tex" if the chapter is in a subdirectory.
 
%!TEX root =  

\chapter{Test cases}
	\setcounter{tocdepth}{1}

	\section{Test cases}
		\vspace{8 mm}		

		\begin{center}
			\begin{tabular}{ |  p{3.5cm} | p{10cm} | }
				\hline
				Item & Description \\ [5pt] \hline \hline
				Name & Command line interface (CLI) functionality \\  [5pt] \hline
				Test identifier & UNIT-01 \\  [5pt] \hline
				Person responsible & Henrik Knutsen \\  [5pt] \hline
				Feature(s) to be tested & All possible commands when running the program with the CLI. That input with incompatible commands does not run. \\  [5pt] \hline
				Pre-conditions & Code for input handling for all possible commands. Error handling for invalid inputs. \\  [5pt] \hline
				Execution steps & 1. Test all the commands one by one. \newline 2. Test the combinations of invalid inputs. \\  [5pt] \hline
				Expected result & 1. The program runs using the input variables. \newline 2. Error warning: Invalid arguments. Abort startup. \\  [5pt] \hline
			\end{tabular}
		\end{center}

		\begin{center}
			\begin{tabular}{ |  p{3.5cm} | p{10cm} | }
				\hline
				Item & Description \\ [5pt] \hline \hline
				Name & P3P parser \\  [5pt] \hline
				Test identifier & UNIT-02 \\  [5pt] \hline
				Person responsible & Henrik Knutsen \\  [5pt] \hline
				Feature(s) to be tested & That the P3P parser correctly parses all the required fields and their values. \\  [5pt] \hline
				Pre-conditions & The P3P parser must be implemented. Need to have P3P policies with a wide range of cases. \\  [5pt] \hline
				Execution steps & 1. Manually go through a P3P XML and obtain all the required fields and the values. \newline 2. Run the same P3P XML in the P3P parser and print the parsed elements and their values to console.
					\newline 3. Compare the results from the two parsing methods. \\  [5pt] \hline
				Expected result & The two parsing methods give identical output. They must both have the same fields, each containing the same values \\  [5pt] \hline
			\end{tabular}
		\end{center}
		
		\begin{center}
			\begin{tabular}{ |  p{3.5cm} | p{10cm} | }
				\hline
				Item & Description \\ [5pt] \hline \hline
				Name & Local database \\  [5pt] \hline
				Test identifier & UNIT-03 \\  [5pt] \hline
				Person responsible & Henrik Knutsen \\  [5pt] \hline
				Feature(s) to be tested & Writing to and reading from the local database. That the serialization of the database is working. \\  [5pt] \hline
				Pre-conditions & Code for writing to and reading from the database file must be implemented. Need to have two different P3P policies. \\  [5pt] \hline
				Execution steps & 1. Write policy A to the local database. \newline 2. Write policy B to the local database. \newline 3. Read policy A from the local database. \newline 4. Read policy B from the local database. \\  [5pt] \hline
				Expected result & The written policy A and the read policy A must be identical. \newline The written policy B and the read policy B must be identical. \\  [5pt] \hline
			\end{tabular}
		\end{center}

		\begin{center}
			\begin{tabular}{ |  p{3.5cm} | p{10cm} | }
				\hline
				Item & Description \\ [5pt] \hline \hline
				Name & Graphical user interface (GUI) functionality \\  [5pt] \hline
				Test identifier & UNIT-04 \\  [5pt] \hline
				Person responsible & Henrik Knutsen \\  [5pt] \hline
				Feature(s) to be tested & That all the interactable elements, buttons, lists etc., is working as intended. \\  [5pt] \hline
				Pre-conditions & GUI with all the necessary listeners must be implemented. Code for running the program with the GUI must be implemented. \\  [5pt] \hline
				Execution steps & 1. Run the program using the GUI. \newline 2. Test all the interactable elements. \\  [5pt] \hline
				Expected result & All the interactable elements is triggering the right methods when used. \\  [5pt] \hline
			\end{tabular}
		\end{center}

		\begin{center}
			\begin{tabular}{ |  p{3.5cm} | p{10cm} | }
				\hline
				Item & Description \\ [5pt] \hline \hline
				Name & Algorithm classification \\  [5pt] \hline
				Test identifier & UNIT-05 \\  [5pt] \hline
				Person responsible & Henrik Knutsen \\  [5pt] \hline
				Feature(s) to be tested & That the k-nearest neighbor algorithm bases its decision on the k most similar policies \\  [5pt] \hline
				Pre-conditions & Code for reading from the weights file must be implemented. A working k-nearest neighbor algorithm that uses the weights must be implemented. Need one policy to test on, and a set of policies to be used as history. \\  [5pt] \hline
				Execution steps & 1. Load a set of policies into the history. \newline 2. Run the k-nn algorithm on the policy to be classified and the history. \newline 3. Manually go through the policies and verify the output of the algorithm. \\  [5pt] \hline
				Expected result & The algorithm finds the most similar policy. \\  [5pt] \hline
			\end{tabular}
		\end{center}

		\begin{center}
			\begin{tabular}{ |  p{3.5cm} | p{10cm} | }
				\hline
				Item & Description \\ [5pt] \hline \hline
				Name & Algorithm learning \\  [5pt] \hline
				Test identifier & UNIT-06 \\  [5pt] \hline
				Person responsible & Henrik Knutsen \\  [5pt] \hline
				Feature(s) to be tested & That the weights file is updated when a new policy is added to history. \\  [5pt] \hline
				Pre-conditions & Code for reading from and writing to the weights file must be implemented. Algorithms for classification and learning must be implemented. \\  [5pt] \hline
				Execution steps & 1. Get the contents of the weights file. \newline 2. Load a set of policies into the history. \newline 3. Run the classification algorithm on the single policy and the history. \newline 4. Choose to store the new policy, the context and the action.
					\newline 5. Get the contents of the weights file. \newline 6. Compare the contents of the weights files obtained in steps 1. and 5. \\  [5pt] \hline
				Expected result & The two weights files obtained in steps 1. and 5. are different \\  [5pt] \hline
			\end{tabular}
		\end{center}

		\begin{center}
			\begin{tabular}{ |  p{3.5cm} | p{10cm} | }
				\hline
				Item & Description \\ [5pt] \hline \hline
				Name & Packet passing through network to community database \\  [5pt] \hline
				Test identifier & UNIT-07 \\  [5pt] \hline
				Person responsible & Henrik Knutsen \\  [5pt] \hline
				Feature(s) to be tested & That packets can be sent between the client program and the community database.  \\  [5pt] \hline
				Pre-conditions & A running local client. A (virtual) server. Code for sending and receiving packets must be implemented. \\  [5pt] \hline
				Execution steps & 1. Start the program locally. \newline 2. Start the (virtual) server. \newline 3. Send packet A from the local client. \newline 4. Receive packet A at the (virtual) server. \newline 5. Send packet B from the (virtual) server.
					\newline 6. Receive packet B at the local client. \\  [5pt] \hline
				Expected result & The received packet A is identical to the sent packet A. The received packet B is identical to the sent packet B. \\  [5pt] \hline
			\end{tabular}
		\end{center}


\setcounter{section}{0}
\addtocounter{chapter}{1}
% Activate the following line by filling in the right side. If for example the name of the root file is Main.tex, write
% "...root = Main.tex" if the chapter file is in the same directory, and "...root = ../Main.tex" if the chapter is in a subdirectory.
 
%!TEX root =  

\chapter{Project Evaluation}\label{eval}

\minitoc

\section*{Project Evaluation}
The final phase of the project consists of evaluating the actual outcome of
the project work and the processes that have led to the outcome. It
seeks to answer questions such as:

\begin{itemize}
\item Has the project acheived its major objectives?
\item In what areas could a better result have been achieved?
\item What could be done differently to acheive a better result?
\item How well did planned resource use reflect actual resource use?
\end{itemize}

\section{Software Evaluation}

\section{Organization and Group Dynamics}

\section{Tools}

\subsection{Programming and Implementation}

\subsection{Reporting and Organizational Tools}

\section{Resource Use}



\setcounter{section}{0}
\addtocounter{chapter}{1}
% Activate the following line by filling in the right side. If for example the name of the root file is Main.tex, write
% "...root = Main.tex" if the chapter file is in the same directory, and "...root = ../Main.tex" if the chapter is in a subdirectory.
 
%!TEX root =  

\chapter{Conclusion}\label{conclusion}

\minitoc

This project has designed and implemented a privacy agent based on the
research ideas set forth by SINTEF. The Privacy Advisor software as it
is today, is a case based reasoning engine that can provide advice
with regard to P3P privacy policies given knowledge of previous user
actions. Both the idea and the software are still work in progress,
which is reflected by the system's user interface which are oriented
towards testing. It is built in a modular fashion, so that the
knowledge base, the retrieval algorithm and the similarity metrics, which are likely to be the
parts of the system that require most of the tweaking, can
be substituted without affecting the remainder of the software.

\section{Future Development}

To conclude this report, we look at the some important challenges that
must be addressed in order to further the Privacy Advisor system to a
end-user product.

\subsection{Testing}
For the Privacy Advisor system to be truly valuable, it has to be made
available to end users; the most likely realization being in the
form of a web-browser plugin. For this to occur, much testing is still
required. This testing is a more involved type of testing than the standard
unit testing that the Privacy Advisor system has been through at present.

For the system to actually be successful it has to provide accurate predictions
of users' Internet decisions so that it can provide good privacy protection
without being invasive in the users' day-to-day Internet activities. This type
of testing will require putting together a group of test users and monitoring their
activities over a \emph{prolonged period of time}\footnote{The time aspect is stressed here
as the system will need time to actually learn the user's preferences.}. 

\subsection{Collaborative Filtering} %% TODO: Nicholas: verify
One of the key improvements to the current Privacy Advisor system is the
collaborative filtering/community portion of the system. While the interface
for a networking system is well integrated in the CBR system, the current 
network resource has limited features. Furthermore, this component of the system
will also require a prolonged period of testing, as a substantial amount of data
needs to be gathered to give proper advice as this operates on a per-user-per-site
basis\footnote{SINTEFs current proposal for a collaborative filtering system is
to find the most similar user(s) and base the recommendation on his actions on the
same/similar sites.}.

\subsection{Distance Metrics}


\subsection{The Current State of Internet Privacy Policies}


% Appendices
\part{Appendices}
\appendix
\Alph{chapter}
\setcounter{chapter}{1}
% Activate the following line by filling in the right side. If for example the name of the root file is Main.tex, write
% "...root = Main.tex" if the chapter file is in the same directory, and "...root = ../Main.tex" if the chapter is in a subdirectory.
 
%!TEX root =  

\chapter{Documentation}\label{doc}

\minitoc

This section provides documentation for the Privacy Advisor system. It gives an overview over the available documentation of source code, instructions for how to compile and install the system, and how it is used via. its GUI and command line interfaces.

\section{Source Code Documentation} 

The source code is documented using JavaDoc which is 	a tool that generates documentation in HTML format based on source code comments in Java, and is a standard part of the Java SDK. The JavaDoc for the Privacy Advisor system follows Sun Microsystems' style guide for writing JavaDoc comments\footnote{See: http://www.oracle.com/technetwork/java/javase/documentation/index-137868.html}. Source code documentation plays an important role in this project, as it is an early software prototype to be used in research which means that the code is then likely to be modified. The aim of the source code documentation is to supplement UML design documents to facilitate future development.

\section{Installation}

\subsection{Local Installation} \label{LocalInst}
To make the program run outside of the Eclipse environment it have to be exported as a runnable jar file. This can be done by right clicking the project in eclipse, choose \texttt{Export}, type in \texttt{jar} in the new window that opens up and choose \texttt{Runnable Jar File}, before clicking \texttt{next}. These steps are showed in the two figures \ref{exportFirstStep} and \ref{exportSecondStep}.

\begin{centering}
  \begin{figure}
    \includegraphics{Documentation/export.png}
    \caption{First step: right click and choose "export".}
    \label{exportFirstStep}
  \end{figure}
\end{centering}

\begin{centering}
  \begin{figure}
    \includegraphics{Documentation/export_jar.png}
    \caption{Second step: choose runnable jar.}
    \label{exportSecondStep}
  \end{figure}
\end{centering}

Then we have to choose which classfile we want to be the main class; in the \texttt{Launch configuration} dropdown menu we can choose between the two classes \texttt{PrivacyAdviser} and \texttt{PrivacyAdvisorGUI}. The first one is choosen if we want to run the program by using the command line, and the second one if we want to use the GUI. Then choose the export destination, and choose \texttt{Package requires libraries into generated JAR} as the library handling. Pressing finish now should start exporting the program. These steps are shown in figure \ref{exportLastStep}.

\begin{centering}
  \begin{figure}
    \includegraphics{Documentation/export_last.png}
    \caption{Third step: choose launch config, destination and library handling.}
    \label{exportLastStep}
  \end{figure}
\end{centering}

Before running the program we have to put the PrivacyAdviser.cfg file in the same folder as the exported jar file. And if the database file and the weights file isn't at the same folder, their location have to be specified in the PrivacyAdviser.cfg file. To run the program, open the command prompt and locate to the folder where the jar file is stored. Then write the command \texttt{java -jar filename.jar}, and now the program should start (even if it's the cli or gui version). This is show in figure \ref{runProgram}.


\begin{centering}
  \begin{figure}
    \includegraphics{Documentation/run.png}
    \caption{Run the program.}
    \label{runProgram}
  \end{figure}
\end{centering}

\subsection{Server Installation}

\section{User Interfaces}
The Privacy Adviser consist, as mentioned in section \ref{LocalInst}, of two different user interfaces. The advantage of the GUI is that it's easier to see the contents in the database before and after the program runs. The advantage of using the command line interface is that we can work much faster than by using the graphical user interface. But the best part of using the command line interface is that we can add options when calling (starting) the program, this is explained in detail in section \ref{cliExplained}.

\subsection{Graphical User Interface}

\subsection{Command Line Interface} \label{cliExplained}
By running the program from the command line we can add options (parameters) directly when starting up the program. If we run the program normally, as described in section \ref{LocalInst}, the options in the PrivacyAdviser.cfg will be used. But when we add options in the command line when starting the program, the added options will override the ones in the config file.
As an example we can take the location of the database. By default the options in PrivacyAdviser.cfg is set to replace the old database with the new database when the program exits. There are two ways to change this. If we never or rarly want the new database to replace the old one we can just change the \texttt{outDBLoc} option to be different from the \texttt{inDBLoc} option in the config file. But if we want to change the location/name just once, we can add an option in the command line like this:
\texttt{java -jar privacyAdvisor.jar -outDBLoc newDatabase.db}
We can add more than one option in this manner, and for the options that we don't specify, the options in the config file will be used. For a complete list of the options see table \ref{configTable}.

\subsubsection{Configuration Files}

Table~\ref{configTable} gives an overview over the configuration file parameters.

\begin{center}
  \begin{table}[h!]
    \label{configTable}
    \begin{tabular} { | l | l | p{7cm} | }
      \hline
      \textbf{Item} & \textbf{Datatype} & \textbf{Description} \\ \hline
      loglocation & string/filepath  & where the log is written to. can't be changed once the UI is  called. \\ \hline
      loglevel & string/logging level	& what level to log at. \\ \hline
      inDBLoc & string/filepath	& where to read the past history from. \\ \hline
      outDBLoc & string/filepath & where to write DB- defaults to where it reads from. \\ \hline
      inWeightsLoc& string/filepath & where to read the weights config file from. \\ \hline
      outWeightsLoc	& string/filepath & where to write DB- defaults to where it reads from. \\ \hline
      newDB & string/boolean & are we overwriting/ignoring an old database. \\ \hline
      p3pLocation & string/filepath & a p3p to be added to the history. \\ \hline
      p3pDirLocation	& string/FOLDERpath& a folder of p3ps to be added to the history. \\ \hline
      blanketAccept & string/boolean & accept the advisers recommendation. \\ \hline
      newPolicyLoc & string/filepath	& the new policy to be parsed. \\ \hline
      userInit & string/boolean	& true if some initialization occurs via the user interface. \\ \hline
      userResponse & string/action	& the response to the suggestion, if know beforehand. \\ \hline
      cbrV & string/CBR & parses for algorithms, etc to use. See CBR:parse(String). \\ \hline
      userIO & string/UIO	& the user interface to use. see Gio:selectUI. \\ \hline
      policyDB & string/policyDB & select the database type. see Gio:selectPDB . \\ \hline
      genConfig	& string/filepath & load an alternate configuration file. \\ \hline
      networkRType & string/classname & the name of a networkR class. \\ \hline
      networkROptions & string/commasepoptions	& the options necessary for the above networkR class. \\ \hline
      confidenceLevel & string/double & the confidence level at which the algorithm trusts itself; if below this, it uses the server's suggestion. \\ \hline
      useNet & string/boolean & whether to activate network functionality. \\ \hline
      \hline
    \end{tabular}
    \caption{Configuration file parameters.}
  \end{table}
\end{center}

\subsubsection{Building a Database}

\textbf{CLI}: To build a new database from a directory \texttt{P3PDir} holding P3P files, Privacy
Advisor can be called from the command line in the following fashion:
\texttt{PrivacyAdvisor -newDB  true -outDBLoc new.db
  -p3pDirLocation P3PDir}.

\textbf{GUI}: To build a new database in the similar fashion using the
graphical user interface, the configuration window can be set up
similarly to that illustrated in figure[XXXX].



\subsubsection{Loading and Viewing a Database}



\subsubsection{Parsing a P3P Policy}


\chapter{Risk Review}\label{riskAppendix}

\minitoc


This appendix contains a listing of the a few risk factors identified during the project planning phase. The risks are categorized in three broader categories. Technical risk factors are problems occurring in the implementation of the software, i.e. important parts of the problem that are not functioning as indended. Communication issues pertain to problems communicating either within the project team or between the team and the customer. The last type of risks are those that have to do with planning and decisions made early on in the project.

The risk factors are quantifed as $probability \times consequence$ on a scale from $1$( lowest) to $25$ (highest).

%%%%%%%%%%%%%%%%%%%%%

\begin{table}[h!]

\begin{center}
\begin{tabularx}{\textwidth}{| X | X |}
\hline
\textbf{Risk item} & 1 \\
\hline
\textbf{Activity} & Implementation.  \\
\hline
\textbf{Risk Factor} & Problems with retrieving data from P3P policies. \\
\hline
\textbf{Probability} & 3 \\
\hline
\textbf{Consequence} & 5 \\
\hline
\textbf{Risk} & 15 \\
\hline
\textbf{Action taken} & Research into P3P, and cooperate with
customer. Focus on modular design. \\
\hline
\textbf{Deadline} & Implementation deadline. \\
\hline
\textbf{Responsible} & Einar Afiouni. \\
\hline
\end{tabularx}
\caption{Problems with policy retrieving}
\end{center}
\label{risk_3}
\end{table}

\begin{table}[h!]

\begin{center}
\begin{tabularx}{\textwidth}{| X | X |}
\hline
\textbf{Risk item} & 2 \\
\hline
\textbf{Activity} & Implementation. \\
\hline
\textbf{Risk Factor} & Problems with storing and/or retrieving data. \\
\hline
\textbf{Probability} & 2 \\
\hline
\textbf{Consequence} & 3 \\
\hline
\textbf{Risk} & 6 \\
\hline
\textbf{Action taken} & Look into several alternative knowledge base alternatives.  \\
\hline
\textbf{Deadline} & End of design phase. \\
\hline
\textbf{Responsible} & Amanpreet Kaur. \\
\hline
\end{tabularx}
\caption{Problems with external storage}
\end{center}
\label{risk_4}
\end{table}

\begin{table}[h!]

\begin{center}
\begin{tabularx}{\textwidth}{| X | X |}
\hline
\textbf{Risk item} & 3 \\
\hline
\textbf{Activity} & Implementation.  \\
\hline
\textbf{Risk Factor} & Obtaining remote server space. \\
\hline
\textbf{Probability} & 1 \\
\hline
\textbf{Consequence} & 3 \\
\hline
\textbf{Risk} & 3 \\
\hline
\textbf{Action taken} & Ask IDI for virtual server. \\
\hline
\textbf{Deadline} & End of design phase. \\
\hline
\textbf{Responsible} & Nicholas. \\
\hline
\end{tabularx}
\caption{Remote server problems}
\end{center}
\label{risk_5}
\end{table}



\begin{table}[h!]
\begin{center}
\begin{tabularx}{\textwidth}{| X | X |}
\hline
\textbf{Risk item} & 4 \\
\hline
\textbf{Activity} & Implementation, testing.  \\
\hline
\textbf{Risk Factor} & 3rd party code may be harmful or not work as intended. \\
\hline
\textbf{Probability} & 1 \\
\hline
\textbf{Consequence} & 3 \\
\hline
\textbf{Risk} & 3 \\
\hline
\textbf{Action taken} & Proper selection criteria and testing routines for selecting 3rd party code. \\
\hline
\textbf{Deadline} & End of design phase. \\
\hline
\textbf{Responsible} & The responsible for the functionality using 3rd party libraries. \\
\hline
\end{tabularx}
\caption{3rd party library problems.}
\end{center}
\label{risk_7}
\end{table}


\begin{table}[h!]

\begin{center}
\begin{tabularx}{\textwidth}{| X | X |}
\hline
\textbf{Risk item} & 5 \\
\hline
\textbf{Activity} & All  \\
\hline
\textbf{Risk Factor} & Misunderstandings between customer and the group. \\
\hline
\textbf{Probability} & 3 \\
\hline
\textbf{Consequence} & 3 \\
\hline
\textbf{Risk} & 9 \\
\hline
\textbf{Action taken} & Proper reporting and documentation. \\
\hline
\textbf{Deadline} & N/A. \\
\hline
\textbf{Responsible} & Ulf Nore, Customer. \\
\hline
\end{tabularx}
\caption{Misunderstandings between customer and project team.}
\end{center}
\label{risk_8}
\end{table}



\begin{table}[h!]
\begin{center}
\begin{tabularx}{\textwidth}{| X | X |}
\hline
\textbf{Risk item} & 6 \\
\hline
\textbf{Activity} & All \\
\hline
\textbf{Risk Factor} & Disagreements between team members on sharing of work, how to approach problems etc. \\
\hline
\textbf{Probability} & 1 \\
\hline
\textbf{Consequence} & 4 \\
\hline
\textbf{Risk} & 4 \\
\hline
\textbf{Action taken} & Attend seminars on group dynamics. Be quite specific on what's expected from the start. Assign leadership roles responsible for resolving conflicts. \\
\hline
\textbf{Deadline} & N/A \\
\hline
\textbf{Responsible} & All project members, project manager in particular. \\
\hline
\end{tabularx}
\caption{Disagreements within the project team.}
\end{center}
\label{risk_9}
\end{table}




\begin{table}[h!]
\begin{center}
\begin{tabularx}{\textwidth}{| X | X |}
\hline
\textbf{Risk item} & 7 \\
\hline
\textbf{Activity} & All. \\
\hline
\textbf{Risk Factor} & The requirements might change. \\
\hline
\textbf{Probability} & 2 \\
\hline
\textbf{Consequence} & 4 \\
\hline
\textbf{Risk} & 8 \\
\hline
\textbf{Action taken} & Clarify the requirements and agree on deadlines for any changes that could happen.\\
\hline
\textbf{Deadline} & By acceptance of requirements specification. \\
\hline
\textbf{Responsible} & Ulf Nore, Customer. \\
\hline
\end{tabularx}
\caption{Changes in requirements.}
\end{center}
\label{risk_1}
\end{table}


\begin{table}[h!]
\begin{center}
\begin{tabularx}{\textwidth}{| X | X |}
\hline
\textbf{Risk item} & 8 \\
\hline
\textbf{Activity} & Design, implementation.  \\
\hline
\textbf{Risk Factor} & The implemented algorithms may not work as intended or not well suited for this project.\\
\hline
\textbf{Probability} & 3 \\
\hline
\textbf{Consequence} & 5 \\
\hline
\textbf{Risk} & 15 \\
\hline
\textbf{Action taken} & Research on similar algorithms and
projects. Focus on modularity. \\
\hline
\textbf{Deadline} & N/A \\
\hline
\textbf{Responsible} & Dimitry Kongevold. \\
\hline
\end{tabularx}
\caption{Poorly chosen algorithms}
\end{center}
\label{risk_2}
\end{table}




\begin{table}[h!]
\begin{center}
\begin{tabularx}{\textwidth}{| X | X |}
\hline
\textbf{Risk item} & 9 \\
\hline
\textbf{Activity} & All. \\
\hline
\textbf{Risk Factor} & Unable to work due to sickness. \\
\hline
\textbf{Probability} & 2 \\
\hline
\textbf{Consequence} & 4 \\
\hline
\textbf{Risk} & 8 \\
\hline
\textbf{Action taken} & Plan with some degree of slack. \newline Properly document work so that other members may take over. \\
\hline
\textbf{Deadline} & N/A \\
\hline
\textbf{Responsible} & Everyone in the group.\\
\hline
\end{tabularx}
\caption{Risk factor: Sickness.}
\end{center}
\label{risk_6}
\end{table}

\documentclass[12pt, fullpage, oneside]{report}
\usepackage{longtable}
\begin{document}

\title{Test Plan}

	\subsection*{Test cases}
	\section{Test cases}


\begin{center}
	\begin{longtable}{ | p{4cm} | p{10cm} | }
	\caption{UNIT-01}\\ \hline
	\textbf{Item} & \textbf{Description} \\ [3pt] \hline \hline
	\endfirsthead
	\multicolumn{2}{c}%
	{\tablename\ \thetable\ -- \textit{Continued from previous page}} \\ \hline
	\textbf{Item} & \textbf{Description}\\ \hline
	\endhead \hline \hline 
	\multicolumn{2}{r}{\textit{Continued on next page}} \\
	\endfoot \hline
	\endlastfoot
				Name & Command line interface (CLI) functionality \\  [3pt] \hline
				Test identifier & UNIT-01 \\  [3pt] \hline
				Person responsible & Henrik Knutsen \\  [3pt] \hline
				Feature(s) to be tested & That all possible commands are working correctly when using the CLI. That the program runs without input \\ [3pt] \hline
				Pre-conditions & Code for input handling for all possible commands. Code for error handling for invalid inputs. \\  [3pt] \hline
				
				Execution steps & 	\begin{enumerate}
								\item Run the program for every type of argument
								\item Run the program without arguments
							\end{enumerate} \\ [3pt] \hline

				Expected results & 	\begin{enumerate}
								\item The specified variable is set to the specified value
								\item All the values are loaded from the config file
							\end{enumerate} \\ [3pt] \hline
	\end{longtable}
\end{center}

\newpage
\begin{center}
	\begin{longtable}{ | p{4cm} | p{10cm} | }
	\caption{UNIT-02}\\ \hline
	\textbf{Item} & \textbf{Description} \\ [3pt] \hline \hline
	\endfirsthead
	\multicolumn{2}{c}%
	{\tablename\ \thetable\ -- \textit{Continued from previous page}} \\ \hline
	\textbf{Item} & \textbf{Description}\\ \hline
	\endhead \hline \hline 
	\multicolumn{2}{r}{\textit{Continued on next page}} \\
	\endfoot \hline
	\endlastfoot
				Name & P3P parser \\  [3pt] \hline
				Test identifier & UNIT-02 \\  [3pt] \hline
				Person responsible & Henrik Knutsen \\  [3pt] \hline
				Feature(s) to be tested & That the P3P parser correctly parses all the required fields and their values. \\  [3pt] \hline
				Pre-conditions & The P3P parser must be implemented. Need to have P3P policies with a wide range of cases. \\  [3pt] \hline
				
				Execution steps & 	\begin{enumerate}
								\item Run a P3P xml in the P3P parser and print the parsed fields and their values to console
								\item Manually compare the printed fields and values with the contents of the P3P xml
							\end{enumerate} \\ [3pt] \hline
	
				Expected results &	\begin{enumerate}
								\item The P3P xml is parsed successfully. It's content is printed to console
								\item The printed output have the same fields, each having the same value as those in the xml
							\end{enumerate}	\\  [3pt] \hline
	\end{longtable}
\end{center}
		
\newpage
\begin{center}
	\begin{longtable}{ | p{4cm} | p{10cm} | }
	\caption{UNIT-03}\\ \hline
	\textbf{Item} & \textbf{Description} \\ [3pt] \hline \hline
	\endfirsthead
	\multicolumn{2}{c}%
	{\tablename\ \thetable\ -- \textit{Continued from previous page}} \\ \hline
	\textbf{Item} & \textbf{Description}\\ \hline
	\endhead \hline \hline 
	\multicolumn{2}{r}{\textit{Continued on next page}} \\
	\endfoot \hline
	\endlastfoot
				Name & Local database \\  [3pt] \hline
				Test identifier & UNIT-03 \\  [3pt] \hline
				Person responsible & Henrik Knutsen \\  [3pt] \hline
				Feature(s) to be tested & Writing to and reading from the local database. That the serialization of the database is working. \\  [3pt] \hline
				Pre-conditions & Code for writing to and reading from the database file. Need to have two different P3P policies. \\  [3pt] \hline
				
				Execution steps & 	\begin{enumerate}
								\item Write policy A to the local database
								\item Write policy B to the local database
								\item Read and print policy A from the local database
								\item Read and print policy A from the local database
								\item Compare the written policy A and the read policy A
								\item Compare the written policy B and the read policy B
							\end{enumerate} \\ [3pt] \hline

				Expected results &	\begin{enumerate}
								\item Policy A is successfully written to the database file
								\item Policy B is successfully written to the database file
								\item Policy A is successfully read from the database file and printed
								\item Policy B is successfully read from the database file and printed
								\item The written policy A and the read policy A are identical. They both have the same fields, with the same values
								\item The written policy B and the read policy B are identical. They both have the same fields, with the same values
							\end{enumerate}  \\  [3pt] \hline
	\end{longtable}
\end{center}

\newpage
\begin{center}
	\begin{longtable}{ | p{4cm} | p{10cm} | }
	\caption{UNIT-04}\\ \hline
	\textbf{Item} & \textbf{Description} \\ [3pt] \hline \hline
	\endfirsthead
	\multicolumn{2}{c}%
	{\tablename\ \thetable\ -- \textit{Continued from previous page}} \\ \hline
	\textbf{Item} & \textbf{Description}\\ \hline
	\endhead \hline \hline 
	\multicolumn{2}{r}{\textit{Continued on next page}} \\
	\endfoot \hline
	\endlastfoot
				Name & Graphical user interface (GUI) functionality \\  [3pt] \hline
				Test identifier & UNIT-04 \\  [3pt] \hline
				Person responsible & Henrik Knutsen \\  [3pt] \hline
				Feature(s) to be tested & That all the elements - buttons, lists etc. - are working as intended. \\  [3pt] \hline
				Pre-conditions & GUI with all the necessary listeners must be implemented. Code for running the program with the GUI. \\  [3pt] \hline
				
				Execution steps & 	\begin{enumerate}
								\item Run the program using the GUI
								\item Test every option in the menu bar
								\item Test every button in the configuration menu
								\item Test every scroll bar
								\item Resize the window
							\end{enumerate} \\ [3pt] \hline

				Expected results &	\begin{enumerate}
								\item The program starts and loads the graphical user interface
								\item The action connected to the option is executed successfully						
								\item The action connected to the button/checkbox is executed successfully						
								\item The scroll bar scrolls through the list, up and down, from end to end, successfully
								\item Window can be resized without having elements of the GUI overlapping. The elements and panes scales with the main window
							\end{enumerate} \\  [3pt] \hline
	\end{longtable}
\end{center}

\newpage
\begin{center}
	\begin{longtable}{ | p{4cm} | p{10cm} | }
	\caption{UNIT-05}\\ \hline
	\textbf{Item} & \textbf{Description} \\ [3pt] \hline \hline
	\endfirsthead
	\multicolumn{2}{c}%
	{\tablename\ \thetable\ -- \textit{Continued from previous page}} \\ \hline
	\textbf{Item} & \textbf{Description}\\ \hline
	\endhead \hline \hline 
	\multicolumn{2}{r}{\textit{Continued on next page}} \\
	\endfoot \hline
	\endlastfoot
				Name & Algorithm classification \\  [3pt] \hline
				Test identifier & UNIT-05 \\  [3pt] \hline
				Person responsible & Henrik Knutsen, Dimitry Kongevold \\  [3pt] \hline
				Feature(s) to be tested & That the k-nearest neighbor algorithm bases its decision on the k most similar policies \\  [3pt] \hline
				Pre-conditions & Code for reading from the weights file must be implemented. A working k-nearest neighbor algorithm that uses the weights must be implemented. Need one policy to test on, and a set of policies to be used as history. \\  [3pt] \hline
				
				Execution steps & 	\begin{enumerate}
								\item Load a set of policies into the database file
								\item Manually calculate and write down the distances between the single policy and each of the policies in the history
								\item Run the distance algorithm on a single policy and the history and compare the distances that are returned by the algorithm with the manually calculated distances from step 2
								\item Manually find the k policies with the lowest distances
								\item Run the reduction algorithm with necessary input to find the k nearest policies and compare the k policies returned by the reduction algorithm with those found in step 4
								\item Run the conclusion algorithm and verify the results returned by the algorithm ???
							\end{enumerate} \\ [3pt] \hline

			Expected results &	\begin{enumerate}
							\item The policies are added to the database file
							\item The six distances are obtained
							\item The algorithm returns the same distances as those found in step 2
							\item The k policies are obtained
							\item The algorithm returns the same k policies as those found in step 4
							\item ??
						\end{enumerate} \\  [3pt] \hline
	\end{longtable}
\end{center}

\newpage
\begin{center}
	\begin{longtable}{ | p{4cm} | p{10cm} | }
	\caption{UNIT-06}\\ \hline
	\textbf{Item} & \textbf{Description} \\ [3pt] \hline \hline
	\endfirsthead
	\multicolumn{2}{c}%
	{\tablename\ \thetable\ -- \textit{Continued from previous page}} \\ \hline
	\textbf{Item} & \textbf{Description}\\ \hline
	\endhead \hline \hline 
	\multicolumn{2}{r}{\textit{Continued on next page}} \\
	\endfoot \hline
	\endlastfoot
				Name & Algorithm learning \\  [3pt] \hline
				Test identifier & UNIT-06 \\  [3pt] \hline
				Person responsible & Henrik Knutsen, Neshahavan Karunakaran\\  [3pt] \hline
				Feature(s) to be tested & That the weights file is updated when a new policy is added to history. \\  [3pt] \hline
				Pre-conditions & Code for reading from and writing to the weights file. Code for writing to the database. Algorithms for classification and learning must be implemented. \\  [3pt] \hline

				Execution steps & 	\begin{enumerate}
								\item Make a set of policies and load the set into the history
								\item Read the weights from the weights file
								\item Run the classification and learning algorithms on the policy to be classified and the history, with the weights from step 2
								\item Read the weights from the weights file
								\item Compare the contents of the weights files obtained in steps 2 and 4
							\end{enumerate} \\ [3pt] \hline

				Expected results &	\begin{enumerate}
								\item The policies are loaded into the history successfully
								\item The weights are written down
								\item The classification and learning algorithm runs successfully on the policy to be classified and the history
								\item The weights are loaded
								\item The weights loaded in step 4 are different from the weights written down in step 2
							\end{enumerate} \\  [3pt] \hline
	\end{longtable}
\end{center}

\newpage
\begin{center}
	\begin{longtable}{ | p{4cm} | p{10cm} | }
	\caption{UNIT-07}\\ \hline
	\textbf{Item} & \textbf{Description} \\ [3pt] \hline \hline
	\endfirsthead
	\multicolumn{2}{c}%
	{\tablename\ \thetable\ -- \textit{Continued from previous page}} \\ \hline
	\textbf{Item} & \textbf{Description}\\ \hline
	\endhead \hline \hline 
	\multicolumn{2}{r}{\textit{Continued on next page}} \\
	\endfoot \hline
	\endlastfoot
				Name & Interaction with community databas \\  [3pt] \hline
				Test identifier & UNIT-07 \\  [3pt] \hline
				Person responsible & Henrik Knutsen \\  [3pt] \hline
				Feature(s) to be tested & That packets can be sent between the client program and the community database.  \\  [3pt] \hline
				Pre-conditions & A running local client. A (virtual) server. Code for sending and receiving packets must be implemented. \\  [3pt] \hline
				
				Execution steps & 	\begin{enumerate}
								\item a
							\end{enumerate} \\ [3pt] \hline

				Expected results &	\begin{enumerate}
								\item a
							\end{enumerate} \\  [3pt] \hline
	\end{longtable}
\end{center}

\end{document}
% Activate the following line by filling in the right side. If for example the name of the root file is Main.tex, write
% "...root = Main.tex" if the chapter file is in the same directory, and "...root = ../Main.tex" if the chapter is in a subdirectory.
 
%!TEX root =  

\chapter{Test execution}

\begin{center}
\begin{longtable}{ | p{4cm} | p{10cm} | }
\caption{UNIT-01}\\
\hline
\textbf{Item} & \textbf{Description} \\ [3pt]
\hline \hline
\endfirsthead
\multicolumn{2}{c}%
{\tablename\ \thetable\ -- \textit{Continued from previous page}} \\
\hline
\textbf{Item} & \textbf{Description}\\
\hline
\endhead
\hline
\hline 
\multicolumn{2}{r}{\textit{Continued on next page}} \\
\endfoot
\hline
\endlastfoot

Name & Command line interface (CLI) functionality \\  [3pt] \hline
Test identifier & UNIT-01 \\  [3pt] \hline
Person responsible & Henrik Knutsen \\  [3pt] \hline
Date of first execution & October 24th \\ [3pt]
Date of completion & November 16th \\ [3pt] \hline
			
			Execution steps & 	\begin{enumerate}
							\item Run the program for every type of argument
							\item Run the program without arguments
						\end{enumerate} \\ [3pt] \hline

			Steps executed & 	\begin{enumerate}
							\item
							\begin{enumerate}
								\item Run with -logloc logTest.txt
								\item Run with -loglevel ALL
								\item Run with -inDBLoc database.db
								\item Run with -outDBLoc database.db
								\item Run with -inWeightsLoc testWeights.cfg
								\item Run with -outWeightsLoc testWeights.cfg
								\item Run with -newDB false
								\item Run with -p3pLocation ticketmaster1.xml
								\item Run with -p3pDirLocation P3P
								\item Run with -blanketAccept true
								\item Run with -newPolicyLoc test.xml
								\item Run with -userInit true
								\item Run with -userResponse //TODO
								\item Run with -cbrV bitmapDistanceWisOne, Reduction\_KNN ,Conclusion\_Simple, LearnAlgSimpler
								\item Run with -userIO UserIO\_Simple
								\item Run with -policyDB PDatabase
								\item Run with -genConfig test.cfg
								\item Run with -NetworkRType NRCouchdb
								\item Run with -NetworkROptions privacydb, false, http, vm-6113.idi.ntnu.no, 5948, PA, 1234
								\item Run with -confidenceLevel 1.5
								\item Run with -useNet true
							\end{enumerate}
							\item Run the program without arguments
						\end{enumerate} \\ [3pt] \hline
			
			Expected results & 	\begin{enumerate}
							\item
							\begin{enumerate}
								\item Logfile is created at the specified filepath
								\item loglevel is set to the specified value
								\item inDBLoc is set to the specified value
								\item outDBLoc is set to the specified value
								\item inWeightsLoc is set to the specified filepath
								\item outWeightsLoc is set to the specified filepath
								\item newDB is set to false
								\item p3pLocation is set to the specified file
								\item p3pDirLocation is set to the specified filepath
								\item recommendation is automatically accepted
								\item newPolicyLoc is set to the specified filepath
								\item userInit is set to the specified value
								\item the specified action is appended to the policy
								\item cbrV is set to the specified classes
								\item userIO is set to the specified class
								\item policyDB is set to the specified class
								\item genConfig is set to the specified file. Values of the specified config file are loaded
								\item NetworkRType is set to the specified class
								\item NetworkROptions is set to the specified values
								\item confidenceLevel is set to the specified value
								\item useNet is set to the specified value
							\end{enumerate}
							\item All the values are loaded from the config file
						\end{enumerate} \\ [3pt] \hline

			Step results & 	\begin{enumerate}
							\item
							\begin{enumerate}
								\item Logfile is created at the specified filepath
								\item loglevel is set to ALL
								\item inDBLoc is set to database.db
								\item outDBLoc is set to database.db
								\item inWeightsLoc is set to testWeights.cfg
								\item outWeightsLoc is set to testWeights.cfg
								\item newDB is set to false
								\item p3pLocation is set to ticketmaster1.xml
								\item p3pDirLocation is set to P3P
								\item Recommendation is automatically accepted
								\item newPolicyLoc is set to test.xml
								\item userInit is set to true
								\item userResponse ---
								\item cbrV is set to bitmapDistanceWisOne, Reduction\_KNN, Conclusion\_Simple, LearnAlgSimpler
								\item userIO is set to UserIO\_Simple
								\item policyDB is set to the specified class
								\item genConfig is set to test.cfg. Values in test.cfg are loaded
								\item NetworkRType is set to NRCouchdb
								\item NetworkROptions is set to privacydb, false, http, vm-6113.idi.ntnu.no, 5984, PA, 1234
								\item confidenceLevel is set to 1.5, recommendation is gives from (default) community database
								\item useNet is set to true, networking is enabled
							\end{enumerate}
							\item All the values are loaded from the default config file
						\end{enumerate} \\ [3pt] \hline

			Test conclusion & 	\begin{enumerate}
							\item
							\begin{enumerate}
								\item PASS
								\item PASS
								\item PASS
								\item PASS
								\item PASS
								\item PASS								
								\item PASS
								\item PASS
								\item PASS
								\item PASS
								\item PASS								
								\item PASS
								\item NO PASS
								\item PASS
								\item PASS
								\item (NO) PASS
								\item PASS
								\item PASS
								\item PASS
								\item PASS
								\item PASS
							\end{enumerate}
							\item PASS
						\end{enumerate}
						Test not passed. 1M (and 1P) failed \\ [3pt] \hline

			Comments & -
					\\ [3pt] \hline
\end{longtable}
\end{center}

\newpage
\begin{center}
\begin{longtable}{ | p{4cm} | p{10cm} | }
\caption{UNIT-02}\\
\hline
\textbf{Item} & \textbf{Description} \\
\hline \hline
\endfirsthead
\multicolumn{2}{c}%
{\tablename\ \thetable\ -- \textit{Continued from previous page}} \\
\hline
\textbf{Item} & \textbf{Description}\\
\hline
\endhead
\hline
\hline 
\multicolumn{2}{r}{\textit{Continued on next page}} \\
\endfoot
\hline
\endlastfoot

Name & P3P parser \\  [3pt] \hline
Test identifier & UNIT-02 \\  [3pt] \hline
Person responsible & Henrik Knutsen \\  [3pt] \hline
Date of first execution & October 24th \\ [3pt]
Date of completion & November 7th \\ [3pt] \hline

Execution steps & 	\begin{enumerate}
				\item Run a P3P xml in the P3P parser and print the parsed fields and their values to console
				\item Manually compare the printed fields and values with the contents of the P3P xml
			\end{enumerate} \\ [3pt] \hline

			Steps executed & 	\begin{enumerate}
							\item Test for barnesandnoble.com
							\begin{enumerate}
								\item barnesandnoble.xml is parsed and printed to console
								\item Contents of the xml is compared with to what was printed in step 1(a)
							\end{enumerate}

							\item Test for daduru.com
							\begin{enumerate}
								\item daduru.com is parsed and printed to console
								\item Contents of the xml is compared with to what was printed in step 2(a)
							\end{enumerate}

							\item Test for ssa.gov
							\begin{enumerate}
								\item ssa.gov is parsed and printed to console
								\item Contents of the xml is compared with to what was printed in step 3(a)
							\end{enumerate}

							\item Test for toysrus.com
							\begin{enumerate}
								\item toysrus.com is parsed and printed to console
								\item Contents of the xml is compared with to what was printed in step 4(a)
							\end{enumerate}

							\item Test for gunbroker.com
							\begin{enumerate}
								\item gunbroker.com is parsed and printed to console
								\item Contents of the xml is compared with to what was printed in step 5(a)
							\end{enumerate}

							\item Test for latimes.com
							\begin{enumerate}
								\item latimes.com is parsed and printed to console
								\item Contents of the xml is compared with to what was printed in step 6(a)
							\end{enumerate}

							\item Test for planedesire.com
							\begin{enumerate}
								\item planedesire.com is parsed and printed to console
								\item Contents of the xml is compared with to what was printed in step 7(a)
							\end{enumerate}

							\item  Test for yahoo.com
							\begin{enumerate}
								\item yahoo.com is parsed and printed to console
								\item Contents of the xml is compared with to what was printed in step 8(a)
							\end{enumerate}

							\item Test for nextel.com
							\begin{enumerate}
								\item nextel.com is parsed and printed to console
								\item Contents of the xml is compared with to what was printed in step 9(a)
							\end{enumerate}

							\item Test for ebay.xml
							\begin{enumerate}
								\item ebay.com is parsed and printed to console
								\item Contents of the xml is compared with to what was printed in step 10(a)
							\end{enumerate}
						\end{enumerate} \\ [3pt] \hline
			
			Expected results &	\begin{enumerate}
							\item The P3P xml is parsed successfully. It's content is printed to console
							\item The printed output have the same fields, each having the same value as those in the xml
						\end{enumerate}
							 \\  [3pt] \hline

			Step results & 	\begin{enumerate}
							\item Results for barnesandnoble.com
							\begin{enumerate}
								\item The P3P xml is parsed successfully. It's content is printed to console
								\item The printed output have the same fields, each having the same values as those in the xml
							\end{enumerate}

							\item Results for daduru.com
							\begin{enumerate}
								\item The P3P xml is parsed successfully. It's content is printed to console
								\item The printed output have the same fields, each having the same values as those in the xml
							\end{enumerate}

							\item Results for ssa.gov
							\begin{enumerate}
								\item The P3P xml is parsed successfully. It's content is printed to console
								\item The printed output have the same fields, each having the same values as those in the xml
							\end{enumerate}

							\item Results for toysrus.com
							\begin{enumerate}
								\item The P3P xml is parsed successfully. It's content is printed to console
								\item The printed output have the same fields, each having the same values as those in the xml
							\end{enumerate}
							
							\item Results for gunbroker.com
							\begin{enumerate}
								\item The P3P xml is parsed successfully. It's content is printed to console
								\item The printed output have the same fields, each having the same values as those in the xml
							\end{enumerate}

							\item Results for latimes.com
							\begin{enumerate}
								\item The P3P xml is parsed successfully. It's content is printed to console
								\item The printed output have the same fields, each having the same values as those in the xml
							\end{enumerate}

							\item Results for planedesire.com
							\begin{enumerate}
								\item The P3P xml is parsed successfully. It's content is printed to console
								\item The printed output have the same fields, each having the same values as those in the xml
							\end{enumerate}

							\item Results for yahoo.com
							\begin{enumerate}
								\item The P3P xml is parsed successfully. It's content is printed to console
								\item The printed output have the same fields, each having the same values as those in the xml
							\end{enumerate}

							\item Results for nextel.com
							\begin{enumerate}
								\item The P3P xml is parsed successfully. It's content is printed to console
								\item The printed output have the same fields, each having the same values as those in the xml
							\end{enumerate}
					
							\item Results for ebay.com
							\begin{enumerate}
								\item The P3P xml is parsed successfully. It's content is printed to console
								\item The printed output have the same fields, each having the same values as those in the xml
							\end{enumerate}
						\end{enumerate}
							 \\  [3pt] \hline

			Test conclusion & 	\begin{enumerate}
							\item PASS
							\item PASS
							\item PASS
							\item PASS
							\item PASS
							\item PASS
							\item PASS
							\item PASS
							\item PASS
							\item PASS
						\end{enumerate}
						Test passed \\  [3pt] \hline
			Comments & As mentioned in the test plan, it is not guaranteed that the parser is successfully parsing every possible field of every possible policy even though it has passed this test
					\\ [3pt] \hline
		\end{longtable}
	\end{center}

\newpage
\begin{center}
\begin{longtable}{ | p{4cm} | p{10cm} | }
\caption{UNIT-03}\\
\hline
\textbf{Item} & \textbf{Description} \\
\hline \hline
\endfirsthead
\multicolumn{2}{c}%
{\tablename\ \thetable\ -- \textit{Continued from previous page}} \\
\hline
\textbf{Item} & \textbf{Description}\\
\hline
\endhead
\hline
\hline 
\multicolumn{2}{r}{\textit{Continued on next page}} \\
\endfoot
\hline
\endlastfoot

Name & Local database \\  [3pt] \hline
Test identifier & UNIT-03 \\  [3pt] \hline
Person responsible & Henrik Knutsen \\  [3pt] \hline
Date of first execution & October 24th \\ [3pt]
Date of completion & October 24th \\ [3pt] \hline

			Execution steps & 	\begin{enumerate}
							\item Write policy A to the local database
							\item Write policy B to the local database
							\item Read and print policy A from the local database
							\item Read and print policy A from the local database
							\item Compare the written policy A and the read policy A
							\item Compare the written policy B and the read policy B
						\end{enumerate} \\ [3pt] \hline

			Steps executed & 	\begin{enumerate}
							\item Program is started writing policy A to an empty database
							\item Accept recommendation and chose to save the new action and policy (policy B)
							\item Print the new database after policy B was added (step 2)
							\item Done in step 3
							\item The contents of policy A that was written in step 1 is compared to what was printed of policy A in step 3
							\item The contents of policy B that was written in step 2 is compared to what was printed of policy B in step 3
						\end{enumerate} \\ [3pt] \hline
			
			Expected results &	\begin{enumerate}
							\item Policy A is successfully written to the database file
							\item Policy B is successfully written to the database file
							\item Policy A is successfully read from the database file and printed
							\item Policy B is successfully read from the database file and printed
							\item The written policy A and the read policy A are identical. They both have the same fields, with the same values
							\item The written policy B and the read policy B are identical. They both have the same fields, with the same values
						\end{enumerate}
							 \\  [3pt] \hline

			Step results & 	\begin{enumerate}
							\item Policy A was successfully written to the database file
							\item Policy B was successfully written to the database file
							\item Database was successfully printed
							\item Same as step 3
							\item The contents of the loaded policy A and printed contents of policy A are identical
							\item The contents of the loaded policy B and printed contents of policy B are identical
						\end{enumerate}
							 \\  [3pt] \hline

			Test conclusion & 	\begin{enumerate}
							\item PASS
							\item PASS
							\item PASS
							\item PASS
							\item PASS
							\item PASS
						\end{enumerate}
						Test passed \\  [3pt] \hline
			Comments &	- \\ [3pt] \hline
		\end{longtable}
	\end{center}

\newpage
\begin{center}
\begin{longtable}{ | p{4cm} | p{10cm} | }
\caption{UNIT-04}\\
\hline
\textbf{Item} & \textbf{Description} \\
\hline \hline
\endfirsthead
\multicolumn{2}{c}%
{\tablename\ \thetable\ -- \textit{Continued from previous page}} \\
\hline
\textbf{Item} & \textbf{Description}\\
\hline
\endhead
\hline
\hline 
\multicolumn{2}{r}{\textit{Continued on next page}} \\
\endfoot
\hline
\endlastfoot

Name & Graphical user interface (GUI) functionality \\  [3pt] \hline
Test identifier & UNIT-04 \\  [3pt] \hline
Person responsible & Henrik Knutsen \\  [3pt] \hline
Date of execution & October 29th \\  [3pt] \hline
Date of completion & November ??th \\ [3pt] \hline

			Execution steps & 	\begin{enumerate}
							\item Run the program using the GUI
							\item Test every option in the menu bar
							\item Test every button in the configuration menu
							\item Test every scroll bar
							\item Resize the window
						\end{enumerate} \\ [3pt] \hline

			Steps executed & 	\begin{enumerate}
							\item Program started with graphical user interface

							\item Chose every option in the menu bar
							\begin{enumerate}
								\item Clicked "Configuration"
								\item Clicked "Reload Database"
								\item Clicked "Run"
								\item Clicked "Exit"
							\end{enumerate}

							\item Clicked every button in the configuration menu
							\begin{enumerate}
								\item a
								\item b
								\item c
							\end{enumerate}

							\item Used every scroll bar
							\begin{enumerate}
								\item Used scroll bar for scrolling up/down in database pane
								\item Used scroll bar for scrolling left/right in database pane
								\item Used scroll bar for scrolling up/down in new policy pane
								\item Used scroll bar for scrolling left/right in new policy pane
								\item Used scroll bar for scrolling up/down in output pane
							\end{enumerate}

							\item Attempted to resize the window
						\end{enumerate} \\ [3pt] \hline
			
			Expected results &	\begin{enumerate}
							\item The program starts and loads the graphical user interface

							\item  
							\begin{enumerate}
								\item Menu for setting config values is opened
								\item The specified database is loaded
								\item The program gives a popup with a recommendation for the selected policy based on the history in the specified database
								\item Program closes
							\end{enumerate}
							
							\item
							\begin{enumerate}
								\item a
								\item b
								\item c
							\end{enumerate}

							\item 
							\begin{enumerate}
								\item The scroll bar scrolls through the list, up and down, from end to end, successfully
								\item The scroll bar scrolls through the list, up and down, from end to end, successfully
								\item The scroll bar scrolls through the list, up and down, from end to end, successfully
								\item The scroll bar scrolls through the list, up and down, from end to end, successfully									\item The scroll bar scrolls through the list, up and down, from end to end, successfully
							\end{enumerate}

							\item Window can be resized without having elements of the GUI overlapping. The elements and panes scales with the main window
						\end{enumerate}
							 \\  [3pt] \hline

			Step results & 	\begin{enumerate}
							\item The program starts and loads the graphical user interface

							\item  
							\begin{enumerate}
								\item Menu for setting config values is not opened
								\item The specified database is loaded
								\item The program gives a popup with a recommendation for the selected policy based on the history in the specified database
								\item Program closes
							\end{enumerate}
							
							\item
							\begin{enumerate}
								\item a
								\item b
								\item c
							\end{enumerate}

							\item 
							\begin{enumerate}
								\item The scroll bar appears when conent of the pane is too big to fit, and it scrolls through the list, up and down, from end to end, successfully
								\item The scroll bar appears when conent of the pane is too big to fit, and it scrolls through the list, left and right, from end to end, successfully
								\item The scroll bar appears when conent of the pane is too big to fit, and it scrolls through the list, up and down, from end to end, successfully		
								\item The scroll bar appears when conent of the pane is too big to fit, and it scrolls through the list, left and right, from end to end, successfully
								\item The scroll bar appears when conent of the pane is too big to fit, and it scrolls through the list, up and down, from end to end, successfully
							\end{enumerate}

							\item The window can be resized. The panes are scaling with the main window
						\end{enumerate}
							\\ [3pt] \hline

			Test conclusion & 	\begin{enumerate}
							\item PASS

							\item  
							\begin{enumerate}
								\item NO PASS
								\item PASS
								\item PASS
								\item PASS
							\end{enumerate}
							
							\item
							\begin{enumerate}
								\item NO PASS
								\item NO PASS
								\item NO PASS
							\end{enumerate}

							\item 
							\begin{enumerate}																							\item PASS
								\item PASS
								\item PASS
								\item PASS
								\item PASS
							\end{enumerate}

							\item PASS
						\end{enumerate}

						Test not passed. 2A and 3 failed \\ [3pt] \hline
			Comments &	- \\ [3pt] \hline
		\end{longtable}
	\end{center}

\newpage
\begin{center}
\begin{longtable}{ | p{4cm} | p{10cm} | }
\caption{UNIT-05}\\
\hline
\textbf{Item} & \textbf{Description} \\
\hline \hline
\endfirsthead
\multicolumn{2}{c}%
{\tablename\ \thetable\ -- \textit{Continued from previous page}} \\
\hline
\textbf{Item} & \textbf{Description}\\
\hline
\endhead
\hline
\hline 
\multicolumn{2}{r}{\textit{Continued on next page}} \\
\endfoot
\hline
\endlastfoot

Name & Algorithm classification \\  [3pt] \hline
Test identifier & UNIT-05 \\  [3pt] \hline
Person responsible & Henrik Knutsen \& Dimitry Kongevold \\  [3pt] \hline
Date of execution & October 30th \\  [3pt]
Date of completion & November 1st \\ [3pt] \hline
			
			Execution steps & 	\begin{enumerate}
							\item Load a set of policies into the database file
							\item Manually calculate and write down the distances between the single policy and each of the policies in the history
							\item Run the distance algorithm on a single policy and the history and compare the distances that are returned by the algorithm with the manually calculated distances from step 2
							\item Manually find the k policies with the lowest distances
							\item Run the reduction algorithm with necessary input to find the k nearest policies and compare the k policies returned by the reduction algorithm with those found in step 4
							\item Run the conclusion algorithm and verify the results returned by the algorithm ???
						\end{enumerate} \\ [3pt] \hline

			Steps executed & 	\begin{enumerate}
							\item Created a test domain by loading six policies into the history
							\item Distances between the policy to be classified and each of the six policies were calculated manually and the results were inserted into a table
							\item A JUnit test testReduction\_KNN was created. This test was used to assert that the six values returned by the algorithm are the same as those calculated manually in step 2
							\item Found the k nearest policies from what was calculated in step 2
							\item A JUnit test testReduction\_KNN was created. This test was used to assert that the reduction algorithm returns the same k nearest policies as those that was found manually in step 4
							\item Created a JUnit test testReduction\_KNN ???
						\end{enumerate} \\ [3pt] \hline
			
			Expected results &	\begin{enumerate}
							\item The policies are added to the database file
							\item The six distances are obtained
							\item The algorithm returns the same distances as those found in step 2
							\item The k policies are obtained
							\item The algorithm returns the same k policies as those found in step 4
							\item ??
						\end{enumerate}
							 \\  [3pt] \hline

			Step results & 	\begin{enumerate}
							\item The policies are added to the database file
							\item The six distances are obtained
							\item The JUnit is successful
							\item The k policies are obtained
							\item The JUnit test is successful
							\item The JUnit test is not run
						\end{enumerate}
							 \\  [3pt] \hline

			Test conclusion & 	\begin{enumerate}
							\item PASS
							\item PASS
							\item PASS
							\item PASS
							\item PASS
							\item NO PASS
						\end{enumerate}
						Test not passed. 6 failed \\  [3pt] \hline
			Comments & As mentioned in the test plan, it is not guaranteed that the algorithm will classify correctly for every possible combination of policy and database history even though it has passed this test
					\\ [3pt] \hline
		\end{longtable}
	\end{center}

\newpage
\begin{center}
\begin{longtable}{ | p{4cm} | p{10cm} | }
\caption{UNIT-06}\\
\hline
\textbf{Item} & \textbf{Description} \\
\hline \hline
\endfirsthead
\multicolumn{2}{c}%
{\tablename\ \thetable\ -- \textit{Continued from previous page}} \\
\hline
\textbf{Item} & \textbf{Description}\\
\hline
\endhead
\hline
\hline 
\multicolumn{2}{r}{\textit{Continued on next page}} \\
\endfoot
\hline
\endlastfoot

Name & Algorithm learning \\  [3pt] \hline
Test identifier & UNIT-06 \\  [3pt] \hline
Person responsible & Henrik Knutsen \& Neshahavan Karunakaran \\  [3pt] \hline
Date of execution & November 4th \\  [3pt] 
Date of completion & November 12th \\ [3pt] \hline

			Execution steps & 	\begin{enumerate}
							\item Make a set of policies and load the set into the history
							\item Read the weights from the weights file
							\item Run the classification and learning algorithms on the policy to be classified and the history, with the weights from step 2
							\item Read the weights from the weights file
							\item Compare the contents of the weights files obtained in steps 2 and 4
						\end{enumerate} \\ [3pt] \hline

			Steps executed & 	\begin{enumerate}
							\item Program is started with a set of policies used to build a history
							\item Contents of the weights file is written down
							\item A JUnit test LearnAlgSimplerTest is created. The test runs the classification and learning algortihms on the policy to be classified and the history
							\item The test loads the weights from the weights file
							\item The test compares the weights loaded in step 4 with the values that was written down in step 2
						\end{enumerate} \\ [3pt] \hline
			
			Expected results &	\begin{enumerate}
							\item The policies are loaded into the history successfully
							\item The weights are written down
							\item The classification and learning algorithm runs successfully on the policy to be classified and the history
							\item The weights are loaded
							\item The weights loaded in step 4 are different from the weights written down in step 2
						\end{enumerate}
							 \\  [3pt] \hline

			Step results & 	\begin{enumerate}
							\item The policies are loaded into the history successfully
							\item The weights are written down
							\item The JUnit test runs the classification and learning algorithm successfully
							\item The JUnit test loads the weights file successfully
							\item The JUnit test confirms that the values in the weights file have changed
						\end{enumerate}
							 \\  [3pt] \hline

			Test conclusion & 	\begin{enumerate}
							\item PASS
							\item PASS
							\item PASS
							\item PASS
							\item PASS
						\end{enumerate}
						Test passed \\ [3pt] \hline
			Comments & Some changes are needed for this test to run. These changes are mentioned in the test class
				\\ [3pt] \hline
		\end{longtable}
	\end{center}

\newpage
\begin{center}
\begin{longtable}{ | p{4cm} | p{10cm} | }
\caption{UNIT-07}\\
\hline
\textbf{Item} & \textbf{Description} \\
\hline \hline
\endfirsthead
\multicolumn{2}{c}%
{\tablename\ \thetable\ -- \textit{Continued from previous page}} \\
\hline
\textbf{Item} & \textbf{Description}\\
\hline
\endhead
\hline
\hline 
\multicolumn{2}{r}{\textit{Continued on next page}} \\
\endfoot
\hline
\endlastfoot

Name & Interaction with community database \\  [3pt] \hline
Test identifier & UNIT-07 \\  [3pt] \hline
Person responsible & Henrik Knutsen \\  [3pt] \hline
Date of execution & November 14th \\  [3pt]
Date of completion & November 14th \\ [3pt] \hline

			Execution steps & 	\begin{enumerate}
							\item a
						\end{enumerate} \\ [3pt] \hline

			Steps executed & 	\begin{enumerate}
							\item b
						\end{enumerate} \\ [3pt] \hline
			
			Expected results &	\begin{enumerate}
							\item c
						\end{enumerate}
							 \\  [3pt] \hline

			Step results & 	\begin{enumerate}
							\item d
						\end{enumerate}
							 \\  [3pt] \hline

			Test conclusion & 	\begin{enumerate}
							\item d
						\end{enumerate}
						Test failed \\ [3pt] \hline
		\end{longtable}
	\end{center}
% Activate the following line by filling in the right side. If for example the name of the root file is Main.tex, write
% "...root = Main.tex" if the chapter file is in the same directory, and "...root = ../Main.tex" if the chapter is in a subdirectory.
 
%!TEX root =  

\chapter{Templates}

\subsection{Status Reports}
\begin{figure}[htbp]
\begin{center}
\includegraphics[width = \textwidth]{Appendix/statusreportTemp.jpg}
\caption{Status Report Template.}
\label{StatusReportTemplate}
\end{center}
\end{figure}

\subsection{Meeting Notes}
\begin{figure}[hbp]
\begin{center}
\includegraphics[width = \textwidth/3*2]{Appendix/meetingreportTemp.jpg}
\caption{Meeting Report Template.}
\label{MeetingReportTemplate}
\end{center}
\end{figure}
\newpage

\subsection{Time Reporting}
\begin{figure}[hbp]
\begin{center}
\includegraphics[width = \textwidth]{Appendix/timereportTemp.jpg}
\caption{Time Report Template.}
\label{TimeReportTemplate}
\end{center}
\end{figure}

\subsection{Test Plan}
\begin{figure}[hthp]
\begin{center}
\includegraphics[width = \textwidth/3*2]{Appendix/testplanTemp.jpg}
\caption{Test Plan Template.}
\label{TestPlanTemplate}
\end{center}
\end{figure}
\newpage

\subsection{Java documentation}
\begin{figure}[hthp]
\begin{center}
\includegraphics[height = \textheight/3*2]{Appendix/javadocTemp.jpg}
\caption{Javadoc Template.}
\label{JavadocTemplate}
\end{center}
\end{figure}


\makeatletter\@openrightfalse
\chapter{Javadoc}

\section{Main}
\begin{jdclass}[class]{Gio}
\begin{jdclassheader}

\jdpublic 
\jdpackage{com.kpro.main}
\jdinherits{\jdtypesimple{Object}}
\end{jdclassheader}
\begin{jdinheritancetable} \jdInhEntry{\jdtypesimple{Object} clone(  )}{Object}
 \jdInhEntry{\jdtypesimple{boolean} equals( \jdtypesimple{Object} )}{Object}
 \jdInhEntry{\jdtypesimple{void} finalize(  )}{Object}
 \jdInhEntry{\jdtypesimple{Class} getClass(  )}{Object}
 \jdInhEntry{\jdtypesimple{int} hashCode(  )}{Object}
 \jdInhEntry{\jdtypesimple{void} notify(  )}{Object}
 \jdInhEntry{\jdtypesimple{void} notifyAll(  )}{Object}
 \jdInhEntry{\jdtypesimple{String} toString(  )}{Object}
 \jdInhEntry{\jdtypesimple{void} wait( \jdtypesimple{long} )}{Object}
 \jdInhEntry{\jdtypesimple{void} wait( \jdtypesimple{long}, \jdtypesimple{int} )}{Object}
 \jdInhEntry{\jdtypesimple{void} wait(  )}{Object}
\end{jdinheritancetable}
\begin{jdconstructor}
\jdpublic 
\JDpara{\jdtypearray{String}{\lbrack{}\rbrack{}}}{args}{}
\JDthrows{Exception}{Mostly from loadWeights, but should also happen for loadFromConfig}
\JDtext{Constructor fo gio class. There should only be one. Consider this a singleton instance to call I/O messages on.
 Constructs and parses command line arguements as well.}
\end{jdconstructor}
\begin{jdconstructor}
\jdpublic 
\JDpara{\jdtypearray{String}{\lbrack{}\rbrack{}}}{args}{any commandline arguements}
\JDpara{\jdtypesimple{UserIO}}{ui}{the known UserIO object}
\JDthrows{Exception}{Mostly from loadWeights, but should also happen for loadFromConfig}
\JDtext{A constructor permitting a user interface class to launch everything and be in control.}
\end{jdconstructor}
\begin{jdmethod}{configUI}
\jdpublic 
\jdtype{\jdtypesimple{void}}
\JDtext{call the user interface's general configuration method if the userInit option is true, and a user interface exists}
\end{jdmethod}
\begin{jdmethod}{setGenProps}
\jdpublic 
\jdtype{\jdtypesimple{void}}
\JDpara{\jdtypesimple{Properties}}{genProps}{}
\end{jdmethod}
\begin{jdmethod}{loadFromConfig}
\jdpublic 
\jdtype{\jdtypesimple{Properties}}
\JDpara{\jdtypesimple{String}}{fileLoc}{}
\JDtext{Loads the general configuration file, either from provided string, or default location (./PrivacyAdviser.cfg)}
\JDreturn{properties object corresponding to given configuration file}
\end{jdmethod}
\begin{jdmethod}{loadWeights}
\jdpublic 
\jdtype{\jdtypesimple{Properties}}
\JDthrows{Exception}{if there's an issue reading the file (if it doesn't exist, or has an IO error)}
\JDtext{Loads the weights configuration file, from the provided location}
\JDreturn{properties object corresponding to given configuration file}
\end{jdmethod}
\begin{jdmethod}{startLogger}
\jdpublic 
\jdtype{\jdtypesimple{Logger}}
\JDpara{\jdtypesimple{String}}{logLoc}{location of the output log file- a string}
\JDpara{\jdtypesimple{String}}{logLevel}{logging level (is parsed by level.parse())}
\JDtext{startLogger initializes and returns a file at logLoc with the results of logging at level logLevel.}
\JDreturn{Logger object to log to.}
\end{jdmethod}
\begin{jdmethod}{loadDB}
\jdpublic 
\jdtype{\jdtypesimple{void}}
\JDtext{Loads the case history into cache. 
 This is where the background database chosen.}
\end{jdmethod}
\begin{jdmethod}{getPDB}
\jdpublic 
\jdtype{\jdtypesimple{PolicyDatabase}}
\JDtext{returns the only policy database}
\JDreturn{the policy database}
\end{jdmethod}
\begin{jdmethod}{shutdown}
\jdpublic 
\jdtype{\jdtypesimple{void}}
\JDtext{closes resources and write everything to file}
\end{jdmethod}
\begin{jdmethod}{userResponse}
\jdpublic 
\jdtype{\jdtypesimple{PolicyObject}}
\JDpara{\jdtypesimple{PolicyObject}}{n}{the processed policy object}
\JDtext{Generates handles response. This is were we would pass stuff to cli or gui, etc}
\JDreturn{the policyObjected as accepted by user (potentially modified}
\end{jdmethod}
\begin{jdmethod}{loadPO}
\jdpublic 
\jdtype{\jdtypesimple{void}}
\JDtext{returns the policy object from the policyObject option}
\JDreturn{the policy object to be processed}
\end{jdmethod}
\begin{jdmethod}{getPO}
\jdpublic 
\jdtype{\jdtypesimple{PolicyObject}}
\end{jdmethod}
\begin{jdmethod}{isBuilding}
\jdpublic 
\jdtype{\jdtypesimple{boolean}}
\JDtext{returns the true if it should only build}
\JDreturn{true if a CBR should NOT be run}
\end{jdmethod}
\begin{jdmethod}{setWeights}
\jdpublic 
\jdtype{\jdtypesimple{void}}
\JDpara{\jdtypesimple{Properties}}{newWeightP}{the new weights file to save}
\JDtext{saves the new weights to a buffer variable before writing in the shutdown call}
\end{jdmethod}
\begin{jdmethod}{getCBR}
\jdpublic 
\jdtype{\jdtypesimple{CBR}}
\JDthrows{Exception}{}
\JDtext{returns the CBR to use}
\JDreturn{the cbr to use}
\end{jdmethod}
\begin{jdmethod}{getWeights}
\jdpublic 
\jdtype{\jdtypesimple{Properties}}
\JDtext{returns the originally imported set of weights}
\JDreturn{the weights for policy attributes}
\end{jdmethod}
\begin{jdmethod}{showDatabase}
\jdpublic 
\jdtype{\jdtypesimple{void}}
\JDtext{shows the database on the user interface, if the user interface exists and no user response is specied 
 and there is no 'blanketAccept' option.}
\end{jdmethod}
\begin{jdmethod}{fileExists}
\jdpublic 
\jdtype{\jdtypesimple{boolean}}
\JDpara{\jdtypesimple{String}}{filepath}{path of the file to check}
\JDtext{GUI classes should use this to ensure the user passes valid files to load.}
\JDreturn{true if the file exists, else false}
\end{jdmethod}
\begin{jdmethod}{getNR}
\jdpublic 
\jdtype{\jdtypesimple{NetworkR}}
\end{jdmethod}
\begin{jdmethod}{getConfLevel}
\jdpublic 
\jdtype{\jdtypesimple{double}}
\JDtext{gets the confidence level threshold from the configuration}
\JDreturn{the confidence threshold}
\end{jdmethod}
\end{jdclass}

\begin{jdclass}[class]{PrivacyAdviser}
\begin{jdclassheader}

\jdpublic 
\jdpackage{com.kpro.main}
\jdinherits{\jdtypesimple{Object}}
\JDtext{Main class.}
\JDversion{29.09.11.1}
\JDauthor{ngerstle}
\end{jdclassheader}
\begin{jdinheritancetable} \jdInhEntry{\jdtypesimple{Object} clone(  )}{Object}
 \jdInhEntry{\jdtypesimple{boolean} equals( \jdtypesimple{Object} )}{Object}
 \jdInhEntry{\jdtypesimple{void} finalize(  )}{Object}
 \jdInhEntry{\jdtypesimple{Class} getClass(  )}{Object}
 \jdInhEntry{\jdtypesimple{int} hashCode(  )}{Object}
 \jdInhEntry{\jdtypesimple{void} notify(  )}{Object}
 \jdInhEntry{\jdtypesimple{void} notifyAll(  )}{Object}
 \jdInhEntry{\jdtypesimple{String} toString(  )}{Object}
 \jdInhEntry{\jdtypesimple{void} wait( \jdtypesimple{long} )}{Object}
 \jdInhEntry{\jdtypesimple{void} wait( \jdtypesimple{long}, \jdtypesimple{int} )}{Object}
 \jdInhEntry{\jdtypesimple{void} wait(  )}{Object}
\end{jdinheritancetable}
\begin{jdconstructor}
\jdpublic 
\end{jdconstructor}
\begin{jdmethod}{main}
\jdpublic \jdstatic 
\jdtype{\jdtypesimple{void}}
\JDpara{\jdtypearray{String}{\lbrack{}\rbrack{}}}{args}{accepts optional command line arguments, including location of general config file (default pwd)}
\JDthrows{Exception}{}
\JDtext{Program in following sequence- init, load stuff for cbr, run cbr, shutdown.
 should alter for more flexible cbr options (different algorithms, selected by switch
 statement on ReduceChoice, ConclusionChoice, LearnChoice, etc, once those are in the
 config file/cli}
\JDauthor{ngerstle}
\end{jdmethod}
\begin{jdmethod}{init}
\jdpublic \jdstatic 
\jdtype{\jdtypesimple{void}}
\JDpara{\jdtypearray{String}{\lbrack{}\rbrack{}}}{args}{accepts optional command line arguments, including location of general config file (default pwd)}
\JDthrows{Exception}{}
\JDtext{Initializes the program- loads general configuration, starts logger, loads weights, loads database}
\JDauthor{ngerstle}
\end{jdmethod}
\end{jdclass}


\section{Algorithm}
\begin{jdclass}[class]{CBR}
\begin{jdclassheader}

\jdpublic 
\jdpackage{com.kpro.algorithm}
\jdinherits{\jdtypesimple{Object}}
\JDtext{Case based reason. This is a working CBR class that handles process flow between init and shutdown.
 Should be easy to extend and overload various features, but should work for most cases as is.}
\JDauthor{ngerstle}
\JDversion{29.09.11.1}
\end{jdclassheader}
\begin{jdinheritancetable} \jdInhEntry{\jdtypesimple{Object} clone(  )}{Object}
 \jdInhEntry{\jdtypesimple{boolean} equals( \jdtypesimple{Object} )}{Object}
 \jdInhEntry{\jdtypesimple{void} finalize(  )}{Object}
 \jdInhEntry{\jdtypesimple{Class} getClass(  )}{Object}
 \jdInhEntry{\jdtypesimple{int} hashCode(  )}{Object}
 \jdInhEntry{\jdtypesimple{void} notify(  )}{Object}
 \jdInhEntry{\jdtypesimple{void} notifyAll(  )}{Object}
 \jdInhEntry{\jdtypesimple{String} toString(  )}{Object}
 \jdInhEntry{\jdtypesimple{void} wait( \jdtypesimple{long} )}{Object}
 \jdInhEntry{\jdtypesimple{void} wait( \jdtypesimple{long}, \jdtypesimple{int} )}{Object}
 \jdInhEntry{\jdtypesimple{void} wait(  )}{Object}
\end{jdinheritancetable}
\begin{jdconstructor}
\jdpublic 
\JDpara{\jdtypesimple{Gio}}{theIO}{the GIO instance}
\JDpara{\jdtypesimple{Properties}}{weightsConfig}{the weights to use for distance metric \&{} learning algorithm}
\JDpara{\jdtypesimple{ReductionAlgorithm}}{reduceAlg}{the retrival algorithm- produce relevent cases from history}
\JDpara{\jdtypesimple{ConclusionAlgorithm}}{conclusAlg}{the 'reuse' algorithm- produces a solution from relevant cases}
\JDpara{\jdtypesimple{LearnAlgorithm}}{learnAlg}{the retain algorithm- modifies the 'weightsConfig' so the distance metric is more accurate}
\JDtext{Our generic constructor.}
\JDauthor{ngerstle}
\end{jdconstructor}
\begin{jdconstructor}
\jdpublic 
\JDpara{\jdtypesimple{Gio}}{theIO}{}
\JDtext{constructor that so we can call cbr.parse(string)}
\JDauthor{ngerstle}
\end{jdconstructor}
\begin{jdmethod}{run}
\jdpublic 
\jdtype{\jdtypesimple{void}}
\JDpara{\jdtypesimple{PolicyObject}}{newpol}{}
\JDtext{runs through CBR with selected algorithms}
\JDauthor{ngerstle}
\end{jdmethod}
\begin{jdmethod}{parse}
\jdpublic 
\jdtype{\jdtypesimple{CBR}}
\JDpara{\jdtypesimple{String}}{string}{the string from either configuration file or commandline}
\JDthrows{Exception}{}
\JDtext{Parses the CBR option to create the class and instantiate correct algorithms.}
\JDreturn{the CBR defined by the input string}
\end{jdmethod}
\end{jdclass}

\begin{jdclass}[class]{ReductionAlgorithm}
\begin{jdclassheader}

\jdabstract \jdpublic 
\jdpackage{com.kpro.algorithm}
\jdinherits{\jdtypesimple{Object}}
\JDtext{The abstract class for implementing reduction algorithms, like Knearestneighbors.
 ReductionAlgorithm objects store the database, and reduce the set of polices to
 only the relevent policies (one or more).
 May include 'Conclusion'/'Summary' algorithms in the future.}
\JDauthor{ngerstle}
\JDversion{29.09.11.1}
\end{jdclassheader}
\begin{jdinheritancetable} \jdInhEntry{\jdtypesimple{Object} clone(  )}{Object}
 \jdInhEntry{\jdtypesimple{boolean} equals( \jdtypesimple{Object} )}{Object}
 \jdInhEntry{\jdtypesimple{void} finalize(  )}{Object}
 \jdInhEntry{\jdtypesimple{Class} getClass(  )}{Object}
 \jdInhEntry{\jdtypesimple{int} hashCode(  )}{Object}
 \jdInhEntry{\jdtypesimple{void} notify(  )}{Object}
 \jdInhEntry{\jdtypesimple{void} notifyAll(  )}{Object}
 \jdInhEntry{\jdtypesimple{String} toString(  )}{Object}
 \jdInhEntry{\jdtypesimple{void} wait( \jdtypesimple{long} )}{Object}
 \jdInhEntry{\jdtypesimple{void} wait( \jdtypesimple{long}, \jdtypesimple{int} )}{Object}
 \jdInhEntry{\jdtypesimple{void} wait(  )}{Object}
\end{jdinheritancetable}
\begin{jdconstructor}
\jdpublic 
\JDpara{\jdtypesimple{PolicyDatabase}}{pdb}{}
\JDpara{\jdtypearray{String}{\lbrack{}\rbrack{}}}{extraArgs}{}
\JDtext{Constructor for a reductionAlgorithm}
\end{jdconstructor}
\begin{jdmethod}{reduce}
\jdpublic \jdabstract 
\jdtype{\jdtypesimple{ArrayList}}
\JDpara{\jdtypesimple{PolicyObject}}{newPO}{the new policy to consider- it shouldn't change within the algorithm}
\JDtext{the reduce call. returns an arraylist of policies in the policydatabase relevent to newPO}
\JDreturn{a modified newpol}
\JDauthor{ngerstle}
\end{jdmethod}
\end{jdclass}

\begin{jdclass}[class]{Reduction\_KNN}
\begin{jdclassheader}

\jdpublic 
\jdpackage{com.kpro.algorithm}
\jdinherits{\jdtypesimple{Object}\jdinh \jdtypesimple{ReductionAlgorithm}}
\JDtext{A k-nearest-neighbors algorithm class. create it and call run on it to
 get the nearest k neighbors to the object passed to run().}
\JDauthor{ngerstle}
\JDversion{29.09.11.1}
\end{jdclassheader}
\begin{jdinheritancetable} \jdInhEntry{\jdtypesimple{PolicyDatabase} pdb}{ReductionAlgorithm}
 \jdInhEntry{\jdtypesimple{ArrayList} reduce( \jdtypesimple{PolicyObject} )}{ReductionAlgorithm}
 \jdInhEntry{\jdtypesimple{Object} clone(  )}{Object}
 \jdInhEntry{\jdtypesimple{boolean} equals( \jdtypesimple{Object} )}{Object}
 \jdInhEntry{\jdtypesimple{void} finalize(  )}{Object}
 \jdInhEntry{\jdtypesimple{Class} getClass(  )}{Object}
 \jdInhEntry{\jdtypesimple{int} hashCode(  )}{Object}
 \jdInhEntry{\jdtypesimple{void} notify(  )}{Object}
 \jdInhEntry{\jdtypesimple{void} notifyAll(  )}{Object}
 \jdInhEntry{\jdtypesimple{String} toString(  )}{Object}
 \jdInhEntry{\jdtypesimple{void} wait( \jdtypesimple{long} )}{Object}
 \jdInhEntry{\jdtypesimple{void} wait( \jdtypesimple{long}, \jdtypesimple{int} )}{Object}
 \jdInhEntry{\jdtypesimple{void} wait(  )}{Object}
\end{jdinheritancetable}
\begin{jdconstructor}
\jdpublic 
\JDpara{\jdtypesimple{DistanceMetric}}{distanceMetric}{the class defining distance between objects}
\JDpara{\jdtypesimple{PolicyDatabase}}{database}{the database of objects to operate on}
\JDpara{\jdtypearray{String}{\lbrack{}\rbrack{}}}{extraArgs}{from the config file}
\JDtext{creates a kNearestNeighbors algorithm to use}
\JDauthor{ngerstle, ulfnore}
\end{jdconstructor}
\begin{jdmethod}{reduce}
\jdpublic 
\jdtype{\jdtypesimple{ArrayList}}
\JDpara{\jdtypesimple{PolicyObject}}{newPO}{the new PolicyObject the thing to find the neighbors of}
\JDtext{the method that returns the closest k objects to the parameter.
 works by sorting elements by distance from passed object, and passing the first
 k elements.}
\JDreturn{ArrayList<PolicyObject> an arraylist of size k of the nearest neighbors}
\JDauthor{ngerstle}
\end{jdmethod}
\end{jdclass}

\begin{jdclass}[class]{DistanceMetric}
\begin{jdclassheader}

\jdabstract \jdpublic 
\jdpackage{com.kpro.algorithm}
\jdinherits{\jdtypesimple{Object}}
\JDtext{An abstract Distance metric class.
 A DistanceMetric interface has to contain 3 methods
 method for calculation of distance between Recipients, Purposes and Retentions
 between cases
 and distance for data-type string}
\JDversion{160911.1}
\JDauthor{dimitryk}
\end{jdclassheader}
\begin{jdinheritancetable} \jdInhEntry{\jdtypesimple{Object} clone(  )}{Object}
 \jdInhEntry{\jdtypesimple{boolean} equals( \jdtypesimple{Object} )}{Object}
 \jdInhEntry{\jdtypesimple{void} finalize(  )}{Object}
 \jdInhEntry{\jdtypesimple{Class} getClass(  )}{Object}
 \jdInhEntry{\jdtypesimple{int} hashCode(  )}{Object}
 \jdInhEntry{\jdtypesimple{void} notify(  )}{Object}
 \jdInhEntry{\jdtypesimple{void} notifyAll(  )}{Object}
 \jdInhEntry{\jdtypesimple{String} toString(  )}{Object}
 \jdInhEntry{\jdtypesimple{void} wait( \jdtypesimple{long} )}{Object}
 \jdInhEntry{\jdtypesimple{void} wait( \jdtypesimple{long}, \jdtypesimple{int} )}{Object}
 \jdInhEntry{\jdtypesimple{void} wait(  )}{Object}
\end{jdinheritancetable}
\begin{jdmethod}{getTotalDistance}
\jdpublic \jdabstract 
\jdtype{\jdtypesimple{double}}
\JDpara{\jdtypesimple{PolicyObject}}{a}{input PolicyObject}
\JDpara{\jdtypesimple{PolicyObject}}{b}{input PolicyObject}
\JDtext{Calculates total distance between two policies}
\JDauthor{dimitryk}
\JDreturn{double 0 if cases are similar and positive integer if they are not}
\end{jdmethod}
\end{jdclass}

\begin{jdclass}[class]{Bitmapwithdata}
\begin{jdclassheader}

\jdpublic 
\jdpackage{com.kpro.algorithm}
\jdinherits{\jdtypesimple{Object}\jdinh \jdtypesimple{DistanceMetric}}
\JDtext{A distance metric that calculates distance based on 
 weighed union of a bit map interception}
\JDversion{240911.01}
\JDauthor{dimitryk}
\end{jdclassheader}
\begin{jdinheritancetable} \jdInhEntry{\jdtypesimple{double} getTotalDistance( \jdtypesimple{PolicyObject}, \jdtypesimple{PolicyObject} )}{DistanceMetric}
 \jdInhEntry{\jdtypesimple{Object} clone(  )}{Object}
 \jdInhEntry{\jdtypesimple{boolean} equals( \jdtypesimple{Object} )}{Object}
 \jdInhEntry{\jdtypesimple{void} finalize(  )}{Object}
 \jdInhEntry{\jdtypesimple{Class} getClass(  )}{Object}
 \jdInhEntry{\jdtypesimple{int} hashCode(  )}{Object}
 \jdInhEntry{\jdtypesimple{void} notify(  )}{Object}
 \jdInhEntry{\jdtypesimple{void} notifyAll(  )}{Object}
 \jdInhEntry{\jdtypesimple{String} toString(  )}{Object}
 \jdInhEntry{\jdtypesimple{void} wait( \jdtypesimple{long} )}{Object}
 \jdInhEntry{\jdtypesimple{void} wait( \jdtypesimple{long}, \jdtypesimple{int} )}{Object}
 \jdInhEntry{\jdtypesimple{void} wait(  )}{Object}
\end{jdinheritancetable}
\begin{jdmethod}{getTotalDistance}
\jdpublic 
\jdtype{\jdtypesimple{double}}
\JDpara{\jdtypesimple{PolicyObject}}{a}{the 1st policy object}
\JDpara{\jdtypesimple{PolicyObject}}{b}{the second policy object}
\JDtext{Initializes weights and returns the distance between two PolicyObjects.}
\JDreturn{the distance between the two policy objects}
\end{jdmethod}
\end{jdclass}

\include{Appendix/javadoc/com.kpro.algorithmbitmapDistanceWisOne}
\include{Appendix/javadoc/com.kpro.algorithmbitmapDistance}
\begin{jdclass}[class]{ConclusionAlgorithm}
\begin{jdclassheader}

\jdabstract \jdpublic 
\jdpackage{com.kpro.algorithm}
\jdinherits{\jdtypesimple{Object}}
\JDtext{abstract class for all conclusion classes (they take the new policy and 
 a reduction of the history versus new policy), and return an Action.
 May extend ReductionAlgorithm in the future (return a modified np, instead of
 an action).
 May be used instead of a ReductionAlgorithm. call with:
 Action a = (new ConclusionAlgorithm()).conclude(newpol,theIO.getPDB());}
\JDauthor{ngerstle}
\JDversion{29.09.11.1}
\end{jdclassheader}
\begin{jdinheritancetable} \jdInhEntry{\jdtypesimple{Object} clone(  )}{Object}
 \jdInhEntry{\jdtypesimple{boolean} equals( \jdtypesimple{Object} )}{Object}
 \jdInhEntry{\jdtypesimple{void} finalize(  )}{Object}
 \jdInhEntry{\jdtypesimple{Class} getClass(  )}{Object}
 \jdInhEntry{\jdtypesimple{int} hashCode(  )}{Object}
 \jdInhEntry{\jdtypesimple{void} notify(  )}{Object}
 \jdInhEntry{\jdtypesimple{void} notifyAll(  )}{Object}
 \jdInhEntry{\jdtypesimple{String} toString(  )}{Object}
 \jdInhEntry{\jdtypesimple{void} wait( \jdtypesimple{long} )}{Object}
 \jdInhEntry{\jdtypesimple{void} wait( \jdtypesimple{long}, \jdtypesimple{int} )}{Object}
 \jdInhEntry{\jdtypesimple{void} wait(  )}{Object}
\end{jdinheritancetable}
\begin{jdconstructor}
\jdpublic 
\JDpara{\jdtypesimple{DistanceMetric}}{dm}{}
\JDpara{\jdtypearray{String}{\lbrack{}\rbrack{}}}{extraArgs}{}
\JDtext{ConclustionAlgorithm constructor}
\end{jdconstructor}
\begin{jdmethod}{conclude}
\jdpublic \jdabstract 
\jdtype{\jdtypesimple{Action}}
\JDpara{\jdtypesimple{PolicyObject}}{np}{the new policy}
\JDpara{\jdtypesimple{Iterable}}{knearestns}{a set of relevant policies}
\JDtext{Provides an action recommendation for np based on the given set of objects}
\JDreturn{a recommended Action}
\end{jdmethod}
\end{jdclass}

\begin{jdclass}[class]{Conclusion\_Simple}
\begin{jdclassheader}

\jdpublic 
\jdpackage{com.kpro.algorithm}
\jdinherits{\jdtypesimple{Object}\jdinh \jdtypesimple{ConclusionAlgorithm}}
\JDtext{a very simple conclusion class. result is based on the closest objects only, as determined by the sum of inverse distances
 of the accepted versus rejected policies. confidences is the ratio of sum inverse distances of the chosen decision, versus the sum
 of all inverse distances.}
\JDauthor{ngerstle}
\JDversion{29.09.11.1}
\end{jdclassheader}
\begin{jdinheritancetable} \jdInhEntry{\jdtypesimple{DistanceMetric} distanceMetric}{ConclusionAlgorithm}
 \jdInhEntry{\jdtypesimple{Action} conclude( \jdtypesimple{PolicyObject}, \jdtypesimple{Iterable} )}{ConclusionAlgorithm}
 \jdInhEntry{\jdtypesimple{Object} clone(  )}{Object}
 \jdInhEntry{\jdtypesimple{boolean} equals( \jdtypesimple{Object} )}{Object}
 \jdInhEntry{\jdtypesimple{void} finalize(  )}{Object}
 \jdInhEntry{\jdtypesimple{Class} getClass(  )}{Object}
 \jdInhEntry{\jdtypesimple{int} hashCode(  )}{Object}
 \jdInhEntry{\jdtypesimple{void} notify(  )}{Object}
 \jdInhEntry{\jdtypesimple{void} notifyAll(  )}{Object}
 \jdInhEntry{\jdtypesimple{String} toString(  )}{Object}
 \jdInhEntry{\jdtypesimple{void} wait( \jdtypesimple{long} )}{Object}
 \jdInhEntry{\jdtypesimple{void} wait( \jdtypesimple{long}, \jdtypesimple{int} )}{Object}
 \jdInhEntry{\jdtypesimple{void} wait(  )}{Object}
\end{jdinheritancetable}
\begin{jdconstructor}
\jdpublic 
\JDpara{\jdtypesimple{DistanceMetric}}{dm}{}
\JDpara{\jdtypearray{String}{\lbrack{}\rbrack{}}}{extraArgs}{}
\end{jdconstructor}
\begin{jdmethod}{conclude}
\jdpublic 
\jdtype{\jdtypesimple{Action}}
\JDpara{\jdtypesimple{PolicyObject}}{np}{the object under consideration}
\JDpara{\jdtypesimple{Iterable}}{releventSet}{the reduced set of neighbors}
\JDtext{makes a decision on the reduced set.
 This class creates two lists, one for accepted policies and one for rejected. Assuming there are policies in both 
 (easy decision otherwise), whether the policy is accepted or not will depend on the difference between the sum of inverse
 distances of the list items (excluding zero-distances), with the smaller sum indicating the more relevent decision.}
\JDauthor{ngerstle}
\JDreturn{an arraylist of \{Action a, double Confidence)}
\end{jdmethod}
\end{jdclass}

\begin{jdclass}[class]{LearnAlgorithm}
\begin{jdclassheader}

\jdabstract \jdpublic 
\jdpackage{com.kpro.algorithm}
\jdinherits{\jdtypesimple{Object}}
\JDtext{An abstract class covering all learning algorithms. The learning algorithm
 alters the weights configuration after examining the current database after
 the addition of a new policy.}
\JDversion{29.09.11}
\JDauthor{ngerstle}
\end{jdclassheader}
\begin{jdinheritancetable} \jdInhEntry{\jdtypesimple{Object} clone(  )}{Object}
 \jdInhEntry{\jdtypesimple{boolean} equals( \jdtypesimple{Object} )}{Object}
 \jdInhEntry{\jdtypesimple{void} finalize(  )}{Object}
 \jdInhEntry{\jdtypesimple{Class} getClass(  )}{Object}
 \jdInhEntry{\jdtypesimple{int} hashCode(  )}{Object}
 \jdInhEntry{\jdtypesimple{void} notify(  )}{Object}
 \jdInhEntry{\jdtypesimple{void} notifyAll(  )}{Object}
 \jdInhEntry{\jdtypesimple{String} toString(  )}{Object}
 \jdInhEntry{\jdtypesimple{void} wait( \jdtypesimple{long} )}{Object}
 \jdInhEntry{\jdtypesimple{void} wait( \jdtypesimple{long}, \jdtypesimple{int} )}{Object}
 \jdInhEntry{\jdtypesimple{void} wait(  )}{Object}
\end{jdinheritancetable}
\begin{jdconstructor}
\jdpublic 
\JDpara{\jdtypesimple{Properties}}{weightsConfig}{the weights configuration file}
\JDpara{\jdtypearray{String}{\lbrack{}\rbrack{}}}{extraArgs}{extra arguments defined in configuration file}
\JDtext{Constructor for a learning algorithm. accepts a weights configuration file.}
\end{jdconstructor}
\begin{jdmethod}{learn}
\jdpublic 
\jdtype{\jdtypesimple{void}}
\JDpara{\jdtypesimple{Gio}}{theIO}{}
\JDtext{runs the learning algorithm, and puts the results in the newWeight buffer in theIO(Gio)}
\JDauthor{ngerstle}
\end{jdmethod}
\end{jdclass}

\begin{jdclass}[class]{Learn\_Constant}
\begin{jdclassheader}

\jdpublic 
\jdpackage{com.kpro.algorithm}
\jdinherits{\jdtypesimple{Object}\jdinh \jdtypesimple{LearnAlgorithm}}
\JDtext{The simplest implementation of learnAlgorithm, this literally
 does nothing, and thus doesn't actually learn.}
\JDauthor{ngerstle}
\JDversion{29.09.11.1}
\end{jdclassheader}
\begin{jdinheritancetable} \jdInhEntry{\jdtypesimple{Properties} applyML( \jdtypesimple{Gio} )}{LearnAlgorithm}
 \jdInhEntry{\jdtypesimple{void} learn( \jdtypesimple{Gio} )}{LearnAlgorithm}
 \jdInhEntry{\jdtypesimple{Object} clone(  )}{Object}
 \jdInhEntry{\jdtypesimple{boolean} equals( \jdtypesimple{Object} )}{Object}
 \jdInhEntry{\jdtypesimple{void} finalize(  )}{Object}
 \jdInhEntry{\jdtypesimple{Class} getClass(  )}{Object}
 \jdInhEntry{\jdtypesimple{int} hashCode(  )}{Object}
 \jdInhEntry{\jdtypesimple{void} notify(  )}{Object}
 \jdInhEntry{\jdtypesimple{void} notifyAll(  )}{Object}
 \jdInhEntry{\jdtypesimple{String} toString(  )}{Object}
 \jdInhEntry{\jdtypesimple{void} wait( \jdtypesimple{long} )}{Object}
 \jdInhEntry{\jdtypesimple{void} wait( \jdtypesimple{long}, \jdtypesimple{int} )}{Object}
 \jdInhEntry{\jdtypesimple{void} wait(  )}{Object}
\end{jdinheritancetable}
\begin{jdconstructor}
\jdpublic 
\JDpara{\jdtypesimple{Properties}}{weightsConfig}{the weights configuration file}
\JDpara{\jdtypearray{String}{\lbrack{}\rbrack{}}}{extraArgs}{from the config file}
\JDtext{constructor}
\JDauthor{ngerstle, ernie}
\end{jdconstructor}
\end{jdclass}

\begin{jdclass}[class]{LearnAlgBasic}
\begin{jdclassheader}

\jdpublic 
\jdpackage{com.kpro.algorithm}
\jdinherits{\jdtypesimple{Object}\jdinh \jdtypesimple{LearnAlgorithm}}
\JDtext{A very slow learning algorithm that goes through every case in every
 PolicyObject in the whole database.}
\JDauthor{Nesha}
\end{jdclassheader}
\begin{jdinheritancetable} \jdInhEntry{\jdtypesimple{Properties} applyML( \jdtypesimple{Gio} )}{LearnAlgorithm}
 \jdInhEntry{\jdtypesimple{void} learn( \jdtypesimple{Gio} )}{LearnAlgorithm}
 \jdInhEntry{\jdtypesimple{Object} clone(  )}{Object}
 \jdInhEntry{\jdtypesimple{boolean} equals( \jdtypesimple{Object} )}{Object}
 \jdInhEntry{\jdtypesimple{void} finalize(  )}{Object}
 \jdInhEntry{\jdtypesimple{Class} getClass(  )}{Object}
 \jdInhEntry{\jdtypesimple{int} hashCode(  )}{Object}
 \jdInhEntry{\jdtypesimple{void} notify(  )}{Object}
 \jdInhEntry{\jdtypesimple{void} notifyAll(  )}{Object}
 \jdInhEntry{\jdtypesimple{String} toString(  )}{Object}
 \jdInhEntry{\jdtypesimple{void} wait( \jdtypesimple{long} )}{Object}
 \jdInhEntry{\jdtypesimple{void} wait( \jdtypesimple{long}, \jdtypesimple{int} )}{Object}
 \jdInhEntry{\jdtypesimple{void} wait(  )}{Object}
\end{jdinheritancetable}
\begin{jdconstructor}
\jdpublic 
\JDpara{\jdtypesimple{Properties}}{weightsConfig}{the weights}
\JDpara{\jdtypearray{String}{\lbrack{}\rbrack{}}}{extraArgs}{from the config}
\JDtext{constructor}
\end{jdconstructor}
\end{jdclass}

\include{Appendix/javadoc/com.kpro.algorithmLearnAlgStand}

\section{Datastorage}
\begin{jdclass}[class]{PolicyDatabase}
\begin{jdclassheader}

\jdabstract \jdpublic 
\jdpackage{com.kpro.datastorage}
\jdimplements{Iterable}
\jdinherits{\jdtypesimple{Object}}
\JDtext{This is abstract class for databases. All policy 
 databases must implement this interface. Auto-generated
 via eclipse from PDatabase.java, then made abstract.
 Does enforce singleton-ness here. need to do it in each subclass.}
\JDauthor{ngerstle}
\JDversion{29.09.11.1}
\end{jdclassheader}
\begin{jdinheritancetable} \jdInhEntry{\jdtypesimple{Object} clone(  )}{Object}
 \jdInhEntry{\jdtypesimple{boolean} equals( \jdtypesimple{Object} )}{Object}
 \jdInhEntry{\jdtypesimple{void} finalize(  )}{Object}
 \jdInhEntry{\jdtypesimple{Class} getClass(  )}{Object}
 \jdInhEntry{\jdtypesimple{int} hashCode(  )}{Object}
 \jdInhEntry{\jdtypesimple{void} notify(  )}{Object}
 \jdInhEntry{\jdtypesimple{void} notifyAll(  )}{Object}
 \jdInhEntry{\jdtypesimple{String} toString(  )}{Object}
 \jdInhEntry{\jdtypesimple{void} wait( \jdtypesimple{long} )}{Object}
 \jdInhEntry{\jdtypesimple{void} wait( \jdtypesimple{long}, \jdtypesimple{int} )}{Object}
 \jdInhEntry{\jdtypesimple{void} wait(  )}{Object}
\end{jdinheritancetable}
\begin{jdmethod}{loadDB}
\jdpublic 
\jdtype{\jdtypesimple{void}}
\JDtext{Loads database from a file dLoc}
\JDauthor{ngerstle}
\end{jdmethod}
\begin{jdmethod}{loadDB}
\jdpublic \jdabstract 
\jdtype{\jdtypesimple{void}}
\JDpara{\jdtypesimple{String}}{dLoc}{the inlocation of the database file on disk}
\JDtext{Loads database from a file dLoc
 we have this public just in case we want to be able to load
 a db from a file other than where we plan to store it.}
\JDauthor{ngerstle}
\end{jdmethod}
\begin{jdmethod}{addPolicy}
\jdpublic 
\jdtype{\jdtypesimple{void}}
\JDpara{\jdtypesimple{PolicyObject}}{n}{the policy to add to the database}
\JDtext{Adds a PolicyObject to the database}
\end{jdmethod}
\begin{jdmethod}{iterator}
\jdpublic 
\jdtype{\jdtypesimple{Iterator}}
\JDtext{provides an iterator over the database's PolicyObjects}
\JDreturn{an iterator over the internal collection}
\end{jdmethod}
\begin{jdmethod}{closeDB}
\jdpublic 
\jdtype{\jdtypesimple{void}}
\JDtext{calls closeDB(PolicyDatabase.outlocation).
 needed due to lack of a good destructor system, as far as java goes (according to google)}
\end{jdmethod}
\begin{jdmethod}{closeDB}
\jdpublic \jdabstract 
\jdtype{\jdtypesimple{void}}
\JDpara{\jdtypesimple{String}}{dLoc}{the inlocation to save the data}
\JDtext{should implement writing the information contained by the PolicyDatabase to a file/disk}
\JDauthor{ngerstle}
\end{jdmethod}
\begin{jdmethod}{getDomain}
\jdpublic \jdabstract 
\jdtype{\jdtypesimple{ArrayList}}
\JDpara{\jdtypesimple{String}}{domain}{the domain to look for}
\JDtext{This class should return all policys for the given domain.}
\JDauthor{ngerstle}
\end{jdmethod}
\begin{jdmethod}{toString}
\jdpublic 
\jdtype{\jdtypesimple{String}}
\JDtext{Standard toString method}
\JDauthor{ulfnore}
\end{jdmethod}
\end{jdclass}

\begin{jdclass}[class]{PDatabase}
\begin{jdclassheader}

\jdpublic 
\jdpackage{com.kpro.datastorage}
\jdimplements{Serializable}
\jdinherits{\jdtypesimple{Object}\jdinh \jdtypesimple{PolicyDatabase}}
\JDtext{This singleton class will store all past P3P/contexts instances in a hashmap, and is saved on disk via java.io.serializable.}
\JDauthor{Nicholas Gerstle, Henrik Knutsen, Aman Kaur}
\JDversion{1.0}
\end{jdclassheader}
\begin{jdinheritancetable} \jdInhEntry{\jdtypesimple{PolicyDatabase} i}{PolicyDatabase}
 \jdInhEntry{\jdtypesimple{Collection} idb}{PolicyDatabase}
 \jdInhEntry{\jdtypesimple{String} inLocation}{PolicyDatabase}
 \jdInhEntry{\jdtypesimple{String} outLocation}{PolicyDatabase}
 \jdInhEntry{\jdtypesimple{void} addPolicy( \jdtypesimple{PolicyObject} )}{PolicyDatabase}
 \jdInhEntry{\jdtypesimple{void} closeDB(  )}{PolicyDatabase}
 \jdInhEntry{\jdtypesimple{void} closeDB( \jdtypesimple{String} )}{PolicyDatabase}
 \jdInhEntry{\jdtypesimple{ArrayList} getDomain( \jdtypesimple{String} )}{PolicyDatabase}
 \jdInhEntry{\jdtypesimple{Iterator} iterator(  )}{PolicyDatabase}
 \jdInhEntry{\jdtypesimple{void} loadDB(  )}{PolicyDatabase}
 \jdInhEntry{\jdtypesimple{void} loadDB( \jdtypesimple{String} )}{PolicyDatabase}
 \jdInhEntry{\jdtypesimple{String} toString(  )}{PolicyDatabase}
 \jdInhEntry{\jdtypesimple{Object} clone(  )}{Object}
 \jdInhEntry{\jdtypesimple{boolean} equals( \jdtypesimple{Object} )}{Object}
 \jdInhEntry{\jdtypesimple{void} finalize(  )}{Object}
 \jdInhEntry{\jdtypesimple{Class} getClass(  )}{Object}
 \jdInhEntry{\jdtypesimple{int} hashCode(  )}{Object}
 \jdInhEntry{\jdtypesimple{void} notify(  )}{Object}
 \jdInhEntry{\jdtypesimple{void} notifyAll(  )}{Object}
 \jdInhEntry{\jdtypesimple{String} toString(  )}{Object}
 \jdInhEntry{\jdtypesimple{void} wait( \jdtypesimple{long} )}{Object}
 \jdInhEntry{\jdtypesimple{void} wait( \jdtypesimple{long}, \jdtypesimple{int} )}{Object}
 \jdInhEntry{\jdtypesimple{void} wait(  )}{Object}
\end{jdinheritancetable}
\begin{jdmethod}{loadDB}
\jdpublic 
\jdtype{\jdtypesimple{void}}
\JDpara{\jdtypesimple{String}}{dLoc}{the location of the database file on disk}
\JDtext{See PolicyDatabase.java. implements loading/closing the db via a serialize PDatabase object.
 WILL overwrite existing database if called more than once.}
\JDauthor{ngerstle}
\JDsee{PolicyDatabase\#{}closeDB()}
\end{jdmethod}
\begin{jdmethod}{closeDB}
\jdpublic 
\jdtype{\jdtypesimple{void}}
\JDpara{\jdtypesimple{String}}{dLoc}{}
\JDtext{See PolicyDatabase.java. implements loading/closing the db via a serialize PDatabase object.}
\JDsee{PolicyDatabase\#{}closeDB()}
\end{jdmethod}
\begin{jdmethod}{getDomain}
\jdpublic 
\jdtype{\jdtypesimple{ArrayList}}
\JDpara{\jdtypesimple{String}}{domain}{}
\JDtext{returns a list of all policies from a given domain.}
\JDreturn{an arraylist of policies from given domain}
\JDauthor{ngerstle}
\JDsee{PolicyDatabase\#{}getDomain()}
\end{jdmethod}
\begin{jdmethod}{getInstance}
\jdpublic \jdstatic 
\jdtype{\jdtypesimple{PolicyDatabase}}
\JDpara{\jdtypesimple{String}}{inloc}{where the database should be loaded from (on disc)}
\JDpara{\jdtypesimple{String}}{outloc}{where the database should be saved to (on disc)}
\JDtext{returns an instance of the database.}
\end{jdmethod}
\end{jdclass}

\begin{jdclass}[class]{NRCouchdb}
\begin{jdclassheader}

\jdpublic 
\jdpackage{com.kpro.datastorage}
\jdinherits{\jdtypesimple{Object}\jdinh \jdtypesimple{NetworkR}}
\JDtext{The network resource class for working with a couchDB server cross network at the specified location.}
\JDauthor{ngerstle}
\JDversion{17.10.11.1}
\end{jdclassheader}
\begin{jdinheritancetable} \jdInhEntry{\jdtypesimple{void} disconnect(  )}{NetworkR}
 \jdInhEntry{\jdtypesimple{String} getInfo(  )}{NetworkR}
 \jdInhEntry{\jdtypesimple{void} parseNOptions( \jdtypesimple{String} )}{NetworkR}
 \jdInhEntry{\jdtypesimple{Action} reqAct( \jdtypesimple{PolicyObject} )}{NetworkR}
 \jdInhEntry{\jdtypesimple{void} saveObj( \jdtypesimple{PolicyObject} )}{NetworkR}
 \jdInhEntry{\jdtypesimple{Object} clone(  )}{Object}
 \jdInhEntry{\jdtypesimple{boolean} equals( \jdtypesimple{Object} )}{Object}
 \jdInhEntry{\jdtypesimple{void} finalize(  )}{Object}
 \jdInhEntry{\jdtypesimple{Class} getClass(  )}{Object}
 \jdInhEntry{\jdtypesimple{int} hashCode(  )}{Object}
 \jdInhEntry{\jdtypesimple{void} notify(  )}{Object}
 \jdInhEntry{\jdtypesimple{void} notifyAll(  )}{Object}
 \jdInhEntry{\jdtypesimple{String} toString(  )}{Object}
 \jdInhEntry{\jdtypesimple{void} wait( \jdtypesimple{long} )}{Object}
 \jdInhEntry{\jdtypesimple{void} wait( \jdtypesimple{long}, \jdtypesimple{int} )}{Object}
 \jdInhEntry{\jdtypesimple{void} wait(  )}{Object}
\end{jdinheritancetable}
\begin{jdfield}{type}
\jdpublic \jdfinal 
\jdtype{\jdtypesimple{String}}
\end{jdfield}
\begin{jdfield}{LVquerybase}
\jdpublic \jdfinal 
\jdtype{\jdtypesimple{String}}
\JDtext{Couchlight object- holds connection to database}
\end{jdfield}
\begin{jdconstructor}
\jdpublic 
\JDpara{\jdtypesimple{String}}{options}{}
\JDtext{The view, and the list to pass through when requesting: append jsonated object as key/value}
\end{jdconstructor}
\begin{jdmethod}{reqAct}
\jdpublic 
\jdtype{\jdtypesimple{Action}}
\JDpara{\jdtypesimple{PolicyObject}}{a}{}
\end{jdmethod}
\begin{jdmethod}{saveObj}
\jdpublic 
\jdtype{\jdtypesimple{void}}
\JDpara{\jdtypesimple{PolicyObject}}{a}{the PolicyObject to save}
\JDtext{Saves the policy object to the database}
\JDreturn{void}
\end{jdmethod}
\begin{jdmethod}{disconnect}
\jdpublic 
\jdtype{\jdtypesimple{void}}
\JDtext{closes/deletes any remaing resources}
\JDreturn{void}
\end{jdmethod}
\end{jdclass}

\begin{jdclass}[class]{NetworkR}
\begin{jdclassheader}

\jdabstract \jdpublic 
\jdpackage{com.kpro.datastorage}
\jdinherits{\jdtypesimple{Object}}
\JDtext{Network resources class. Abstract. See javadocs for what contains what. Should allow request for an action 
 (if local knowledge is insufficient), as well as saving to the community database.}
\JDauthor{ngerstle}
\JDversion{17.10.11.1}
\end{jdclassheader}
\begin{jdinheritancetable} \jdInhEntry{\jdtypesimple{Object} clone(  )}{Object}
 \jdInhEntry{\jdtypesimple{boolean} equals( \jdtypesimple{Object} )}{Object}
 \jdInhEntry{\jdtypesimple{void} finalize(  )}{Object}
 \jdInhEntry{\jdtypesimple{Class} getClass(  )}{Object}
 \jdInhEntry{\jdtypesimple{int} hashCode(  )}{Object}
 \jdInhEntry{\jdtypesimple{void} notify(  )}{Object}
 \jdInhEntry{\jdtypesimple{void} notifyAll(  )}{Object}
 \jdInhEntry{\jdtypesimple{String} toString(  )}{Object}
 \jdInhEntry{\jdtypesimple{void} wait( \jdtypesimple{long} )}{Object}
 \jdInhEntry{\jdtypesimple{void} wait( \jdtypesimple{long}, \jdtypesimple{int} )}{Object}
 \jdInhEntry{\jdtypesimple{void} wait(  )}{Object}
\end{jdinheritancetable}
\begin{jdconstructor}
\jdpublic 
\JDpara{\jdtypesimple{String}}{options}{}
\JDtext{Constructor- should accept as minimum the location of the networked database.}
\end{jdconstructor}
\begin{jdmethod}{reqAct}
\jdpublic \jdabstract 
\jdtype{\jdtypesimple{Action}}
\JDpara{\jdtypesimple{PolicyObject}}{a}{the PolicyObject to obtain a response for.}
\JDtext{accepts a PolicyObject, and returns the remote suggested action for it (the suggestion from the server).}
\JDreturn{the action to give the policy object.}
\end{jdmethod}
\begin{jdmethod}{saveObj}
\jdpublic \jdabstract 
\jdtype{\jdtypesimple{void}}
\JDpara{\jdtypesimple{PolicyObject}}{a}{the policy to upload remotely.}
\JDtext{sends a PolicyObject to the server. It needs to have complete Action attached.}
\end{jdmethod}
\begin{jdmethod}{disconnect}
\jdpublic \jdabstract 
\jdtype{\jdtypesimple{void}}
\JDtext{closes whatever resources were opened during initialization by the constructor}
\end{jdmethod}
\begin{jdmethod}{getInfo}
\jdpublic 
\jdtype{\jdtypesimple{String}}
\JDtext{returns the server type and location.}
\JDreturn{String "A "+ typeOfServer + " database located at "+ locationOfServer}
\end{jdmethod}
\end{jdclass}


\section{Dataobjects}
\begin{jdclass}[class]{PolicyObject}
\begin{jdclassheader}

\jdpublic 
\jdpackage{com.kpro.dataobjects}
\jdimplements{Serializable}
\jdimplements{Iterable}
\jdinherits{\jdtypesimple{Object}}
\JDtext{The PolicyObject class acts to hold all the relevent information for a given policy. This informations is broken into 'cases' (see the 'case' class- different items, indexed by datatype), the context (see the 'context' class- holds information that applies to whole policy), and action (see the 'action' class- holds the action taken).}
\JDauthor{ernie, ngerstle}
\end{jdclassheader}
\begin{jdinheritancetable} \jdInhEntry{\jdtypesimple{Object} clone(  )}{Object}
 \jdInhEntry{\jdtypesimple{boolean} equals( \jdtypesimple{Object} )}{Object}
 \jdInhEntry{\jdtypesimple{void} finalize(  )}{Object}
 \jdInhEntry{\jdtypesimple{Class} getClass(  )}{Object}
 \jdInhEntry{\jdtypesimple{int} hashCode(  )}{Object}
 \jdInhEntry{\jdtypesimple{void} notify(  )}{Object}
 \jdInhEntry{\jdtypesimple{void} notifyAll(  )}{Object}
 \jdInhEntry{\jdtypesimple{String} toString(  )}{Object}
 \jdInhEntry{\jdtypesimple{void} wait( \jdtypesimple{long} )}{Object}
 \jdInhEntry{\jdtypesimple{void} wait( \jdtypesimple{long}, \jdtypesimple{int} )}{Object}
 \jdInhEntry{\jdtypesimple{void} wait(  )}{Object}
\end{jdinheritancetable}
\begin{jdconstructor}
\jdpublic 
\JDtext{This is the constructor
 The constructor initializes the variables within the class when you make a new instance of it}
\JDauthor{ernie}
\end{jdconstructor}
\begin{jdmethod}{getContext}
\jdpublic 
\jdtype{\jdtypesimple{Context}}
\end{jdmethod}
\begin{jdmethod}{setContext}
\jdpublic 
\jdtype{\jdtypesimple{void}}
\JDpara{\jdtypesimple{Context}}{context}{}
\end{jdmethod}
\begin{jdmethod}{getAction}
\jdpublic 
\jdtype{\jdtypesimple{Action}}
\end{jdmethod}
\begin{jdmethod}{setAction}
\jdpublic 
\jdtype{\jdtypesimple{PolicyObject}}
\JDpara{\jdtypesimple{Action}}{action}{}
\end{jdmethod}
\begin{jdmethod}{addCase}
\jdpublic 
\jdtype{\jdtypesimple{void}}
\JDpara{\jdtypesimple{Case}}{c}{input Case}
\JDtext{Adds a case to the policy}
\JDauthor{ernie}
\end{jdmethod}
\begin{jdmethod}{addEntityData}
\jdpublic 
\jdtype{\jdtypesimple{void}}
\JDpara{\jdtypesimple{String}}{key}{input String}
\JDpara{\jdtypesimple{String}}{value}{input String}
\JDtext{Adds a data to the entity hashmap of the policy}
\JDauthor{ernie}
\end{jdmethod}
\begin{jdmethod}{getCase}
\jdpublic 
\jdtype{\jdtypesimple{Case}}
\JDpara{\jdtypesimple{int}}{i}{input int}
\JDtext{Returns a specific case for the policy}
\JDauthor{ernie}
\JDreturn{Case}
\end{jdmethod}
\begin{jdmethod}{getCases}
\jdpublic 
\jdtype{\jdtypesimple{ArrayList}}
\JDtext{Returns all cases for the policy}
\JDauthor{ernie}
\JDreturn{ArrayList<Case>}
\end{jdmethod}
\begin{jdmethod}{getEntities}
\jdpublic 
\jdtype{\jdtypesimple{HashMap}}
\JDtext{Returns the entity for the policy}
\JDauthor{ernie}
\JDreturn{HashMap<String, String>}
\end{jdmethod}
\begin{jdmethod}{getEntity}
\jdpublic 
\jdtype{\jdtypesimple{String}}
\JDpara{\jdtypesimple{String}}{key}{input String}
\JDtext{Returns the entity data for a specific key}
\JDauthor{ernie}
\end{jdmethod}
\begin{jdmethod}{getContextDomain}
\jdpublic 
\jdtype{\jdtypesimple{String}}
\JDtext{returns domain of policy (URL) as string.}
\JDreturn{the domain/url from which the policy came.}
\JDauthor{ngerstle}
\end{jdmethod}
\begin{jdmethod}{toString}
\jdpublic 
\jdtype{\jdtypesimple{String}}
\JDtext{This is based on the debug\_print}
\JDauthor{ulfnore}
\end{jdmethod}
\begin{jdmethod}{equalsCases}
\jdpublic 
\jdtype{\jdtypesimple{boolean}}
\JDpara{\jdtypesimple{PolicyObject}}{newpol}{the policy to compare 'this' to}
\JDtext{A simple true/false check to see if policies are identical- if all the strings inside them are, then the policies are.}
\JDreturn{boolean; true if getContextDomains are equal and getCases() are equal (two string comparisons) else false.}
\end{jdmethod}
\begin{jdmethod}{iterator}
\jdpublic 
\jdtype{\jdtypesimple{Iterator}}
\end{jdmethod}
\end{jdclass}

\begin{jdclass}[class]{Context}
\begin{jdclassheader}

\jdpublic 
\jdpackage{com.kpro.dataobjects}
\jdinherits{\jdtypesimple{Object}}
\JDtext{Holds domain (from p3p), time, and other contextual information that applies to an entire p3p policy.}
\JDauthor{ngerstle, iernie}
\end{jdclassheader}
\begin{jdinheritancetable} \jdInhEntry{\jdtypesimple{Object} clone(  )}{Object}
 \jdInhEntry{\jdtypesimple{boolean} equals( \jdtypesimple{Object} )}{Object}
 \jdInhEntry{\jdtypesimple{void} finalize(  )}{Object}
 \jdInhEntry{\jdtypesimple{Class} getClass(  )}{Object}
 \jdInhEntry{\jdtypesimple{int} hashCode(  )}{Object}
 \jdInhEntry{\jdtypesimple{void} notify(  )}{Object}
 \jdInhEntry{\jdtypesimple{void} notifyAll(  )}{Object}
 \jdInhEntry{\jdtypesimple{String} toString(  )}{Object}
 \jdInhEntry{\jdtypesimple{void} wait( \jdtypesimple{long} )}{Object}
 \jdInhEntry{\jdtypesimple{void} wait( \jdtypesimple{long}, \jdtypesimple{int} )}{Object}
 \jdInhEntry{\jdtypesimple{void} wait(  )}{Object}
\end{jdinheritancetable}
\begin{jdconstructor}
\jdpublic 
\JDpara{\jdtypesimple{Date}}{accessTime}{}
\JDpara{\jdtypesimple{Date}}{creationTime}{}
\JDpara{\jdtypesimple{String}}{domain}{}
\end{jdconstructor}
\begin{jdmethod}{getAccessTime}
\jdpublic 
\jdtype{\jdtypesimple{Date}}
\JDtext{Gets the date of when the object was last accessed}
\JDreturn{Date}
\end{jdmethod}
\begin{jdmethod}{setAccessTime}
\jdpublic 
\jdtype{\jdtypesimple{void}}
\JDpara{\jdtypesimple{Date}}{accessTime}{}
\JDtext{Sets the date of when the object was last accessed}
\end{jdmethod}
\begin{jdmethod}{getCreationTime}
\jdpublic 
\jdtype{\jdtypesimple{Date}}
\JDtext{Gets the date of when the object was created}
\JDreturn{Date}
\end{jdmethod}
\begin{jdmethod}{setCreationTime}
\jdpublic 
\jdtype{\jdtypesimple{void}}
\JDpara{\jdtypesimple{Date}}{creationTime}{}
\JDtext{Sets the date of when the object was created}
\end{jdmethod}
\begin{jdmethod}{getExpiryDate}
\jdpublic 
\jdtype{\jdtypesimple{Date}}
\JDtext{Gets the date of when the object is to expire}
\JDreturn{Date}
\end{jdmethod}
\begin{jdmethod}{setExpiryDate}
\jdpublic 
\jdtype{\jdtypesimple{void}}
\JDpara{\jdtypesimple{Date}}{expiryDate}{}
\JDtext{Sets the date of when the object is to expire}
\end{jdmethod}
\begin{jdmethod}{getDomain}
\jdpublic 
\jdtype{\jdtypesimple{String}}
\JDtext{Gets the specific domain for this context}
\JDreturn{String}
\end{jdmethod}
\begin{jdmethod}{setDomain}
\jdpublic 
\jdtype{\jdtypesimple{void}}
\JDpara{\jdtypesimple{String}}{domain}{}
\JDtext{Sets the specific domain for this contact}
\end{jdmethod}
\end{jdclass}

\begin{jdclass}[class]{Action}
\begin{jdclassheader}

\jdpublic 
\jdpackage{com.kpro.dataobjects}
\jdinherits{\jdtypesimple{Object}}
\JDtext{holds results of a algorithmic comparison- t/f on approve, with the nearest neighbor, as well as a string \&{} enum for exception, if it is one}
\end{jdclassheader}
\begin{jdinheritancetable} \jdInhEntry{\jdtypesimple{Object} clone(  )}{Object}
 \jdInhEntry{\jdtypesimple{boolean} equals( \jdtypesimple{Object} )}{Object}
 \jdInhEntry{\jdtypesimple{void} finalize(  )}{Object}
 \jdInhEntry{\jdtypesimple{Class} getClass(  )}{Object}
 \jdInhEntry{\jdtypesimple{int} hashCode(  )}{Object}
 \jdInhEntry{\jdtypesimple{void} notify(  )}{Object}
 \jdInhEntry{\jdtypesimple{void} notifyAll(  )}{Object}
 \jdInhEntry{\jdtypesimple{String} toString(  )}{Object}
 \jdInhEntry{\jdtypesimple{void} wait( \jdtypesimple{long} )}{Object}
 \jdInhEntry{\jdtypesimple{void} wait( \jdtypesimple{long}, \jdtypesimple{int} )}{Object}
 \jdInhEntry{\jdtypesimple{void} wait(  )}{Object}
\end{jdinheritancetable}
\begin{jdconstructor}
\jdpublic 
\end{jdconstructor}
\begin{jdconstructor}
\jdpublic 
\JDpara{\jdtypesimple{boolean}}{accept}{}
\JDpara{\jdtypesimple{ArrayList}}{domains}{}
\JDpara{\jdtypesimple{double}}{confidence}{}
\JDpara{\jdtypesimple{boolean}}{override}{}
\end{jdconstructor}
\begin{jdmethod}{getReason}
\jdpublic 
\jdtype{\jdtypesimple{ArrayList}}
\end{jdmethod}
\begin{jdmethod}{setReason}
\jdpublic 
\jdtype{\jdtypesimple{void}}
\JDpara{\jdtypesimple{ArrayList}}{reason}{}
\end{jdmethod}
\begin{jdmethod}{getAcceptedStr}
\jdpublic 
\jdtype{\jdtypesimple{String}}
\JDtext{converts the internal accept/reject values to a String}
\JDreturn{a boolean that can be sent to the user with a accept/reject}
\end{jdmethod}
\begin{jdmethod}{getAccepted}
\jdpublic 
\jdtype{\jdtypesimple{boolean}}
\JDtext{Returns true if the action was accepted, and false otherwise.}
\JDreturn{boolean}
\end{jdmethod}
\begin{jdmethod}{setAccepted}
\jdpublic 
\jdtype{\jdtypesimple{void}}
\JDpara{\jdtypesimple{boolean}}{accept}{}
\JDtext{Sets the accepted state of the action.}
\end{jdmethod}
\begin{jdmethod}{getReasons}
\jdpublic 
\jdtype{\jdtypesimple{ArrayList}}
\JDreturn{an arraylist that verbalizes why the policy was accepted or rejected}
\end{jdmethod}
\begin{jdmethod}{isOverridden}
\jdpublic 
\jdtype{\jdtypesimple{boolean}}
\JDtext{Returns true if the action is manually overridden.}
\JDreturn{boolean}
\end{jdmethod}
\begin{jdmethod}{setConfidence}
\jdpublic 
\jdtype{\jdtypesimple{void}}
\JDpara{\jdtypesimple{double}}{confidence}{}
\JDtext{Sets the confidence, with checks on the value: if confidence = abs(input) if 1>=input>=-1, else negative infinity}
\end{jdmethod}
\begin{jdmethod}{getConfidence}
\jdpublic 
\jdtype{\jdtypesimple{double}}
\end{jdmethod}
\begin{jdmethod}{parse}
\jdpublic 
\jdtype{\jdtypesimple{Action}}
\JDpara{\jdtypesimple{String}}{optionValue}{the option string- see above. must have 3 commas, no spaces}
\JDtext{Parse a comma-seperated string into an Action. The string needs to have four comma-seperated tokens (thus three commas) and no spaces. The format is accept,domains,confidence,override where accept is 'accept' if accept==true, or anything else if accept!=true; domains is a semi-colon seperated string list of domains (no commas, no spaces, etc), eg 'www.google.com;www.yahoo.com;domain3;domain4' ; confidence is a double that is the confidence (parsed by parseDouble), and override is a boolean (parsed by parseBoolean).}
\JDreturn{an Action parsed from above}
\end{jdmethod}
\begin{jdmethod}{setOverride}
\jdpublic 
\jdtype{\jdtypesimple{Action}}
\JDpara{\jdtypesimple{boolean}}{b}{}
\end{jdmethod}
\begin{jdmethod}{toString}
\jdpublic 
\jdtype{\jdtypesimple{String}}
\JDtext{Overriden toString. format: ('Accepted.'|'Rejected.')('Override.'|'No Override.')("Confidences: %f")\lbrack{}"reasonDomain: \lbrack{}" (" string,")+\rbrack{}.}
\JDreturn{fancy string version see above. eg "Accepted. Override. Confidence: 0.5 reasonDomain: \lbrack{} google.com"}
\end{jdmethod}
\end{jdclass}

\begin{jdclass}[class]{Case}
\begin{jdclassheader}

\jdpublic 
\jdpackage{com.kpro.dataobjects}
\jdimplements{Comparable}
\jdinherits{\jdtypesimple{Object}}
\JDtext{A class that contains a single datatype from a PolicyObject.}
\JDauthor{ernie}
\end{jdclassheader}
\begin{jdinheritancetable} \jdInhEntry{\jdtypesimple{Object} clone(  )}{Object}
 \jdInhEntry{\jdtypesimple{boolean} equals( \jdtypesimple{Object} )}{Object}
 \jdInhEntry{\jdtypesimple{void} finalize(  )}{Object}
 \jdInhEntry{\jdtypesimple{Class} getClass(  )}{Object}
 \jdInhEntry{\jdtypesimple{int} hashCode(  )}{Object}
 \jdInhEntry{\jdtypesimple{void} notify(  )}{Object}
 \jdInhEntry{\jdtypesimple{void} notifyAll(  )}{Object}
 \jdInhEntry{\jdtypesimple{String} toString(  )}{Object}
 \jdInhEntry{\jdtypesimple{void} wait( \jdtypesimple{long} )}{Object}
 \jdInhEntry{\jdtypesimple{void} wait( \jdtypesimple{long}, \jdtypesimple{int} )}{Object}
 \jdInhEntry{\jdtypesimple{void} wait(  )}{Object}
\end{jdinheritancetable}
\begin{jdconstructor}
\jdpublic 
\end{jdconstructor}
\begin{jdconstructor}
\jdpublic 
\JDpara{\jdtypesimple{ArrayList}}{purpose}{}
\JDpara{\jdtypesimple{ArrayList}}{retention}{}
\JDpara{\jdtypesimple{ArrayList}}{recipients}{}
\JDpara{\jdtypesimple{ArrayList}}{categories}{}
\JDpara{\jdtypesimple{String}}{datatype}{}
\end{jdconstructor}
\begin{jdmethod}{addPurpose}
\jdpublic 
\jdtype{\jdtypesimple{void}}
\JDpara{\jdtypesimple{Purpose}}{p}{}
\JDtext{Adds a purpose to the case}
\end{jdmethod}
\begin{jdmethod}{addRetention}
\jdpublic 
\jdtype{\jdtypesimple{void}}
\JDpara{\jdtypesimple{Retention}}{t}{}
\JDtext{Adds a retention to the case}
\end{jdmethod}
\begin{jdmethod}{addRecipient}
\jdpublic 
\jdtype{\jdtypesimple{void}}
\JDpara{\jdtypesimple{Recipient}}{r}{}
\JDtext{Adds a recipient to the case}
\end{jdmethod}
\begin{jdmethod}{addCategory}
\jdpublic 
\jdtype{\jdtypesimple{void}}
\JDpara{\jdtypesimple{Category}}{c}{}
\JDtext{Adds a category to the case}
\end{jdmethod}
\begin{jdmethod}{setDataType}
\jdpublic 
\jdtype{\jdtypesimple{void}}
\JDpara{\jdtypesimple{String}}{s}{}
\JDtext{Sets the datatype to the case}
\end{jdmethod}
\begin{jdmethod}{getPurposes}
\jdpublic 
\jdtype{\jdtypesimple{ArrayList}}
\JDtext{Gets the purposes}
\JDreturn{ArrayList<Purpose>}
\end{jdmethod}
\begin{jdmethod}{getPurpose}
\jdpublic 
\jdtype{\jdtypesimple{Purpose}}
\JDpara{\jdtypesimple{int}}{i}{}
\JDtext{Returns the ith purpose}
\JDreturn{Purpose}
\end{jdmethod}
\begin{jdmethod}{getRetentions}
\jdpublic 
\jdtype{\jdtypesimple{ArrayList}}
\JDtext{Gets the retentions}
\JDreturn{ArrayList<Retention>}
\end{jdmethod}
\begin{jdmethod}{getRetention}
\jdpublic 
\jdtype{\jdtypesimple{Retention}}
\JDpara{\jdtypesimple{int}}{i}{}
\JDtext{Returns the ith retention}
\JDreturn{Retention}
\end{jdmethod}
\begin{jdmethod}{getRecipients}
\jdpublic 
\jdtype{\jdtypesimple{ArrayList}}
\JDtext{Gets the recipients}
\JDreturn{ArrayList<Recipient>}
\end{jdmethod}
\begin{jdmethod}{getRecipient}
\jdpublic 
\jdtype{\jdtypesimple{Recipient}}
\JDpara{\jdtypesimple{int}}{i}{}
\JDtext{Returns the ith recipient}
\JDreturn{Recipient}
\end{jdmethod}
\begin{jdmethod}{getCategories}
\jdpublic 
\jdtype{\jdtypesimple{ArrayList}}
\JDtext{Gets the categories}
\JDreturn{ArrayList<Category>}
\end{jdmethod}
\begin{jdmethod}{getCategory}
\jdpublic 
\jdtype{\jdtypesimple{Category}}
\JDpara{\jdtypesimple{int}}{i}{}
\JDtext{Returns the ith category}
\JDreturn{Category}
\end{jdmethod}
\begin{jdmethod}{getDataType}
\jdpublic 
\jdtype{\jdtypesimple{String}}
\JDtext{Returns the datatype}
\JDreturn{String}
\end{jdmethod}
\begin{jdmethod}{toString}
\jdpublic 
\jdtype{\jdtypesimple{String}}
\JDtext{Based on debug.print}
\JDauthor{ulfnore}
\end{jdmethod}
\begin{jdmethod}{compareTo}
\jdpublic 
\jdtype{\jdtypesimple{int}}
\JDpara{\jdtypesimple{Object}}{o}{}
\JDtext{to allow comparision of cases, primarily for white/blacklisting.}
\JDreturn{-1 if this > other, 0 if equal, else 1}
\JDauthor{ngerstle}
\end{jdmethod}
\end{jdclass}

\begin{jdclass}[class]{Retention}
\begin{jdclassheader}

\jdpublic \jdfinal 
\jdpackage{com.kpro.dataobjects}
\jdinherits{\jdtypesimple{Object}\jdinh \jdtypesimple{Enum}}
\JDtext{The retentions that a case can contain. See P3P specs for more info.}
\JDauthor{ernie}
\end{jdclassheader}
\begin{jdinheritancetable} \jdInhEntry{\jdtypesimple{Object} clone(  )}{Enum}
 \jdInhEntry{\jdtypesimple{int} compareTo( \jdtypesimple{Enum} )}{Enum}
 \jdInhEntry{\jdtypesimple{int} compareTo( \jdtypesimple{Object} )}{Enum}
 \jdInhEntry{\jdtypesimple{boolean} equals( \jdtypesimple{Object} )}{Enum}
 \jdInhEntry{\jdtypesimple{void} finalize(  )}{Enum}
 \jdInhEntry{\jdtypesimple{Class} getDeclaringClass(  )}{Enum}
 \jdInhEntry{\jdtypesimple{int} hashCode(  )}{Enum}
 \jdInhEntry{\jdtypesimple{String} name(  )}{Enum}
 \jdInhEntry{\jdtypesimple{int} ordinal(  )}{Enum}
 \jdInhEntry{\jdtypesimple{String} toString(  )}{Enum}
 \jdInhEntry{\jdtypesimple{Enum} valueOf( \jdtypesimple{Class}, \jdtypesimple{String} )}{Enum}
 \jdInhEntry{\jdtypesimple{Object} clone(  )}{Object}
 \jdInhEntry{\jdtypesimple{boolean} equals( \jdtypesimple{Object} )}{Object}
 \jdInhEntry{\jdtypesimple{void} finalize(  )}{Object}
 \jdInhEntry{\jdtypesimple{Class} getClass(  )}{Object}
 \jdInhEntry{\jdtypesimple{int} hashCode(  )}{Object}
 \jdInhEntry{\jdtypesimple{void} notify(  )}{Object}
 \jdInhEntry{\jdtypesimple{void} notifyAll(  )}{Object}
 \jdInhEntry{\jdtypesimple{String} toString(  )}{Object}
 \jdInhEntry{\jdtypesimple{void} wait( \jdtypesimple{long} )}{Object}
 \jdInhEntry{\jdtypesimple{void} wait( \jdtypesimple{long}, \jdtypesimple{int} )}{Object}
 \jdInhEntry{\jdtypesimple{void} wait(  )}{Object}
\end{jdinheritancetable}
\begin{jdfield}{NO\_RETENTION}
\jdpublic \jdfinal \jdstatic 
\jdtype{\jdtypesimple{Retention}}
\end{jdfield}
\begin{jdfield}{STATED\_PURPOSE}
\jdpublic \jdfinal \jdstatic 
\jdtype{\jdtypesimple{Retention}}
\end{jdfield}
\begin{jdfield}{LEGAL\_REQUIREMENT}
\jdpublic \jdfinal \jdstatic 
\jdtype{\jdtypesimple{Retention}}
\end{jdfield}
\begin{jdfield}{BUSINESS\_PRACTICES}
\jdpublic \jdfinal \jdstatic 
\jdtype{\jdtypesimple{Retention}}
\end{jdfield}
\begin{jdfield}{INDEFINITELY}
\jdpublic \jdfinal \jdstatic 
\jdtype{\jdtypesimple{Retention}}
\end{jdfield}
\begin{jdmethod}{values}
\jdpublic \jdstatic 
\jdtype{\jdtypearray{Retention}{\lbrack{}\rbrack{}}}
\end{jdmethod}
\begin{jdmethod}{valueOf}
\jdpublic \jdstatic 
\jdtype{\jdtypesimple{Retention}}
\JDpara{\jdtypesimple{String}}{name}{}
\end{jdmethod}
\end{jdclass}

\begin{jdclass}[class]{Recipient}
\begin{jdclassheader}

\jdpublic \jdfinal 
\jdpackage{com.kpro.dataobjects}
\jdinherits{\jdtypesimple{Object}\jdinh \jdtypesimple{Enum}}
\JDtext{The recipients that a case can contain. See P3P specs for more info.}
\JDauthor{ernie}
\end{jdclassheader}
\begin{jdinheritancetable} \jdInhEntry{\jdtypesimple{Object} clone(  )}{Enum}
 \jdInhEntry{\jdtypesimple{int} compareTo( \jdtypesimple{Enum} )}{Enum}
 \jdInhEntry{\jdtypesimple{int} compareTo( \jdtypesimple{Object} )}{Enum}
 \jdInhEntry{\jdtypesimple{boolean} equals( \jdtypesimple{Object} )}{Enum}
 \jdInhEntry{\jdtypesimple{void} finalize(  )}{Enum}
 \jdInhEntry{\jdtypesimple{Class} getDeclaringClass(  )}{Enum}
 \jdInhEntry{\jdtypesimple{int} hashCode(  )}{Enum}
 \jdInhEntry{\jdtypesimple{String} name(  )}{Enum}
 \jdInhEntry{\jdtypesimple{int} ordinal(  )}{Enum}
 \jdInhEntry{\jdtypesimple{String} toString(  )}{Enum}
 \jdInhEntry{\jdtypesimple{Enum} valueOf( \jdtypesimple{Class}, \jdtypesimple{String} )}{Enum}
 \jdInhEntry{\jdtypesimple{Object} clone(  )}{Object}
 \jdInhEntry{\jdtypesimple{boolean} equals( \jdtypesimple{Object} )}{Object}
 \jdInhEntry{\jdtypesimple{void} finalize(  )}{Object}
 \jdInhEntry{\jdtypesimple{Class} getClass(  )}{Object}
 \jdInhEntry{\jdtypesimple{int} hashCode(  )}{Object}
 \jdInhEntry{\jdtypesimple{void} notify(  )}{Object}
 \jdInhEntry{\jdtypesimple{void} notifyAll(  )}{Object}
 \jdInhEntry{\jdtypesimple{String} toString(  )}{Object}
 \jdInhEntry{\jdtypesimple{void} wait( \jdtypesimple{long} )}{Object}
 \jdInhEntry{\jdtypesimple{void} wait( \jdtypesimple{long}, \jdtypesimple{int} )}{Object}
 \jdInhEntry{\jdtypesimple{void} wait(  )}{Object}
\end{jdinheritancetable}
\begin{jdfield}{OURS}
\jdpublic \jdfinal \jdstatic 
\jdtype{\jdtypesimple{Recipient}}
\end{jdfield}
\begin{jdfield}{DELIVERY}
\jdpublic \jdfinal \jdstatic 
\jdtype{\jdtypesimple{Recipient}}
\end{jdfield}
\begin{jdfield}{SAME}
\jdpublic \jdfinal \jdstatic 
\jdtype{\jdtypesimple{Recipient}}
\end{jdfield}
\begin{jdfield}{OTHER\_RECIPIENT}
\jdpublic \jdfinal \jdstatic 
\jdtype{\jdtypesimple{Recipient}}
\end{jdfield}
\begin{jdfield}{UNRELATED}
\jdpublic \jdfinal \jdstatic 
\jdtype{\jdtypesimple{Recipient}}
\end{jdfield}
\begin{jdfield}{PUBLIC}
\jdpublic \jdfinal \jdstatic 
\jdtype{\jdtypesimple{Recipient}}
\end{jdfield}
\begin{jdmethod}{values}
\jdpublic \jdstatic 
\jdtype{\jdtypearray{Recipient}{\lbrack{}\rbrack{}}}
\end{jdmethod}
\begin{jdmethod}{valueOf}
\jdpublic \jdstatic 
\jdtype{\jdtypesimple{Recipient}}
\JDpara{\jdtypesimple{String}}{name}{}
\end{jdmethod}
\begin{jdmethod}{setOptional}
\jdpublic 
\jdtype{\jdtypesimple{void}}
\end{jdmethod}
\begin{jdmethod}{isOptional}
\jdpublic 
\jdtype{\jdtypesimple{boolean}}
\end{jdmethod}
\end{jdclass}

\begin{jdclass}[class]{Purpose}
\begin{jdclassheader}

\jdpublic \jdfinal 
\jdpackage{com.kpro.dataobjects}
\jdinherits{\jdtypesimple{Object}\jdinh \jdtypesimple{Enum}}
\JDtext{The purposes that a case can contain. See P3P specs for more info.}
\JDauthor{ernie}
\end{jdclassheader}
\begin{jdinheritancetable} \jdInhEntry{\jdtypesimple{Object} clone(  )}{Enum}
 \jdInhEntry{\jdtypesimple{int} compareTo( \jdtypesimple{Enum} )}{Enum}
 \jdInhEntry{\jdtypesimple{int} compareTo( \jdtypesimple{Object} )}{Enum}
 \jdInhEntry{\jdtypesimple{boolean} equals( \jdtypesimple{Object} )}{Enum}
 \jdInhEntry{\jdtypesimple{void} finalize(  )}{Enum}
 \jdInhEntry{\jdtypesimple{Class} getDeclaringClass(  )}{Enum}
 \jdInhEntry{\jdtypesimple{int} hashCode(  )}{Enum}
 \jdInhEntry{\jdtypesimple{String} name(  )}{Enum}
 \jdInhEntry{\jdtypesimple{int} ordinal(  )}{Enum}
 \jdInhEntry{\jdtypesimple{String} toString(  )}{Enum}
 \jdInhEntry{\jdtypesimple{Enum} valueOf( \jdtypesimple{Class}, \jdtypesimple{String} )}{Enum}
 \jdInhEntry{\jdtypesimple{Object} clone(  )}{Object}
 \jdInhEntry{\jdtypesimple{boolean} equals( \jdtypesimple{Object} )}{Object}
 \jdInhEntry{\jdtypesimple{void} finalize(  )}{Object}
 \jdInhEntry{\jdtypesimple{Class} getClass(  )}{Object}
 \jdInhEntry{\jdtypesimple{int} hashCode(  )}{Object}
 \jdInhEntry{\jdtypesimple{void} notify(  )}{Object}
 \jdInhEntry{\jdtypesimple{void} notifyAll(  )}{Object}
 \jdInhEntry{\jdtypesimple{String} toString(  )}{Object}
 \jdInhEntry{\jdtypesimple{void} wait( \jdtypesimple{long} )}{Object}
 \jdInhEntry{\jdtypesimple{void} wait( \jdtypesimple{long}, \jdtypesimple{int} )}{Object}
 \jdInhEntry{\jdtypesimple{void} wait(  )}{Object}
\end{jdinheritancetable}
\begin{jdfield}{CURRENT}
\jdpublic \jdfinal \jdstatic 
\jdtype{\jdtypesimple{Purpose}}
\end{jdfield}
\begin{jdfield}{ADMIN}
\jdpublic \jdfinal \jdstatic 
\jdtype{\jdtypesimple{Purpose}}
\end{jdfield}
\begin{jdfield}{DEVELOP}
\jdpublic \jdfinal \jdstatic 
\jdtype{\jdtypesimple{Purpose}}
\end{jdfield}
\begin{jdfield}{TAILORING}
\jdpublic \jdfinal \jdstatic 
\jdtype{\jdtypesimple{Purpose}}
\end{jdfield}
\begin{jdfield}{PSEUDO\_ANALYSIS}
\jdpublic \jdfinal \jdstatic 
\jdtype{\jdtypesimple{Purpose}}
\end{jdfield}
\begin{jdfield}{PSEUDO\_DECISION}
\jdpublic \jdfinal \jdstatic 
\jdtype{\jdtypesimple{Purpose}}
\end{jdfield}
\begin{jdfield}{INDIVIDUAL\_ANALYSIS}
\jdpublic \jdfinal \jdstatic 
\jdtype{\jdtypesimple{Purpose}}
\end{jdfield}
\begin{jdfield}{INDIVIDUAL\_DECISION}
\jdpublic \jdfinal \jdstatic 
\jdtype{\jdtypesimple{Purpose}}
\end{jdfield}
\begin{jdfield}{CONTACT}
\jdpublic \jdfinal \jdstatic 
\jdtype{\jdtypesimple{Purpose}}
\end{jdfield}
\begin{jdfield}{HISTORICAL}
\jdpublic \jdfinal \jdstatic 
\jdtype{\jdtypesimple{Purpose}}
\end{jdfield}
\begin{jdfield}{TELEMARKETING}
\jdpublic \jdfinal \jdstatic 
\jdtype{\jdtypesimple{Purpose}}
\end{jdfield}
\begin{jdfield}{OTHER\_PURPOSE}
\jdpublic \jdfinal \jdstatic 
\jdtype{\jdtypesimple{Purpose}}
\end{jdfield}
\begin{jdfield}{CUSTOMIZATION}
\jdpublic \jdfinal \jdstatic 
\jdtype{\jdtypesimple{Purpose}}
\end{jdfield}
\begin{jdmethod}{values}
\jdpublic \jdstatic 
\jdtype{\jdtypearray{Purpose}{\lbrack{}\rbrack{}}}
\end{jdmethod}
\begin{jdmethod}{valueOf}
\jdpublic \jdstatic 
\jdtype{\jdtypesimple{Purpose}}
\JDpara{\jdtypesimple{String}}{name}{}
\end{jdmethod}
\begin{jdmethod}{setOptional}
\jdpublic 
\jdtype{\jdtypesimple{void}}
\end{jdmethod}
\begin{jdmethod}{isOptional}
\jdpublic 
\jdtype{\jdtypesimple{boolean}}
\end{jdmethod}
\end{jdclass}

\begin{jdclass}[class]{Category}
\begin{jdclassheader}

\jdpublic \jdfinal 
\jdpackage{com.kpro.dataobjects}
\jdinherits{\jdtypesimple{Object}\jdinh \jdtypesimple{Enum}}
\JDtext{The categories that a case can contain. See P3P specs for more info.}
\JDauthor{ernie}
\end{jdclassheader}
\begin{jdinheritancetable} \jdInhEntry{\jdtypesimple{Object} clone(  )}{Enum}
 \jdInhEntry{\jdtypesimple{int} compareTo( \jdtypesimple{Enum} )}{Enum}
 \jdInhEntry{\jdtypesimple{int} compareTo( \jdtypesimple{Object} )}{Enum}
 \jdInhEntry{\jdtypesimple{boolean} equals( \jdtypesimple{Object} )}{Enum}
 \jdInhEntry{\jdtypesimple{void} finalize(  )}{Enum}
 \jdInhEntry{\jdtypesimple{Class} getDeclaringClass(  )}{Enum}
 \jdInhEntry{\jdtypesimple{int} hashCode(  )}{Enum}
 \jdInhEntry{\jdtypesimple{String} name(  )}{Enum}
 \jdInhEntry{\jdtypesimple{int} ordinal(  )}{Enum}
 \jdInhEntry{\jdtypesimple{String} toString(  )}{Enum}
 \jdInhEntry{\jdtypesimple{Enum} valueOf( \jdtypesimple{Class}, \jdtypesimple{String} )}{Enum}
 \jdInhEntry{\jdtypesimple{Object} clone(  )}{Object}
 \jdInhEntry{\jdtypesimple{boolean} equals( \jdtypesimple{Object} )}{Object}
 \jdInhEntry{\jdtypesimple{void} finalize(  )}{Object}
 \jdInhEntry{\jdtypesimple{Class} getClass(  )}{Object}
 \jdInhEntry{\jdtypesimple{int} hashCode(  )}{Object}
 \jdInhEntry{\jdtypesimple{void} notify(  )}{Object}
 \jdInhEntry{\jdtypesimple{void} notifyAll(  )}{Object}
 \jdInhEntry{\jdtypesimple{String} toString(  )}{Object}
 \jdInhEntry{\jdtypesimple{void} wait( \jdtypesimple{long} )}{Object}
 \jdInhEntry{\jdtypesimple{void} wait( \jdtypesimple{long}, \jdtypesimple{int} )}{Object}
 \jdInhEntry{\jdtypesimple{void} wait(  )}{Object}
\end{jdinheritancetable}
\begin{jdfield}{PHYSICAL}
\jdpublic \jdfinal \jdstatic 
\jdtype{\jdtypesimple{Category}}
\end{jdfield}
\begin{jdfield}{ONLINE}
\jdpublic \jdfinal \jdstatic 
\jdtype{\jdtypesimple{Category}}
\end{jdfield}
\begin{jdfield}{UNIQUEID}
\jdpublic \jdfinal \jdstatic 
\jdtype{\jdtypesimple{Category}}
\end{jdfield}
\begin{jdfield}{PURCHASE}
\jdpublic \jdfinal \jdstatic 
\jdtype{\jdtypesimple{Category}}
\end{jdfield}
\begin{jdfield}{FINANCIAL}
\jdpublic \jdfinal \jdstatic 
\jdtype{\jdtypesimple{Category}}
\end{jdfield}
\begin{jdfield}{COMPUTER}
\jdpublic \jdfinal \jdstatic 
\jdtype{\jdtypesimple{Category}}
\end{jdfield}
\begin{jdfield}{NAVIGATION}
\jdpublic \jdfinal \jdstatic 
\jdtype{\jdtypesimple{Category}}
\end{jdfield}
\begin{jdfield}{INTERACTIVE}
\jdpublic \jdfinal \jdstatic 
\jdtype{\jdtypesimple{Category}}
\end{jdfield}
\begin{jdfield}{DEMOGRAPHIC}
\jdpublic \jdfinal \jdstatic 
\jdtype{\jdtypesimple{Category}}
\end{jdfield}
\begin{jdfield}{CONTENT}
\jdpublic \jdfinal \jdstatic 
\jdtype{\jdtypesimple{Category}}
\end{jdfield}
\begin{jdfield}{STATE}
\jdpublic \jdfinal \jdstatic 
\jdtype{\jdtypesimple{Category}}
\end{jdfield}
\begin{jdfield}{POLITICAL}
\jdpublic \jdfinal \jdstatic 
\jdtype{\jdtypesimple{Category}}
\end{jdfield}
\begin{jdfield}{HEALTH}
\jdpublic \jdfinal \jdstatic 
\jdtype{\jdtypesimple{Category}}
\end{jdfield}
\begin{jdfield}{PREFERENCE}
\jdpublic \jdfinal \jdstatic 
\jdtype{\jdtypesimple{Category}}
\end{jdfield}
\begin{jdfield}{LOCATION}
\jdpublic \jdfinal \jdstatic 
\jdtype{\jdtypesimple{Category}}
\end{jdfield}
\begin{jdfield}{GOVERNMENT}
\jdpublic \jdfinal \jdstatic 
\jdtype{\jdtypesimple{Category}}
\end{jdfield}
\begin{jdfield}{OTHER\_CATEGORY}
\jdpublic \jdfinal \jdstatic 
\jdtype{\jdtypesimple{Category}}
\end{jdfield}
\begin{jdmethod}{values}
\jdpublic \jdstatic 
\jdtype{\jdtypearray{Category}{\lbrack{}\rbrack{}}}
\end{jdmethod}
\begin{jdmethod}{valueOf}
\jdpublic \jdstatic 
\jdtype{\jdtypesimple{Category}}
\JDpara{\jdtypesimple{String}}{name}{}
\end{jdmethod}
\end{jdclass}


\section{Parser}
\begin{jdclass}[class]{P3PParser}
\begin{jdclassheader}

\jdpublic 
\jdpackage{com.kpro.parser}
\jdinherits{\jdtypesimple{Object}}
\JDtext{Parser that parses a P3P policy and makes it into a PolicyObject}
\JDauthor{ernie}
\end{jdclassheader}
\begin{jdinheritancetable} \jdInhEntry{\jdtypesimple{Object} clone(  )}{Object}
 \jdInhEntry{\jdtypesimple{boolean} equals( \jdtypesimple{Object} )}{Object}
 \jdInhEntry{\jdtypesimple{void} finalize(  )}{Object}
 \jdInhEntry{\jdtypesimple{Class} getClass(  )}{Object}
 \jdInhEntry{\jdtypesimple{int} hashCode(  )}{Object}
 \jdInhEntry{\jdtypesimple{void} notify(  )}{Object}
 \jdInhEntry{\jdtypesimple{void} notifyAll(  )}{Object}
 \jdInhEntry{\jdtypesimple{String} toString(  )}{Object}
 \jdInhEntry{\jdtypesimple{void} wait( \jdtypesimple{long} )}{Object}
 \jdInhEntry{\jdtypesimple{void} wait( \jdtypesimple{long}, \jdtypesimple{int} )}{Object}
 \jdInhEntry{\jdtypesimple{void} wait(  )}{Object}
\end{jdinheritancetable}
\begin{jdconstructor}
\jdpublic 
\end{jdconstructor}
\begin{jdmethod}{parse}
\jdpublic 
\jdtype{\jdtypesimple{PolicyObject}}
\JDpara{\jdtypesimple{String}}{p3p}{input String}
\JDtext{Parses a P3P Policy by URL}
\JDauthor{ernie}
\JDreturn{Parsed P3P Policy as PolicyObject}
\end{jdmethod}
\end{jdclass}


\section{UI}
\begin{jdclass}[class]{PrivacyAdvisorGUI}
\begin{jdclassheader}

\jdpublic 
\jdpackage{com.kpro.ui}
\jdinherits{\jdtypesimple{Object}\jdinh \jdtypesimple{UserIO}}
\JDtext{Privacy Advisor GUI to run on top of}
\JDauthor{ulfnore}
\end{jdclassheader}
\begin{jdinheritancetable} \jdInhEntry{\jdtypesimple{void} closeResources(  )}{UserIO}
 \jdInhEntry{\jdtypesimple{ArrayList} loadHistory(  )}{UserIO}
 \jdInhEntry{\jdtypesimple{void} showDatabase( \jdtypesimple{PolicyDatabase} )}{UserIO}
 \jdInhEntry{\jdtypesimple{PolicyObject} userResponse( \jdtypesimple{PolicyObject} )}{UserIO}
 \jdInhEntry{\jdtypesimple{void} user\_init( \jdtypesimple{Properties} )}{UserIO}
 \jdInhEntry{\jdtypesimple{Object} clone(  )}{Object}
 \jdInhEntry{\jdtypesimple{boolean} equals( \jdtypesimple{Object} )}{Object}
 \jdInhEntry{\jdtypesimple{void} finalize(  )}{Object}
 \jdInhEntry{\jdtypesimple{Class} getClass(  )}{Object}
 \jdInhEntry{\jdtypesimple{int} hashCode(  )}{Object}
 \jdInhEntry{\jdtypesimple{void} notify(  )}{Object}
 \jdInhEntry{\jdtypesimple{void} notifyAll(  )}{Object}
 \jdInhEntry{\jdtypesimple{String} toString(  )}{Object}
 \jdInhEntry{\jdtypesimple{void} wait( \jdtypesimple{long} )}{Object}
 \jdInhEntry{\jdtypesimple{void} wait( \jdtypesimple{long}, \jdtypesimple{int} )}{Object}
 \jdInhEntry{\jdtypesimple{void} wait(  )}{Object}
\end{jdinheritancetable}
\begin{jdconstructor}
\jdpublic 
\JDtext{Default no-arg constructor}
\JDauthor{ulfnore}
\end{jdconstructor}
\begin{jdmethod}{main}
\jdpublic \jdstatic 
\jdtype{\jdtypesimple{void}}
\JDpara{\jdtypearray{String}{\lbrack{}\rbrack{}}}{args}{}
\JDtext{Launch the application.}
\JDauthor{ulfnore}
\end{jdmethod}
\begin{jdmethod}{user\_init}
\jdpublic 
\jdtype{\jdtypesimple{void}}
\JDpara{\jdtypesimple{Properties}}{genProps}{}
\JDtext{Called from GIO. Takes default properties file as argument.}
\end{jdmethod}
\begin{jdmethod}{loadHistory}
\jdpublic 
\jdtype{\jdtypesimple{ArrayList}}
\end{jdmethod}
\begin{jdmethod}{userResponse}
\jdpublic 
\jdtype{\jdtypesimple{PolicyObject}}
\JDpara{\jdtypesimple{PolicyObject}}{n}{}
\JDtext{Shows recommendation and prompts for user action
 
 Needs improvement to allow for giving reasons as for why  
 recommendation is not accepted.}
\JDauthor{ulfnore}
\end{jdmethod}
\begin{jdmethod}{closeResources}
\jdpublic 
\jdtype{\jdtypesimple{void}}
\end{jdmethod}
\begin{jdmethod}{showDatabase}
\jdpublic 
\jdtype{\jdtypesimple{void}}
\JDpara{\jdtypesimple{PolicyDatabase}}{pdb}{}
\end{jdmethod}
\end{jdclass}

\begin{jdclass}[class]{UserIO}
\begin{jdclassheader}

\jdabstract \jdpublic 
\jdpackage{com.kpro.ui}
\jdinherits{\jdtypesimple{Object}}
\JDtext{The UserIO provides an abstract model of all the methods a user interface method must implement. There are five essential portions: construction of the user interface if necessary, via the object constructor; user reconfiguration (of the same options found in configuration file or on the commandline); the ability to display the database and all loaded policies; user revision, in which the suggested solution is provided to the user so the user can accept or reject it; and shutdown/deconstruction of any resources needed for the interface.}
\end{jdclassheader}
\begin{jdinheritancetable} \jdInhEntry{\jdtypesimple{Object} clone(  )}{Object}
 \jdInhEntry{\jdtypesimple{boolean} equals( \jdtypesimple{Object} )}{Object}
 \jdInhEntry{\jdtypesimple{void} finalize(  )}{Object}
 \jdInhEntry{\jdtypesimple{Class} getClass(  )}{Object}
 \jdInhEntry{\jdtypesimple{int} hashCode(  )}{Object}
 \jdInhEntry{\jdtypesimple{void} notify(  )}{Object}
 \jdInhEntry{\jdtypesimple{void} notifyAll(  )}{Object}
 \jdInhEntry{\jdtypesimple{String} toString(  )}{Object}
 \jdInhEntry{\jdtypesimple{void} wait( \jdtypesimple{long} )}{Object}
 \jdInhEntry{\jdtypesimple{void} wait( \jdtypesimple{long}, \jdtypesimple{int} )}{Object}
 \jdInhEntry{\jdtypesimple{void} wait(  )}{Object}
\end{jdinheritancetable}
\begin{jdconstructor}
\jdpublic 
\end{jdconstructor}
\begin{jdmethod}{user\_init}
\jdpublic \jdabstract 
\jdtype{\jdtypesimple{void}}
\JDpara{\jdtypesimple{Properties}}{genProps}{the default values for all commandline arguments}
\JDtext{returns a modified Properties to use init on.}
\JDreturn{the values to initialize the program with- same as usual args from main}
\end{jdmethod}
\begin{jdmethod}{showDatabase}
\jdpublic \jdabstract 
\jdtype{\jdtypesimple{void}}
\JDpara{\jdtypesimple{PolicyDatabase}}{pdb}{the database to display}
\JDtext{display the contents of the database}
\JDauthor{ngerstle}
\end{jdmethod}
\begin{jdmethod}{loadHistory}
\jdpublic \jdabstract 
\jdtype{\jdtypesimple{ArrayList}}
\JDtext{gets any policies not already provided for the history}
\JDdeprecated{}
\JDreturn{an arraylist of policy objects to be added to history prior to the CBR run.}
\JDauthor{ngerstle}
\end{jdmethod}
\begin{jdmethod}{userResponse}
\jdpublic \jdabstract 
\jdtype{\jdtypesimple{PolicyObject}}
\JDpara{\jdtypesimple{PolicyObject}}{n}{the policy display}
\JDtext{Displays recommended action for policyObject, and returns used accept verion-
 same thing if no change, or altered if user disagrees.}
\JDreturn{the policy given}
\JDauthor{ngerstle}
\end{jdmethod}
\begin{jdmethod}{closeResources}
\jdpublic \jdabstract 
\jdtype{\jdtypesimple{void}}
\JDtext{closes all resources used by UserIO - windows, files, streams, etc}
\JDauthor{ngerstle}
\end{jdmethod}
\end{jdclass}

\begin{jdclass}[class]{UserIO\_Simple}
\begin{jdclassheader}

\jdpublic 
\jdpackage{com.kpro.ui}
\jdinherits{\jdtypesimple{Object}\jdinh \jdtypesimple{UserIO}}
\JDtext{This is a very simple commandline version of a user interface. It doesn't permit user configuration (of running options), 
 it has a ugly database display, but it works.}
\JDauthor{ngerstle}
\JDauthor{ulfnore}
\end{jdclassheader}
\begin{jdinheritancetable} \jdInhEntry{\jdtypesimple{void} closeResources(  )}{UserIO}
 \jdInhEntry{\jdtypesimple{ArrayList} loadHistory(  )}{UserIO}
 \jdInhEntry{\jdtypesimple{void} showDatabase( \jdtypesimple{PolicyDatabase} )}{UserIO}
 \jdInhEntry{\jdtypesimple{PolicyObject} userResponse( \jdtypesimple{PolicyObject} )}{UserIO}
 \jdInhEntry{\jdtypesimple{void} user\_init( \jdtypesimple{Properties} )}{UserIO}
 \jdInhEntry{\jdtypesimple{Object} clone(  )}{Object}
 \jdInhEntry{\jdtypesimple{boolean} equals( \jdtypesimple{Object} )}{Object}
 \jdInhEntry{\jdtypesimple{void} finalize(  )}{Object}
 \jdInhEntry{\jdtypesimple{Class} getClass(  )}{Object}
 \jdInhEntry{\jdtypesimple{int} hashCode(  )}{Object}
 \jdInhEntry{\jdtypesimple{void} notify(  )}{Object}
 \jdInhEntry{\jdtypesimple{void} notifyAll(  )}{Object}
 \jdInhEntry{\jdtypesimple{String} toString(  )}{Object}
 \jdInhEntry{\jdtypesimple{void} wait( \jdtypesimple{long} )}{Object}
 \jdInhEntry{\jdtypesimple{void} wait( \jdtypesimple{long}, \jdtypesimple{int} )}{Object}
 \jdInhEntry{\jdtypesimple{void} wait(  )}{Object}
\end{jdinheritancetable}
\begin{jdconstructor}
\jdpublic 
\end{jdconstructor}
\begin{jdmethod}{showDatabase}
\jdpublic 
\jdtype{\jdtypesimple{void}}
\JDpara{\jdtypesimple{PolicyDatabase}}{pdb}{}
\JDtext{does nothing}
\JDsee{UserIO\#{}showDatabase(PolicyDatabase)}
\end{jdmethod}
\begin{jdmethod}{loadHistory}
\jdpublic 
\jdtype{\jdtypesimple{ArrayList}}
\JDtext{does nothing}
\JDsee{UserIO\#{}loadHistory()}
\end{jdmethod}
\begin{jdmethod}{userResponse}
\jdpublic 
\jdtype{\jdtypesimple{PolicyObject}}
\JDpara{\jdtypesimple{PolicyObject}}{n}{the policy display}
\JDtext{A super simple, static user display of the result on command line. does not wait for user response}
\JDreturn{the policy given}
\JDauthor{ngerstle}
\JDauthor{ulfnore}
\JDsee{UserIO\#{}userResponse(PolicyObject)}
\end{jdmethod}
\begin{jdmethod}{closeResources}
\jdpublic 
\jdtype{\jdtypesimple{void}}
\JDtext{nothing to close}
\JDauthor{ngerstle}
\end{jdmethod}
\begin{jdmethod}{user\_init}
\jdpublic 
\jdtype{\jdtypesimple{void}}
\JDpara{\jdtypesimple{Properties}}{genProps}{}
\JDtext{This user interface doesn't actually let the user reconfigure anything. Would do so through the reference to Gio if necessary.}
\end{jdmethod}
\end{jdclass}

\begin{jdclass}[class]{IO\_Listing}
\begin{jdclassheader}

\jdpublic \jdfinal 
\jdpackage{com.kpro.ui}
\jdinherits{\jdtypesimple{Object}\jdinh \jdtypesimple{Enum}}
\JDauthor{ulfnore}
\end{jdclassheader}
\begin{jdinheritancetable} \jdInhEntry{\jdtypesimple{Object} clone(  )}{Enum}
 \jdInhEntry{\jdtypesimple{int} compareTo( \jdtypesimple{Enum} )}{Enum}
 \jdInhEntry{\jdtypesimple{int} compareTo( \jdtypesimple{Object} )}{Enum}
 \jdInhEntry{\jdtypesimple{boolean} equals( \jdtypesimple{Object} )}{Enum}
 \jdInhEntry{\jdtypesimple{void} finalize(  )}{Enum}
 \jdInhEntry{\jdtypesimple{Class} getDeclaringClass(  )}{Enum}
 \jdInhEntry{\jdtypesimple{int} hashCode(  )}{Enum}
 \jdInhEntry{\jdtypesimple{String} name(  )}{Enum}
 \jdInhEntry{\jdtypesimple{int} ordinal(  )}{Enum}
 \jdInhEntry{\jdtypesimple{String} toString(  )}{Enum}
 \jdInhEntry{\jdtypesimple{Enum} valueOf( \jdtypesimple{Class}, \jdtypesimple{String} )}{Enum}
 \jdInhEntry{\jdtypesimple{Object} clone(  )}{Object}
 \jdInhEntry{\jdtypesimple{boolean} equals( \jdtypesimple{Object} )}{Object}
 \jdInhEntry{\jdtypesimple{void} finalize(  )}{Object}
 \jdInhEntry{\jdtypesimple{Class} getClass(  )}{Object}
 \jdInhEntry{\jdtypesimple{int} hashCode(  )}{Object}
 \jdInhEntry{\jdtypesimple{void} notify(  )}{Object}
 \jdInhEntry{\jdtypesimple{void} notifyAll(  )}{Object}
 \jdInhEntry{\jdtypesimple{String} toString(  )}{Object}
 \jdInhEntry{\jdtypesimple{void} wait( \jdtypesimple{long} )}{Object}
 \jdInhEntry{\jdtypesimple{void} wait( \jdtypesimple{long}, \jdtypesimple{int} )}{Object}
 \jdInhEntry{\jdtypesimple{void} wait(  )}{Object}
\end{jdinheritancetable}
\begin{jdfield}{UserIO\_Simple}
\jdpublic \jdfinal \jdstatic 
\jdtype{\jdtypesimple{IO\_Listing}}
\end{jdfield}
\begin{jdfield}{PrivacyAdvisorGUI}
\jdpublic \jdfinal \jdstatic 
\jdtype{\jdtypesimple{IO\_Listing}}
\end{jdfield}
\begin{jdmethod}{values}
\jdpublic \jdstatic 
\jdtype{\jdtypearray{IO\_Listing}{\lbrack{}\rbrack{}}}
\end{jdmethod}
\begin{jdmethod}{valueOf}
\jdpublic \jdstatic 
\jdtype{\jdtypesimple{IO\_Listing}}
\JDpara{\jdtypesimple{String}}{name}{}
\end{jdmethod}
\end{jdclass}


\section{Test}
\include{Appendix/javadoc/com.kpro.testrestReduction-KNN}
\begin{jdclass}[class]{testConclusion\_Simple}
\begin{jdclassheader}

\jdpublic 
\jdpackage{com.kpro.test}
\jdinherits{\jdtypesimple{Object}\jdinh \jdtypesimple{Assert}\jdinh \jdtypesimple{TestCase}}
\end{jdclassheader}
\begin{jdinheritancetable} \jdInhEntry{\jdtypesimple{int} countTestCases(  )}{TestCase}
 \jdInhEntry{\jdtypesimple{TestResult} createResult(  )}{TestCase}
 \jdInhEntry{\jdtypesimple{String} getName(  )}{TestCase}
 \jdInhEntry{\jdtypesimple{TestResult} run(  )}{TestCase}
 \jdInhEntry{\jdtypesimple{void} run( \jdtypesimple{TestResult} )}{TestCase}
 \jdInhEntry{\jdtypesimple{void} runBare(  )}{TestCase}
 \jdInhEntry{\jdtypesimple{void} runTest(  )}{TestCase}
 \jdInhEntry{\jdtypesimple{void} setName( \jdtypesimple{String} )}{TestCase}
 \jdInhEntry{\jdtypesimple{void} setUp(  )}{TestCase}
 \jdInhEntry{\jdtypesimple{void} tearDown(  )}{TestCase}
 \jdInhEntry{\jdtypesimple{String} toString(  )}{TestCase}
 \jdInhEntry{\jdtypesimple{void} assertEquals( \jdtypesimple{String}, \jdtypesimple{Object}, \jdtypesimple{Object} )}{Assert}
 \jdInhEntry{\jdtypesimple{void} assertEquals( \jdtypesimple{Object}, \jdtypesimple{Object} )}{Assert}
 \jdInhEntry{\jdtypesimple{void} assertEquals( \jdtypesimple{String}, \jdtypesimple{String}, \jdtypesimple{String} )}{Assert}
 \jdInhEntry{\jdtypesimple{void} assertEquals( \jdtypesimple{String}, \jdtypesimple{String} )}{Assert}
 \jdInhEntry{\jdtypesimple{void} assertEquals( \jdtypesimple{String}, \jdtypesimple{double}, \jdtypesimple{double}, \jdtypesimple{double} )}{Assert}
 \jdInhEntry{\jdtypesimple{void} assertEquals( \jdtypesimple{double}, \jdtypesimple{double}, \jdtypesimple{double} )}{Assert}
 \jdInhEntry{\jdtypesimple{void} assertEquals( \jdtypesimple{String}, \jdtypesimple{float}, \jdtypesimple{float}, \jdtypesimple{float} )}{Assert}
 \jdInhEntry{\jdtypesimple{void} assertEquals( \jdtypesimple{float}, \jdtypesimple{float}, \jdtypesimple{float} )}{Assert}
 \jdInhEntry{\jdtypesimple{void} assertEquals( \jdtypesimple{String}, \jdtypesimple{long}, \jdtypesimple{long} )}{Assert}
 \jdInhEntry{\jdtypesimple{void} assertEquals( \jdtypesimple{long}, \jdtypesimple{long} )}{Assert}
 \jdInhEntry{\jdtypesimple{void} assertEquals( \jdtypesimple{String}, \jdtypesimple{boolean}, \jdtypesimple{boolean} )}{Assert}
 \jdInhEntry{\jdtypesimple{void} assertEquals( \jdtypesimple{boolean}, \jdtypesimple{boolean} )}{Assert}
 \jdInhEntry{\jdtypesimple{void} assertEquals( \jdtypesimple{String}, \jdtypesimple{byte}, \jdtypesimple{byte} )}{Assert}
 \jdInhEntry{\jdtypesimple{void} assertEquals( \jdtypesimple{byte}, \jdtypesimple{byte} )}{Assert}
 \jdInhEntry{\jdtypesimple{void} assertEquals( \jdtypesimple{String}, \jdtypesimple{char}, \jdtypesimple{char} )}{Assert}
 \jdInhEntry{\jdtypesimple{void} assertEquals( \jdtypesimple{char}, \jdtypesimple{char} )}{Assert}
 \jdInhEntry{\jdtypesimple{void} assertEquals( \jdtypesimple{String}, \jdtypesimple{short}, \jdtypesimple{short} )}{Assert}
 \jdInhEntry{\jdtypesimple{void} assertEquals( \jdtypesimple{short}, \jdtypesimple{short} )}{Assert}
 \jdInhEntry{\jdtypesimple{void} assertEquals( \jdtypesimple{String}, \jdtypesimple{int}, \jdtypesimple{int} )}{Assert}
 \jdInhEntry{\jdtypesimple{void} assertEquals( \jdtypesimple{int}, \jdtypesimple{int} )}{Assert}
 \jdInhEntry{\jdtypesimple{void} assertFalse( \jdtypesimple{String}, \jdtypesimple{boolean} )}{Assert}
 \jdInhEntry{\jdtypesimple{void} assertFalse( \jdtypesimple{boolean} )}{Assert}
 \jdInhEntry{\jdtypesimple{void} assertNotNull( \jdtypesimple{Object} )}{Assert}
 \jdInhEntry{\jdtypesimple{void} assertNotNull( \jdtypesimple{String}, \jdtypesimple{Object} )}{Assert}
 \jdInhEntry{\jdtypesimple{void} assertNotSame( \jdtypesimple{String}, \jdtypesimple{Object}, \jdtypesimple{Object} )}{Assert}
 \jdInhEntry{\jdtypesimple{void} assertNotSame( \jdtypesimple{Object}, \jdtypesimple{Object} )}{Assert}
 \jdInhEntry{\jdtypesimple{void} assertNull( \jdtypesimple{Object} )}{Assert}
 \jdInhEntry{\jdtypesimple{void} assertNull( \jdtypesimple{String}, \jdtypesimple{Object} )}{Assert}
 \jdInhEntry{\jdtypesimple{void} assertSame( \jdtypesimple{String}, \jdtypesimple{Object}, \jdtypesimple{Object} )}{Assert}
 \jdInhEntry{\jdtypesimple{void} assertSame( \jdtypesimple{Object}, \jdtypesimple{Object} )}{Assert}
 \jdInhEntry{\jdtypesimple{void} assertTrue( \jdtypesimple{String}, \jdtypesimple{boolean} )}{Assert}
 \jdInhEntry{\jdtypesimple{void} assertTrue( \jdtypesimple{boolean} )}{Assert}
 \jdInhEntry{\jdtypesimple{void} fail( \jdtypesimple{String} )}{Assert}
 \jdInhEntry{\jdtypesimple{void} fail(  )}{Assert}
 \jdInhEntry{\jdtypesimple{void} failNotEquals( \jdtypesimple{String}, \jdtypesimple{Object}, \jdtypesimple{Object} )}{Assert}
 \jdInhEntry{\jdtypesimple{void} failNotSame( \jdtypesimple{String}, \jdtypesimple{Object}, \jdtypesimple{Object} )}{Assert}
 \jdInhEntry{\jdtypesimple{void} failSame( \jdtypesimple{String} )}{Assert}
 \jdInhEntry{\jdtypesimple{Object} clone(  )}{Object}
 \jdInhEntry{\jdtypesimple{boolean} equals( \jdtypesimple{Object} )}{Object}
 \jdInhEntry{\jdtypesimple{void} finalize(  )}{Object}
 \jdInhEntry{\jdtypesimple{Class} getClass(  )}{Object}
 \jdInhEntry{\jdtypesimple{int} hashCode(  )}{Object}
 \jdInhEntry{\jdtypesimple{void} notify(  )}{Object}
 \jdInhEntry{\jdtypesimple{void} notifyAll(  )}{Object}
 \jdInhEntry{\jdtypesimple{String} toString(  )}{Object}
 \jdInhEntry{\jdtypesimple{void} wait( \jdtypesimple{long} )}{Object}
 \jdInhEntry{\jdtypesimple{void} wait( \jdtypesimple{long}, \jdtypesimple{int} )}{Object}
 \jdInhEntry{\jdtypesimple{void} wait(  )}{Object}
\end{jdinheritancetable}
\begin{jdconstructor}
\jdpublic 
\end{jdconstructor}
\begin{jdmethod}{testConclusion}
\jdpublic 
\jdtype{\jdtypesimple{void}}
\end{jdmethod}
\end{jdclass}

\include{Appendix/javadoc/com.kpro.testtestBitmap}
\begin{jdclass}[class]{LearnAlgSimplerTest}
\begin{jdclassheader}

\jdpublic 
\jdpackage{com.kpro.test}
\jdinherits{\jdtypesimple{Object}\jdinh \jdtypesimple{Assert}\jdinh \jdtypesimple{TestCase}}
\JDtext{This class is a junit test class to test the class LearnAlgSimpler
 For this junit test to work correctly some changes have to be made in LearnAlgSimpler.java
 These are:
 		-comment out "extends LearnAlgorithm"
 		-comment out the constructor 
 		-make the applyML method take in a Properties and a PolicyDatabase,
 		 and make the method public, like this: 
 		 			public Properties applyML(Properties prop, PolicyDatabase pd)
 		-change the two first lines in applyML to:
 					Properties weights = prop;
					pdb = pd;}
\JDauthor{Nesha}
\end{jdclassheader}
\begin{jdinheritancetable} \jdInhEntry{\jdtypesimple{int} countTestCases(  )}{TestCase}
 \jdInhEntry{\jdtypesimple{TestResult} createResult(  )}{TestCase}
 \jdInhEntry{\jdtypesimple{String} getName(  )}{TestCase}
 \jdInhEntry{\jdtypesimple{TestResult} run(  )}{TestCase}
 \jdInhEntry{\jdtypesimple{void} run( \jdtypesimple{TestResult} )}{TestCase}
 \jdInhEntry{\jdtypesimple{void} runBare(  )}{TestCase}
 \jdInhEntry{\jdtypesimple{void} runTest(  )}{TestCase}
 \jdInhEntry{\jdtypesimple{void} setName( \jdtypesimple{String} )}{TestCase}
 \jdInhEntry{\jdtypesimple{void} setUp(  )}{TestCase}
 \jdInhEntry{\jdtypesimple{void} tearDown(  )}{TestCase}
 \jdInhEntry{\jdtypesimple{String} toString(  )}{TestCase}
 \jdInhEntry{\jdtypesimple{void} assertEquals( \jdtypesimple{String}, \jdtypesimple{Object}, \jdtypesimple{Object} )}{Assert}
 \jdInhEntry{\jdtypesimple{void} assertEquals( \jdtypesimple{Object}, \jdtypesimple{Object} )}{Assert}
 \jdInhEntry{\jdtypesimple{void} assertEquals( \jdtypesimple{String}, \jdtypesimple{String}, \jdtypesimple{String} )}{Assert}
 \jdInhEntry{\jdtypesimple{void} assertEquals( \jdtypesimple{String}, \jdtypesimple{String} )}{Assert}
 \jdInhEntry{\jdtypesimple{void} assertEquals( \jdtypesimple{String}, \jdtypesimple{double}, \jdtypesimple{double}, \jdtypesimple{double} )}{Assert}
 \jdInhEntry{\jdtypesimple{void} assertEquals( \jdtypesimple{double}, \jdtypesimple{double}, \jdtypesimple{double} )}{Assert}
 \jdInhEntry{\jdtypesimple{void} assertEquals( \jdtypesimple{String}, \jdtypesimple{float}, \jdtypesimple{float}, \jdtypesimple{float} )}{Assert}
 \jdInhEntry{\jdtypesimple{void} assertEquals( \jdtypesimple{float}, \jdtypesimple{float}, \jdtypesimple{float} )}{Assert}
 \jdInhEntry{\jdtypesimple{void} assertEquals( \jdtypesimple{String}, \jdtypesimple{long}, \jdtypesimple{long} )}{Assert}
 \jdInhEntry{\jdtypesimple{void} assertEquals( \jdtypesimple{long}, \jdtypesimple{long} )}{Assert}
 \jdInhEntry{\jdtypesimple{void} assertEquals( \jdtypesimple{String}, \jdtypesimple{boolean}, \jdtypesimple{boolean} )}{Assert}
 \jdInhEntry{\jdtypesimple{void} assertEquals( \jdtypesimple{boolean}, \jdtypesimple{boolean} )}{Assert}
 \jdInhEntry{\jdtypesimple{void} assertEquals( \jdtypesimple{String}, \jdtypesimple{byte}, \jdtypesimple{byte} )}{Assert}
 \jdInhEntry{\jdtypesimple{void} assertEquals( \jdtypesimple{byte}, \jdtypesimple{byte} )}{Assert}
 \jdInhEntry{\jdtypesimple{void} assertEquals( \jdtypesimple{String}, \jdtypesimple{char}, \jdtypesimple{char} )}{Assert}
 \jdInhEntry{\jdtypesimple{void} assertEquals( \jdtypesimple{char}, \jdtypesimple{char} )}{Assert}
 \jdInhEntry{\jdtypesimple{void} assertEquals( \jdtypesimple{String}, \jdtypesimple{short}, \jdtypesimple{short} )}{Assert}
 \jdInhEntry{\jdtypesimple{void} assertEquals( \jdtypesimple{short}, \jdtypesimple{short} )}{Assert}
 \jdInhEntry{\jdtypesimple{void} assertEquals( \jdtypesimple{String}, \jdtypesimple{int}, \jdtypesimple{int} )}{Assert}
 \jdInhEntry{\jdtypesimple{void} assertEquals( \jdtypesimple{int}, \jdtypesimple{int} )}{Assert}
 \jdInhEntry{\jdtypesimple{void} assertFalse( \jdtypesimple{String}, \jdtypesimple{boolean} )}{Assert}
 \jdInhEntry{\jdtypesimple{void} assertFalse( \jdtypesimple{boolean} )}{Assert}
 \jdInhEntry{\jdtypesimple{void} assertNotNull( \jdtypesimple{Object} )}{Assert}
 \jdInhEntry{\jdtypesimple{void} assertNotNull( \jdtypesimple{String}, \jdtypesimple{Object} )}{Assert}
 \jdInhEntry{\jdtypesimple{void} assertNotSame( \jdtypesimple{String}, \jdtypesimple{Object}, \jdtypesimple{Object} )}{Assert}
 \jdInhEntry{\jdtypesimple{void} assertNotSame( \jdtypesimple{Object}, \jdtypesimple{Object} )}{Assert}
 \jdInhEntry{\jdtypesimple{void} assertNull( \jdtypesimple{Object} )}{Assert}
 \jdInhEntry{\jdtypesimple{void} assertNull( \jdtypesimple{String}, \jdtypesimple{Object} )}{Assert}
 \jdInhEntry{\jdtypesimple{void} assertSame( \jdtypesimple{String}, \jdtypesimple{Object}, \jdtypesimple{Object} )}{Assert}
 \jdInhEntry{\jdtypesimple{void} assertSame( \jdtypesimple{Object}, \jdtypesimple{Object} )}{Assert}
 \jdInhEntry{\jdtypesimple{void} assertTrue( \jdtypesimple{String}, \jdtypesimple{boolean} )}{Assert}
 \jdInhEntry{\jdtypesimple{void} assertTrue( \jdtypesimple{boolean} )}{Assert}
 \jdInhEntry{\jdtypesimple{void} fail( \jdtypesimple{String} )}{Assert}
 \jdInhEntry{\jdtypesimple{void} fail(  )}{Assert}
 \jdInhEntry{\jdtypesimple{void} failNotEquals( \jdtypesimple{String}, \jdtypesimple{Object}, \jdtypesimple{Object} )}{Assert}
 \jdInhEntry{\jdtypesimple{void} failNotSame( \jdtypesimple{String}, \jdtypesimple{Object}, \jdtypesimple{Object} )}{Assert}
 \jdInhEntry{\jdtypesimple{void} failSame( \jdtypesimple{String} )}{Assert}
 \jdInhEntry{\jdtypesimple{Object} clone(  )}{Object}
 \jdInhEntry{\jdtypesimple{boolean} equals( \jdtypesimple{Object} )}{Object}
 \jdInhEntry{\jdtypesimple{void} finalize(  )}{Object}
 \jdInhEntry{\jdtypesimple{Class} getClass(  )}{Object}
 \jdInhEntry{\jdtypesimple{int} hashCode(  )}{Object}
 \jdInhEntry{\jdtypesimple{void} notify(  )}{Object}
 \jdInhEntry{\jdtypesimple{void} notifyAll(  )}{Object}
 \jdInhEntry{\jdtypesimple{String} toString(  )}{Object}
 \jdInhEntry{\jdtypesimple{void} wait( \jdtypesimple{long} )}{Object}
 \jdInhEntry{\jdtypesimple{void} wait( \jdtypesimple{long}, \jdtypesimple{int} )}{Object}
 \jdInhEntry{\jdtypesimple{void} wait(  )}{Object}
\end{jdinheritancetable}
\begin{jdconstructor}
\jdpublic 
\end{jdconstructor}
\begin{jdmethod}{testApplyML}
\jdpublic 
\jdtype{\jdtypesimple{void}}
\JDtext{The actual testing is done here
 
 Some lines are commented out because it shows error messages
 when the LearnAlgSimpler.java isn't as it should be when this
 junit test class is run (see the description of this class).
 
 Uncomment the commented lines before running the junit test}
\end{jdmethod}
\end{jdclass}


\section{Sample}
\begin{jdclass}[class]{TemplateClass}
\begin{jdclassheader}

\jdpublic 
\jdpackage{com.kpro.sample}
\jdinherits{\jdtypesimple{Object}}
\JDtext{Paragraph describing purpose of class in detail- what it contains, stores etc.}
\JDversion{versionnum //NB (use date.version- eg DDMMYY.1, etc)}
\JDauthor{author1}
\end{jdclassheader}
\begin{jdinheritancetable} \jdInhEntry{\jdtypesimple{Object} clone(  )}{Object}
 \jdInhEntry{\jdtypesimple{boolean} equals( \jdtypesimple{Object} )}{Object}
 \jdInhEntry{\jdtypesimple{void} finalize(  )}{Object}
 \jdInhEntry{\jdtypesimple{Class} getClass(  )}{Object}
 \jdInhEntry{\jdtypesimple{int} hashCode(  )}{Object}
 \jdInhEntry{\jdtypesimple{void} notify(  )}{Object}
 \jdInhEntry{\jdtypesimple{void} notifyAll(  )}{Object}
 \jdInhEntry{\jdtypesimple{String} toString(  )}{Object}
 \jdInhEntry{\jdtypesimple{void} wait( \jdtypesimple{long} )}{Object}
 \jdInhEntry{\jdtypesimple{void} wait( \jdtypesimple{long}, \jdtypesimple{int} )}{Object}
 \jdInhEntry{\jdtypesimple{void} wait(  )}{Object}
\end{jdinheritancetable}
\begin{jdfield}{status}
\jdpublic 
\jdtype{\jdtypesimple{int}}
\end{jdfield}
\begin{jdconstructor}
\jdpublic 
\JDpara{\jdtypesimple{int}}{status}{default starting status, should be 1}
\JDpara{\jdtypesimple{int}}{internalStatus}{default internal status, needs to be 1}
\JDpara{\jdtypesimple{String}}{name}{random string}
\JDtext{Paragraph laying out constructor details, etc. 
 Constructor can be auto-generated by Source->Generate Constructor...}
\JDauthor{author1}
\end{jdconstructor}
\begin{jdmethod}{getResult}
\jdpublic 
\jdtype{\jdtypesimple{String}}
\JDpara{\jdtypesimple{int}}{a}{input int for method...}
\JDtext{description of method}
\JDauthor{author1}
\JDreturn{and garbled string}
\end{jdmethod}
\begin{jdmethod}{setInternalStatus}
\jdpublic 
\jdtype{\jdtypesimple{void}}
\JDpara{\jdtypesimple{int}}{internalStatus}{}
\JDtext{description of method}
\JDauthor{author1}
\JDreturn{void}
\end{jdmethod}
\begin{jdmethod}{getInternalStatus}
\jdpublic 
\jdtype{\jdtypesimple{int}}
\JDtext{description of method}
\JDreturn{absolute value of the internal status times pi}
\end{jdmethod}
\begin{jdmethod}{setA}
\jdpublic 
\jdtype{\jdtypesimple{void}}
\JDpara{\jdtypesimple{ArrayList}}{a}{}
\end{jdmethod}
\begin{jdmethod}{getA}
\jdpublic 
\jdtype{\jdtypesimple{ArrayList}}
\end{jdmethod}
\end{jdclass}

\include{Appendix/javadoc/com.kpro.sampleSerializationDemo}
\begin{jdclass}[class]{readWeightConfig}
\begin{jdclassheader}

\jdpublic 
\jdpackage{com.kpro.sample}
\jdinherits{\jdtypesimple{Object}}
\JDtext{A class that reads the weights from the weights.cfg file, and returns 
 an array with values.
 The returned array have values that are arranged so that they correspond to the 
 values as they are written in the config file.
 That means that if the first uncommented line in the config file 
 is Recipient.OUR, then the first value in the returned array
 from the recipientWeight() method would be the value for Recipient.OUR.}
\JDauthor{Nesha}
\end{jdclassheader}
\begin{jdinheritancetable} \jdInhEntry{\jdtypesimple{Object} clone(  )}{Object}
 \jdInhEntry{\jdtypesimple{boolean} equals( \jdtypesimple{Object} )}{Object}
 \jdInhEntry{\jdtypesimple{void} finalize(  )}{Object}
 \jdInhEntry{\jdtypesimple{Class} getClass(  )}{Object}
 \jdInhEntry{\jdtypesimple{int} hashCode(  )}{Object}
 \jdInhEntry{\jdtypesimple{void} notify(  )}{Object}
 \jdInhEntry{\jdtypesimple{void} notifyAll(  )}{Object}
 \jdInhEntry{\jdtypesimple{String} toString(  )}{Object}
 \jdInhEntry{\jdtypesimple{void} wait( \jdtypesimple{long} )}{Object}
 \jdInhEntry{\jdtypesimple{void} wait( \jdtypesimple{long}, \jdtypesimple{int} )}{Object}
 \jdInhEntry{\jdtypesimple{void} wait(  )}{Object}
\end{jdinheritancetable}
\begin{jdconstructor}
\jdpublic 
\end{jdconstructor}
\begin{jdmethod}{recipientWeight}
\jdpublic 
\jdtype{\jdtypearray{int}{\lbrack{}\rbrack{}}}
\JDtext{reads the weights of the recipients 
 and returns them as a list}
\JDauthor{Nesha}
\JDreturn{an array of the weights of the recipients}
\end{jdmethod}
\begin{jdmethod}{retentionWeight}
\jdpublic 
\jdtype{\jdtypearray{int}{\lbrack{}\rbrack{}}}
\JDtext{reads the weights of the retentions 
 and returns them as a list}
\JDauthor{Nesha}
\JDreturn{an array of the weights of the retentions}
\end{jdmethod}
\begin{jdmethod}{purposeWeight}
\jdpublic 
\jdtype{\jdtypearray{int}{\lbrack{}\rbrack{}}}
\JDtext{reads the weights of the purposes 
 and returns them as a list}
\JDauthor{Nesha}
\JDreturn{an array of the weights of the purposes}
\end{jdmethod}
\end{jdclass}

\begin{jdclass}[class]{msgBox}
\begin{jdclassheader}

\jdpublic 
\jdpackage{com.kpro.sample}
\jdinherits{\jdtypesimple{Object}}
\end{jdclassheader}
\begin{jdinheritancetable} \jdInhEntry{\jdtypesimple{Object} clone(  )}{Object}
 \jdInhEntry{\jdtypesimple{boolean} equals( \jdtypesimple{Object} )}{Object}
 \jdInhEntry{\jdtypesimple{void} finalize(  )}{Object}
 \jdInhEntry{\jdtypesimple{Class} getClass(  )}{Object}
 \jdInhEntry{\jdtypesimple{int} hashCode(  )}{Object}
 \jdInhEntry{\jdtypesimple{void} notify(  )}{Object}
 \jdInhEntry{\jdtypesimple{void} notifyAll(  )}{Object}
 \jdInhEntry{\jdtypesimple{String} toString(  )}{Object}
 \jdInhEntry{\jdtypesimple{void} wait( \jdtypesimple{long} )}{Object}
 \jdInhEntry{\jdtypesimple{void} wait( \jdtypesimple{long}, \jdtypesimple{int} )}{Object}
 \jdInhEntry{\jdtypesimple{void} wait(  )}{Object}
\end{jdinheritancetable}
\begin{jdconstructor}
\jdpublic 
\end{jdconstructor}
\begin{jdmethod}{main}
\jdpublic \jdstatic 
\jdtype{\jdtypesimple{void}}
\JDpara{\jdtypearray{String}{\lbrack{}\rbrack{}}}{args}{}
\JDthrows{IOException}{}
\end{jdmethod}
\end{jdclass}

\include{Appendix/javadoc/com.kpro.samplejlistest}
\include{Appendix/javadoc/com.kpro.samplejfilechoosertest}
\include{Appendix/javadoc/com.kpro.sampleGioTest}
\include{Appendix/javadoc/com.kpro.sampledummy}
\begin{jdclass}[class]{distanceMetricTest}
\begin{jdclassheader}

\jdpublic 
\jdpackage{com.kpro.sample}
\jdinherits{\jdtypesimple{Object}\jdinh \jdtypesimple{DistanceMetric}}
\end{jdclassheader}
\begin{jdinheritancetable} \jdInhEntry{\jdtypesimple{double} getTotalDistance( \jdtypesimple{PolicyObject}, \jdtypesimple{PolicyObject} )}{DistanceMetric}
 \jdInhEntry{\jdtypesimple{Object} clone(  )}{Object}
 \jdInhEntry{\jdtypesimple{boolean} equals( \jdtypesimple{Object} )}{Object}
 \jdInhEntry{\jdtypesimple{void} finalize(  )}{Object}
 \jdInhEntry{\jdtypesimple{Class} getClass(  )}{Object}
 \jdInhEntry{\jdtypesimple{int} hashCode(  )}{Object}
 \jdInhEntry{\jdtypesimple{void} notify(  )}{Object}
 \jdInhEntry{\jdtypesimple{void} notifyAll(  )}{Object}
 \jdInhEntry{\jdtypesimple{String} toString(  )}{Object}
 \jdInhEntry{\jdtypesimple{void} wait( \jdtypesimple{long} )}{Object}
 \jdInhEntry{\jdtypesimple{void} wait( \jdtypesimple{long}, \jdtypesimple{int} )}{Object}
 \jdInhEntry{\jdtypesimple{void} wait(  )}{Object}
\end{jdinheritancetable}
\begin{jdconstructor}
\jdpublic 
\JDpara{\jdtypesimple{Properties}}{weightsConfig}{}
\end{jdconstructor}
\begin{jdmethod}{getDistResip}
\jdpublic 
\jdtype{\jdtypesimple{double}}
\JDpara{\jdtypesimple{Case}}{a}{}
\JDpara{\jdtypesimple{Case}}{b}{}
\end{jdmethod}
\begin{jdmethod}{getDistReten}
\jdpublic 
\jdtype{\jdtypesimple{double}}
\JDpara{\jdtypesimple{Case}}{a}{}
\JDpara{\jdtypesimple{Case}}{b}{}
\end{jdmethod}
\begin{jdmethod}{getDistPurpose}
\jdpublic 
\jdtype{\jdtypesimple{double}}
\JDpara{\jdtypesimple{Case}}{a}{}
\JDpara{\jdtypesimple{Case}}{b}{}
\end{jdmethod}
\begin{jdmethod}{getWeigth}
\jdpublic 
\jdtype{\jdtypesimple{double}}
\JDpara{\jdtypesimple{Recipient}}{r}{}
\JDtext{Just used some random return numbers, should be changed something meaningful later}
\end{jdmethod}
\begin{jdmethod}{getWeigth}
\jdpublic 
\jdtype{\jdtypesimple{double}}
\JDpara{\jdtypesimple{Retention}}{r}{}
\end{jdmethod}
\begin{jdmethod}{getWeigth}
\jdpublic 
\jdtype{\jdtypesimple{double}}
\JDpara{\jdtypesimple{Purpose}}{r}{}
\end{jdmethod}
\begin{jdmethod}{getTotalDistance}
\jdpublic 
\jdtype{\jdtypesimple{double}}
\JDpara{\jdtypesimple{PolicyObject}}{a}{}
\JDpara{\jdtypesimple{PolicyObject}}{b}{}
\end{jdmethod}
\end{jdclass}

\begin{jdclass}[class]{DeserializationDemo}
\begin{jdclassheader}

\jdpublic 
\jdpackage{com.kpro.sample}
\jdinherits{\jdtypesimple{Object}}
\JDauthor{Aman
 This class reads the serialized objects from file system and displaying it}
\end{jdclassheader}
\begin{jdinheritancetable} \jdInhEntry{\jdtypesimple{Object} clone(  )}{Object}
 \jdInhEntry{\jdtypesimple{boolean} equals( \jdtypesimple{Object} )}{Object}
 \jdInhEntry{\jdtypesimple{void} finalize(  )}{Object}
 \jdInhEntry{\jdtypesimple{Class} getClass(  )}{Object}
 \jdInhEntry{\jdtypesimple{int} hashCode(  )}{Object}
 \jdInhEntry{\jdtypesimple{void} notify(  )}{Object}
 \jdInhEntry{\jdtypesimple{void} notifyAll(  )}{Object}
 \jdInhEntry{\jdtypesimple{String} toString(  )}{Object}
 \jdInhEntry{\jdtypesimple{void} wait( \jdtypesimple{long} )}{Object}
 \jdInhEntry{\jdtypesimple{void} wait( \jdtypesimple{long}, \jdtypesimple{int} )}{Object}
 \jdInhEntry{\jdtypesimple{void} wait(  )}{Object}
\end{jdinheritancetable}
\begin{jdconstructor}
\jdpublic 
\end{jdconstructor}
\begin{jdmethod}{main}
\jdpublic \jdstatic 
\jdtype{\jdtypesimple{void}}
\JDpara{\jdtypearray{String}{\lbrack{}\rbrack{}}}{args}{}
\end{jdmethod}
\end{jdclass}


\@openrighttrue\makeatother

% Activate the following line by filling in the right side. If for example the name of the root file is Main.tex, write
% "...root = Main.tex" if the chapter file is in the same directory, and "...root = ../Main.tex" if the chapter is in a subdirectory.
 
%!TEX root =  


\begin{thebibliography}{99}

\bibitem{lamport94}
  Leslie Lamport,
  \emph{\LaTeX: A Document Preparation System}.
  Addison Wesley, Massachusetts,
  2nd Edition,
  1994.

\end{thebibliography}


\bibliographystyle{plain-annote}
\bibliography{mybibliography}


\printindex

\end{document}
\end

