% Activate the following line by filling in the right side. If for example the name of the root file is Main.tex, write
% "...root = Main.tex" if the chapter file is in the same directory, and "...root = ../Main.tex" if the chapter is in a subdirectory.
 
%!TEX root =  

\chapter{Preliminary Study}

\minitoc

This section describes the preliminary study phase of the project which is the time spent by the project team gaining insight into the problem the software system is to solve. In our case, the problem is very well defined from the customer's perspective, so little time is spent looking into similar products and various possible solutions to the overarching problem of computer aided privacy advice. A large portion of this time was spent on investigating the proposed solution and the technologies involved, that is Case Based Reasoning (CBR) and the technologies used by it. Another critical work laid down in this phase were choices with respect to project scope.

\section{Privacy Technology}
This section briefly describes the situation today with respect to Internet privacy tools and which needs the project seeks to address.

As discussed in T�ndel and Nyre (2010), there is a disproportion between the average Internet user's concern for privacy and the actual control he has over private information. While most websites provide a privacy policy, these tend to very long documents written in a obscure "legalesque" language meant more for protecting the websites' interests than those of users. To aid this problem, machine readable standards such as P3P have been introduced. P3P seeks to compress the information contained in the privacy policy in an XML document that can be parsed and summarized by computer programs. 

The AT&T \emph{Privacy Bird} is mentioned in T�ndel and Nyre (2010) as an example of current Internet privacy software. Privacy Bird can parse P3P documents and match a site's privacy policy with the user's preferences. The problem with this however, is that the user has to explicitly state his preferences, which is a rather time-consuming endeavor. The novel feature of this project is to introduce an intelligent system that \emph{learns} the user's preferences, seeking to limit the amount of user interference. This is to be acheived by the use of CBR agent, that can look at previous examples of user choices in similar situations. A particular advantage of the CBR approach is the feedback loop where the system can actually \emph{explain} its choice in terms of similar cases which sets CBR apart from alternative reasoning models such as artificial neural networks. It also allows for better tuning to user response in those cases that the user disagrees with the recommendation.

\section{Project Scope}
The research nature of this project makes it a very open-ended one. In particular, the direction of the project

\section{Evaluation Criteria}